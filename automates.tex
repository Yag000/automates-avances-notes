\documentclass{article}
\usepackage[utf8]{inputenc}
\usepackage[T1]{fontenc}

\usepackage[french]{babel}

\usepackage{amsmath}
\usepackage{amssymb} 
\usepackage{amsthm}  
\usepackage{dsfont}
\usepackage{mathrsfs}
\usepackage{mathtools}
\usepackage{mathpartir}

\usepackage{multirow}


\usepackage{graphicx}
\usepackage{float}
\usepackage{geometry}
\usepackage{hyperref}


\usepackage{caption}

\usepackage{fancyhdr}

\usepackage[pdf, singlefile]{graphviz}
\usepackage{tikz-cd}

\usepackage{stmaryrd}

\usepackage[noend]{algpseudocode}
\usepackage{algorithmicx}

\usepackage[hyperref,paper]{knowledge}

\usepackage{enumitem}

\setlist[itemize,2]{label={$\bullet$}}

\usepackage{automates}

\knowledgeconfigure{label scope=false, notion, quotation, protect quotation={tikzcd, automata}}

\knowledgenewrobustcmd \congN {\cmdkl {\cong}}

\knowledge{relation d'équivalence}{url={https://fr.wikipedia.org/wiki/Relation_d\%27\%C3\%A9quivalence}}
    

\knowledge{alphabet}{notion}
\knowledge{mot}[mots]{notion}
\knowledge{concaténation}{notion}
\knowledge{langage}[langages]{notion}


\knowledge{expressions rationnelles}[expression rationnelle|ERat]{notion}
\knowledge{rationnel}[langage rationnel|régulier|langage régulier]{notion}
\knowledge{automate fini déterministe}[automate déterministe|AFD]{notion}
\knowledge{automate fini non déterministe}[automate non déterministe|AFN]{notion}
\knowledge{langage reconnu}[langage accepté]{notion}
\knowledge{déterminisation}[Déterminisation|determinisé|déterminisant|det|déterministe|déterminise|déterminiser]{notion}
\knowledge{accesible}{notion}
\knowledge{chemin}{notion}


\knowledge{morphisme d'automates}{notion}
\knowledge{quotient}[quotients]{notion}
\knowledge{automate des quotients}[automate des résiduels]{notion}
\knowledge{automate minimal}[automate déterministe minimal|minimal]{notion}
\knowledge{automate miroir}[mirr]{notion}
\knowledge{reach}{notion}
\knowledge{congruence}{notion}
\knowledge{Congruence de Nérode}[congruence de Nérode]{notion}


\knowledge{logique du premier ordre}{notion}
\knowledge{Prédicats numériques}{notion}
\knowledge{Formules atomiques}{notion}
\knowledge{Formules du premier ordre}{notion}
\knowledge{Quantificateur existentiel}{notion}
\knowledge{logique du second ordre}{notion}
\knowledge{variables du second ordre}{notion}
\knowledge{Formules du second ordre}{notion}
\knowledge{Quantificateur existentiel sur des ensembles}{notion}
\knowledge{Quantificateur universel}{notion}
\knowledge{interpretation}[interpret]{notion}


\knowledge{learner}[étudiant]{notion}
\knowledge{oracle}[teacher]{notion}
\knowledge{membership queries}{notion}
\knowledge{conjectures}[conjecture|conjecturer]{notion}
\knowledge{contre-exemple}{notion}
\knowledge{clos par préfixes}[clos par suffixes]{notion}
\knowledge{table}[tables]{notion}
\knowledge{row}[rows]{notion}
\knowledge{close}[closes]{notion}
\knowledge{cohérente}[cohérentes]{notion}
\knowledge{en accord}{notion}


\knowledge{bord}[bords]{notion}


\knowledge{automate de Büchi}[automates de Büchi]{notion}
\knowledge{réussi}{notion}
\knowledge{cheminB}{notion}


\knowledge{voisinage}[voisinages]{notion}
\knowledge{quiescent}{notion}
\knowledge{Jardin d'Éden}[GOE | JDE | Jardins d'Éden]{notion}
\knowledge{orphelin}{notion}


\begin{document}

\titlepageY{Automates Avancés et Applications} {Yago \textsc{Iglesias}} { }

\tableofcontents

\newpage

\section{Introduction}

Ce document est un recueil de notes du cours d'Automates Avancés et Applications de niveau M1. Il est
basé sur les cours de Mme.~\textsc{Daniela Petrisan} et M.~\textsc{Roberto Mantaci} à Université Paris Cité, cependant toute
erreur ou inexactitude est de ma responsabilité.

Ce document a été rédigé principalement par \textsc{Yago Iglesias}, mais tout contributeur peut être retrouvé dans
la section contributeurs du répertoire \href{https://github.com/Yag000/automates-avances-notes/graphs/contributors}{GitHub}.
Un remerciement particulier est adressé à \textsc{Erin Le Boulc’h} pour sa participation active à la rédaction et correction
de ce document.

Dans ces notes, les notions sont reliées à leurs définitions grâce à la librairie LaTeX \href{https://ctan.org/pkg/knowledge?lang=en}{knowledge}.
La lecture en version électronique est donc recommandée afin de profiter pleinement des liens interactifs.

\section{Langages Rationnels}


\digraph[scale=0.5]{langage}{
	rankdir=TB;

	Node0 [shape=ellipse, style=filled, color=lightblue, label=<Langages<br/>rationnels>];
	Node1 [shape=ellipse, style=filled, color=lightgreen, label=<Automates finis<br/>déterministes>];
	Node2 [shape=ellipse, style=filled, color=lightgreen, label=<Automates finis<br/>non déterministes<br/>+ ε-transitions>];
	Node3 [shape=ellipse, style=filled, color=lightgreen, label=<Expressions<br/>rationnelles>];
	Node4 [shape=ellipse, style=filled, color=lightgreen, label=<Logique monadique<br/>du second ordre>];
	Node5 [shape=ellipse, style=filled, color=lightgreen, label=<Théorie<br/>des monoïdes>];

	Node0 -> Node1;
	Node0 -> Node2;
	Node0 -> Node3;
	Node0 -> Node4;
	Node0 -> Node5;
}

\subsection{Définitions de base}

\begin{definition}[alphabet]
	Un \textbf{alphabet} est un ensemble fini de lettres ou de symboles.
\end{definition}

\begin{definition}[mot]
	Un \textbf{mot} sur un alphabet $\alphabet$ est une séquence de lettres de $\alphabet$.
	On écrit $\motDecomp w n$ où $w_i \in \alphabet, \forall i \in \{1, \ldots, n\}$.
	La longueur d'un mot $w$ est notée $\len w = n$, pour $\motDecomp w n$.
\end{definition}

\begin{notation}
	On note $\motvide$ le mot vide et $\mots$ l'ensemble des mots sur $\alphabet$.
\end{notation}

\begin{definition}[concaténation]
	La \textbf{concaténation} de deux mots $w$ et $v$ est notée $wv$.
	Si $\motDecomp w n$ et $\motDecomp v m$, alors $wv = \decomp w n \decomp v m$.
\end{definition}

\begin{definition}[langage]
	Un langage sur un alphabet $\alphabet$ est un sous-ensemble de $\mots$.
\end{definition}

\begin{definition}[concaténation de langages]
	Soient $L_1, L_2$ deux langages sur $\alphabet$, leur \textbf{concaténation} est le langage
	$$
		L_1L_2 = \setdef {w_1w_2} {w_1 \in L_1, w_2 \in L_2}
	$$
\end{definition}

\begin{exemple}
	$$ \alphabet = \set{a, b}, \quad L = \set{\red a, \red {ab}, \red{bb}}, \quad K = \set{\blue b, \blue {ab}} $$
	$$ LK = \set{\red a \blue b, \red {ab} \blue b, \red {bb} \blue b, \red a \blue{ab}, \red {ab} \blue {ab}, \red {bb} \blue {ab}} $$
\end{exemple}


\begin{definition}[étoile de Kleene]
	Soit $L$ un langage sur $\alphabet$, son \textbf{étoile de Kleene} est le langage
	$$
		\kleene L = \bigcup_{n \in \N} L^n
	$$
	où $L^0 = \set\motvide$ et $L^{n+1} = LL^n$.
\end{definition}

\begin{remarque}
	Est-ce que $\kleene L = \setdef {w^n} {w \in L, n \in \N}$ ?

	Non, on peut trouver un contre-exemple avec $L = \set{a, b}$.

	On a bien que $ab \in \kleene L$, mais $ab \notin \setdef {w^n} {w \in L, n \in \N} = \setdef {a^n} {n \in \N} \cup \setdef {b^n} {n \in \N}$.
\end{remarque}

\begin{remarque}
	Est-ce que $L(M \cap N) = LM \cap LN$ ?

	Non, on peut trouver un contre-exemple avec $L = \set{a, ab}$, $M = \set{b}$ et $N = \set{\motvide}$.

	On a que $M \cap N = \emptyset$, donc $L(M \cap N) = \emptyset$.

	Mais $LM = \set{ab, abb}$ et $LN = \set{a, ab}$, donc $LM \cap LN = \set{ab} \neq \emptyset$.
\end{remarque}

\begin{exercice}
	Montrer que la concaténation de langages est distributive par rapport à l'union, \ie que $L(M \cup N) = LM \cup LN, \forall L, M, N$ langages.
\end{exercice}

\begin{proof}
	\begin{eqnarray*}
		w \in L(M \cup N) &\iff& \exists w_L \in L, \exists w_{M \cup N} \in M \cup N, w = w_Lw_{M \cup N} \\
		&\iff& \exists w_L \in L, (\exists w_M \in M, w = w_Lw_M \lor \exists w_N \in N, w = w_Lw_N) \\
		&\iff& (\exists w_L \in L, \exists w_M \in M, w = w_Lw_M) \lor (\exists w_L \in L,\exists w_N \in N, w = w_Lw_N) \\
		&\iff& w \in LM \lor w \in LN  \\
		&\iff& w \in (LM \cup LN)
	\end{eqnarray*}
\end{proof}


\subsection{Langages rationnels : définitions}

\begin{definition}[langage rationnel]
	Soit $\alphabet$ un alphabet fini, l'ensemble \eratsym \ des expressions rationnelles sur $\alphabet$ est défini comme suit :
	\begin{itemize}
		\item $\motvide \in \eratsym$
		\item $\emptyset \in \eratsym$
		\item $\forall a \in \alphabet, a \in \eratsym$
		\item $\forall E, F \in \eratsym, E + F \in \eratsym$
		\item $\forall E, F \in \eratsym, EF \in \eratsym$
		\item $\forall E \in \eratsym, \kleene E \in \eratsym$
	\end{itemize}
\end{definition}

\begin{definition}[sémantique des expressions rationnelles]
	Soit $r \in \erat$, on définit le langage $\lang r$ associé à $r$ par induction sur la structure de $r$ :
	\begin{itemize}
		\item $\lang \motvide = \set\motvide$
		\item $\lang \emptyset = \emptyset$
		\item $\lang a = \set a$
		\item $\lang {E + F} = \lang E \cup \lang F$
		\item $\lang {EF} = \lang E \lang F$
		\item $\lang {\kleene E} = \kleene {\lang E}$
	\end{itemize}
\end{definition}

\begin{exemple}
	$ \alphabet = \set{a, b}, \quad r = \kleene {(a + b)} a \in \erat $, alors
	$ \lang r = \setdef {wa} {w \in \mots} $
\end{exemple}


\begin{definition}[langage rationnel]
	Un langage $L$ sur un alphabet $\alphabet$ est dit \textbf{rationnel} s'il existe une \exprat $r \in \erat$ \tlq $L = \lang r$.

\end{definition}

\begin{exemple}
	$\setdef {a^n} {n \in \N}$ est un langage rationnel engendré par l'expression rationnelle $\kleene a$.

	Cependant, $\setdef {a^n b^n} {n \in \N}$ n'est pas un langage rationnel.
\end{exemple}


\section{Automates finis}

\subsection{Automates finis déterministes}

\begin{definition}[automate fini déterministe]
	Soit $\alphabet$ un alphabet, un \textbf{automate fini déterministe} (AFD) est un tuple $\AFD$ où
	\begin{itemize}
		\item $Q$ est un ensemble fini d'états
		\item $q_0 \in Q$ est appelé l'état initial
		\item $F \subseteq Q$ est l'ensemble des états finaux / acceptants
		\item $\delta : Q \times \alphabet \to Q$ est la fonction de transition
	\end{itemize}
\end{definition}


\begin{exemple} Un automate fini déterministe :

	\vspace{0.5cm}
	\begin{minipage}{0.5\textwidth}
		\begin{itemize}
			\item $\alphabet = \set{a, b}$
			\item $Q = \set{q_0, q_1}$
			\item $q_0$ est l'état initial
			\item $F = \set{q_0, q_1}$
			\item $\delta :
				      \left\{
				      \begin{array}{cc}
					      (q_0, a) & \mapsto q_0 \\
					      (q_0, b) & \mapsto q_1 \\
					      (q_1, a) & \mapsto q_0 \\
					      (q_1, b) & \mapsto q_1 \\
				      \end{array}
				      \right.$
		\end{itemize}
	\end{minipage}
	\begin{minipage}{0.5\textwidth}
		\begin{automata}
			\digraph[scale=0.75]{automateDet}{
				rankdir=LR;

				node [shape=circle, style=filled, color=lightblue];
				q0 [label="q_0"];
				q1 [label="q_1"];

				start [shape=point];
				start -> q0;

				q0 -> q0 [label="a"];
				q0 -> q1 [label="b"];
				q1 -> q1 [label="b"];
				q1 -> q0 [label="a"];

				q0 [shape=doublecircle];
				q1 [shape=doublecircle];
			}
		\end{automata}
	\end{minipage}
\end{exemple}

\begin{definition}[lecture d'un mot par un AFD]
	Soit $A = \AFD$ un AFD qui a pour alphabet $\alphabet$. On définit la fonction $\kleene \delta$ par induction sur la longueur du mot $w$ :
	$$ \begin{array}{rcl}
			\kleene \delta : Q \times \mots & \to     & Q                               \\
			(q, \motvide)                   & \mapsto & q                               \\
			(q, wa)                         & \mapsto & \delta(\kleene \delta(q, w), a)
		\end{array} $$
	On a alors que $\kleene \delta(q, w)$ est l'état atteint par $A$ après avoir lu le mot $w$ depuis l'état $q$.
\end{definition}

\begin{definition}[langage reconnu par un AFD]
	Le langage reconnu / accepté par un AFD $A = \AFD$ est le langage
	$$ \lang A = \setdef w {w \in \mots, \kleene \delta(q_0, w) \in F} $$
\end{definition}

\begin{exemple}
	Pour l'automate $A$ suivant, avec $\alphabet = \set {a,b}$, le langage reconnu est $\lang A = \lang {\kleene {(a + b)} b} = \lang {\kleene {(a + b)} b \kleene b} = \setdef {wb} {w \in \mots}$.

	\begin{center}
		\begin{automata}
			\digraph[scale=0.75]{automateDet2}{
				rankdir=LR;

				node [shape=circle, style=filled, color=lightblue];
				q0 [label="q_0"];
				q1 [label="q_1"];

				start [shape=point];
				start -> q0;

				q0 -> q0 [label="a"];
				q0 -> q1 [label="b"];
				q1 -> q1 [label="b"];
				q1 -> q0 [label="a"];

				q1 [shape=doublecircle];
			}
		\end{automata}
	\end{center}
\end{exemple}

\subsection{Automates finis non déterministes \darkgreen{+ $\motvide$-transitions}}

Dans cette partie, on traite en meme temps les cas avec et sans $\motvide$-transitions. Les additions en
\darkgreen{vert} correspondent à la définitions avec $\motvide$-transitions. En ignorant cela, on retrouve la
definition d'un automate fini non déterministe.

\begin{definition}[automate fini non déterministe \darkgreen{+ $\motvide$-transitions}]
	Soit $\alphabet$ un alphabet, un \textbf{automate fini non déterministe} (AFN) est un tuple $\AFN$ où
	\begin{itemize}
		\item $Q$ est un ensemble fini d'états
		\item $I \subseteq Q$ est l'ensemble des états initiaux
		\item $F \subseteq Q$ est l'ensemble des états finaux / acceptants
		\item $\delta : Q \times (\alphabet \darkgreen {\ \cup \set\motvide}) \to \mathcal{P}(Q)$ est la fonction de transition
	\end{itemize}
\end{definition}


\begin{exemple}
	Un automate fini non déterministe avec $\alphabet = \set{a, b}$ :
	\begin{center}
		\begin{automata}

			\digraph[scale=0.65]{automateNonDet}{
				rankdir=LR;

				node [shape=circle, style=filled, color=lightblue];
				q0 [label="q_0"];
				q1 [label="q_1"];
				q2 [label="q_2"];

				start [shape=point];
				start -> q0;

				q0 -> q0 [label="b"];
				q0 -> q1 [label="a"];
				q1 -> q1 [label="b"];
				q1 -> q0 [label="b"];
				q0 -> q2 [label="a"];
				q2 -> q2 [label="b"];

				q1 [shape=doublecircle];
				q2 [shape=doublecircle];
			}
		\end{automata}
	\end{center}
\end{exemple}


\begin{definition}[lecture d'un mot par un AFN]
	Soit $A = \AFN$ un AFN qui a pour alphabet $\alphabet$. On définit la fonction $\kleene \delta$ par induction sur la longueur du mot $w$ :
	$$ \begin{array}{rcl}
			\kleene \delta : Q \times \mots & \to     & \parts Q                                                  \\
			(q, \motvide)                   & \mapsto & \set q                                                    \\
			(q, wa)                         & \mapsto & \bigcup\limits_{p \in \kleene \delta (q, w)} \delta(p, a)
		\end{array} $$
\end{definition}


\begin{definition}[langage reconnu par un AFN]
	Le langage reconnu / accepté par un AFN $A = \AFN$ est le langage
	$$ \lang A = \setdef {w \in \mots} {\exists q_0 \in I, \kleene \delta(q_0, w) \cap F \neq \emptyset} $$

\end{definition}

\subsection{Déterminisation d'un AFN}

Soit $A = \AFN$ un AFN, on considère l'AFD $A' = \antuple{\parts Q, I, F', \delta'}$ où
\begin{itemize}
	\item $F' = \setdef {q \in \parts Q} {q \cap F \neq \emptyset}$
	      \vspace{0.25cm}
	\item $ \begin{array}{rcl}
			      \delta' : \parts Q \times \alphabet & \to     & \parts Q                              \\
			      (Q, a)                              & \mapsto & \bigcup\limits_{p \in Q} \delta(p, a)
		      \end{array} $
\end{itemize}

\vspace{0.25cm}

Ce processus est appelé \textbf{déterminisation} d'un AFN et nous permet de transformer un AFN en un AFD équivalent.

\begin{exemple}
	Déterminisation d'un automate non déterministe :

	\begin{minipage}{0.4\textwidth}
		\begin{center}
			Automate non déterministe :

			\begin{automata}
				\digraph[scale=0.65]{automateNonDet2}{
					rankdir=LR;

					node [shape=circle, style=filled, color=lightblue];
					q0 [label="q_0"];
					q1 [label="q_1"];

					start [shape=point];
					start -> q0;

					q0 -> q0 [label="a"];
					q0 -> q1 [label="a,b"];
					q1 -> q1 [label="a,b"];

					q1 [shape=doublecircle];
				}
			\end{automata}
		\end{center}
	\end{minipage}
	\begin{minipage}{0.6\textwidth}
		\begin{center}
			Automate déterminisé :

			\begin{automata}
				\digraph[scale=0.5]{automateDet3}{
				rankdir=LR;

				node [shape=circle, style=filled, color=lightblue];
				q0 [label="{q_0}"];
				q0q1 [label="{q_0, q_1}"];
				q1 [label="{q_1}"];

				start [shape=point];
				start -> q0;

				q0 -> q0q1 [label="a"];
				q0q1 -> q0q1 [label="q"];
				q0q1 -> q1 [label="b"];
				q0 -> q1 [label="b"];
				q1 -> q1 [label="a,b"];

				q0q1 [shape=doublecircle];
				q1 [shape=doublecircle];
				}
			\end{automata}
		\end{center}
	\end{minipage}
\end{exemple}


\begin{theorem}
	Soit $A$ un automate fini non déterministe avec des $\motvide$-transitions, alors il existe un automate fini non déterministe $A'$, \tq $\lang A = \lang {A'}$.
\end{theorem}

\subsection{Équivalence entre expressions rationnelles et automates finis déterministes}

\begin{theorem}[Injection des expressions rationnelles vers les automates finis déterministes]
	Soit $r \in \erat$, alors il existe un automate fini déterministe $N(r)$ \tq $\lang A = \lang r$.
\end{theorem}

\begin{proof}
	Nous allons construire un tel automate par induction sur la structure de $r$ cependant la preuve du fait que cet automate reconnaît le langage associé à $r$ est omise.
	Cette construction est appelée \textbf{construction de Thompson}.

	\begin{twoautomata}
		\digraph[scale=0.5]{thompson1}{
			rankdir=LR;
			node [shape=circle, style=filled, color=lightblue];
			q0 [label=" "];
			q1 [label=" "];
			start [shape=point];
			start -> q0; q0 -> q1 [label="ε"];
			q1 [shape=doublecircle];
		}
		\caption*{$N(\motvide)$}
	\end{twoautomata}
	\begin{twoautomata}
		\digraph[scale=0.5]{thompson2}{
			rankdir=LR;
			node [shape=circle, style=filled, color=lightblue];

			start [shape=point];
			q0 [label=" "];
			q1 [label=" ", shape=doublecircle];

			start -> q0;
			q0 -> q1 [style=invis];
		}
		\caption*{$N(\emptyset)$}
	\end{twoautomata}

	\begin{twoautomata}
		\digraph[scale=0.5]{thompson3}{
			rankdir=LR;
			node [shape=circle, style=filled, color=lightblue];
			q0 [label=" "];
			q1 [label=" "];
			start [shape=point];
			start -> q0; q0 -> q1 [label="a"];
			q1 [shape=doublecircle];
		}
		\caption*{$N(a), a \in \alphabet$}
	\end{twoautomata}
	\begin{twoautomata}
		\digraph[scale=0.5]{thompson4}{
		rankdir=LR;

		node [shape=circle, style=filled, color=lightblue];

		subgraph cluster_r {
		style=rounded;
		label="N(r)";
		color=lightcoral;

		nrin [label=" "];
		nrout [label=" "];
		nrin -> nrout [style=invis];
		}


		subgraph cluster_s {
		style=rounded;
		label="N(s)";
		color=lightgreen;

		nsin [label=" "];
		nsout [label=" "];
		nsin -> nsout [style=invis];
		}


		q0 [label=" "];
		q1 [label=" "];
		start [shape=point];

		start -> q0;
		q0 -> nrin [label="ε"];
		q0 -> nsin [label="ε"];

		nrout -> q1 [label="ε"];
		nsout -> q1 [label="ε"];

		q1 [shape=doublecircle];
		}

		\caption*{$N(r + s)$}
	\end{twoautomata}

	\begin{twoautomata}
		\digraph[scale=0.5]{thompson5}{
		rankdir=LR;

		node [shape=circle, style=filled, color=lightblue];

		subgraph cluster_r {
		style=rounded;
		label="N(r)";
		color=lightcoral;

		nrin [label=" "];
		nrout [label=" "];
		nrin -> nrout [style=invis];
		}

		subgraph cluster_s {
		style=rounded;
		label="N(s)";
		color=lightgreen;

		nsin  [label=" "];
		nsout [label=" "];
		nsin -> nsout [style=invis];
		}

		start [shape=point];

		start -> nrin;
		nrout -> nsin [label="ε"];
		nsout [shape=doublecircle];
		}
		\caption*{$N(rs)$}
	\end{twoautomata}
	\begin{twoautomata}
		\digraph[scale=0.5]{thompson6}{
		rankdir=LR;

		node [shape=circle, style=filled, color=lightblue];

		subgraph cluster_r {
		style=rounded;
		label="N(r)";
		color=lightcoral;

		nrin [label=" "];
		nrout [label=" "];
		nrin -> nrout [style=invis];
		}

		start [shape=point];

		q0 [label=" "];
		q1 [label=" "];
		start [shape=point];
		start -> q0;
		q0 -> q1 [label="ε"];
		q0 -> nrin [label="ε"];
		nrout -> nrin [label="ε"];
		nrout -> q1 [label="ε"];

		q1 [shape=doublecircle];
		}

		\caption*{$N(\kleene r)$}
	\end{twoautomata}

	En combinant ces constructions, on peut construire un automate fini non déterministe pour n'importe quelle
	expression rationnelle qui peut être determinisé pour obtenir un automate fini déterministe.

\end{proof}

\section{Minimisation d'automates}

\subsection{Morphismes d'automates}

\begin{definition}[Morphisme d'automates]
	Soit $A = \antuple{Q, q_0, F, \delta}$ et $ A' = \antuple{Q', q_0', F', \delta'} $ sur un alphabet $\alphabet$.
	Alors un morphisme d'automates $\phi: A \to A'$ est une fonction $\phi: Q \to Q'$ \tlq
	\begin{enumerate}
		\item $\phi (q_0) = q_0'$ \label{morph:1}
		\item $\forall q \in Q,\quad  \phi (q) \in F' \iff q \in F$ \label{morph:2}
		\item $\forall q \in Q, \forall a \in \alphabet,\quad \delta'(\phi (q), a) = \phi (\delta(q,a))$ \ie $\phi \circ \delta_a = \delta_a '\circ \phi$ \label{morph:3}
	\end{enumerate}

	% https://tex.stackexchange.com/questions/218274/how-can-i-draw-commutative-diagrams-in-latex
	\[
		\begin{tikzcd}
			Q \arrow{r}{\delta_a} \arrow[swap]{d}{\phi} & Q \arrow{d}{\phi} \\
			Q' \arrow{r}{\delta_a} & Q'
		\end{tikzcd}
	\]

\end{definition}


\begin{exercice}
	Soit $\phi : A \to A'$ un morphisme d'automates déterministes, alors $\lang A = \lang {A'}$
\end{exercice}

\begin{proof}
	\begin{eqnarray*}
		w \in \lang A &\iff& \kleene \delta (q_0, w) \in F \\
		&\iff& \phi (\kleene \delta (q_0, w)) \in F' \quad \text{par } \ref{morph:1} \\
		&\iff& \kleene \delta( \phi (q_0), w) \in F' \quad \text{par } \ref{morph:3} \\
		&\iff& \kleene \delta( q_0', w) \in F' \quad \text{par } \ref{morph:2} \\
		&\iff& w \in \lang {A'}
	\end{eqnarray*}
\end{proof}


\begin{definition}[Quotient d'un automate]
	Soit $L \subseteq \mots$ et $w \in \mots$. Le quotient $\quot w L$ est défini par
	$$ \quot w L = \setdef {u \in \mots} {wu \in L} $$
\end{definition}

\begin{definition}
	Soit A = $\AFD$ sur l'alphabet $\alphabet$ et $q \in Q$ alors on définit
	$$ L_q = \setdef {w \in \mots} {\kleene \delta (q,w) \in F} $$
\end{definition}


\begin{lemma}
	Soit A = $\AFD$ un automate fini déterministe, $q \in Q$ et $w \in \mots$. Si $\deltaS {q_0} w = q$ alors
	$L_q = \quot w \lang A$
\end{lemma}

\begin{proof}
	\begin{eqnarray*}
		u \in L_q &\iff& \deltaS q u \in F \\
		&\iff& \deltaS {\deltaS {q_0} w} u \in F \\
		&\iff& \deltaS {q_0} {wu} \in F \\
		&\iff& wu \in \lang A \\
		&\iff& u \in \quot w \lang A
	\end{eqnarray*}
\end{proof}

\begin{coro}
	Si $L \subseteq \mots$ est un langage régulier alors l'ensemble $\setdef {\quot w L} {w \in \mots}$ est fini.
\end{coro}

\begin{proof}
	Si $L = \lang A$, avec A = $\AFD$, alors $\abs {\setdef {\quot w L} {w \in \mots}} \leq \abs Q$. Car $\forall w \in \mots,$ $\quot w L$ est
	le langage accepté par $A$ à partir de l'état $\deltaS {q_0} w$.
\end{proof}

\begin{coro} \label{coro:2}
	Si $\AFN$ accepte un langage $L$ alors $\abs Q \geq $ \# quotients de $L$
\end{coro}

\begin{definition} [Automate des quotients]

	Soit $L \subseteq \mots$ un langage régulier, soit $Q = \setdef {\quot w L} {w \in \mots}$ et $\AFD$. Alors l'automate des quotients de $L$
	est défini par

	\begin{itemize}
		\item L'état initial est $\quot {\motvide} L$
		\item $F = \setdef {\quot w L} {w \in L}$
		\item $\delta (\quot w L, a) = \quot {(wa)} L$
	\end{itemize}
\end{definition}

\begin{remarque}
	Si $\quot w L = \quot {(w')} L$, alors $\quot {(wa)}L = \quot {(w'a)} L$.

	Soit $w,w' \in \mots$ \tq $\quot w L = \quot {(w')} L$. Soit $a \in \alphabet$ et $u \in \mots$,
	\begin{eqnarray*}
		u \in \quot {(wa)}L &\iff& wau \in L \\
		&\iff& au \in \quot w L \\
		&\iff& au \in \quot {w'} L \\
		&\iff& w'au \in L \\
		&\iff& (w'a)u \in L \\
		&\iff& u \in \quot {(w'a)}L
	\end{eqnarray*}
	Donc la fonction de transition $\delta$ de l'automate des quotients est bien définie.
\end{remarque}

\begin{prop}
	Étant donné un automate $A$ qui accepte le langage $L$, l'automate des quotients accepte aussi $L$ et il est minimal parmi les automates acceptant $L$.
\end{prop}

\begin{proof}
	$\deltaS L w = \quot w L$ et donc $\underbrace{\deltaS L w \in F}_{w \in \lang A} \iff w \in L$

	Et ainsi, par le corollaire \ref{coro:2}, l'automate des quotients est minimal.
\end{proof}

\begin{lemma}\label{lem:reach}
	Soit $L \subseteq \mots$ un langage régulier. Soit $A_L$ l'automate des quotients de $L$ et soit $B$ un autre automate acceptant $L$.
	Soit $\reach B$ un sous-automate accessible de $B$. Alors on a un morphisme surjectif (un quotient) d'automates $\reach B \twoheadrightarrow A_L$.

	\begin{tikzcd}[row sep=large]
		&\reach B \arrow[dr, hook] \arrow[dl, twoheadrightarrow] \\
		A_L & & B
	\end{tikzcd}

	Si $B = <T,t_0,F,\delta>$ alors $\reach B =  <T',t_0,F',\delta'>$ où
	\begin{itemize}
		\item $T' = \setdef {t \in T} {\exists w \in \mots, \deltaS {t_0} w = t}$
		\item $F' = F \cap T'$
		\item $\delta ' = \delta \cap (T' \times \alphabet \times T')$
	\end{itemize}
\end{lemma}


\begin{exercice}
	On définit $\phi : \reach B \to A$.

	Si $t \in T', \exists w \in \mots$ \tq $ t = \deltaS {t_0} w$
	On définit $\phi (t) = \quot w L$

	Montrer que $\phi$ est bien définie, \ie, $\phi (t) = \quot {w'} L \implies \quot w L = \quot {w'} L$.

	Vérifions $\phi$ surjective. Si $w \in \mots$ et $\quot w L$ est un état de $A_L$ alors $\phi (\deltaS {t_0} w) = \quot w L$

	Si $\abs B = \abs {A_L}$, alors $B$ est un automate accesible
	(Sinon $\abs A \leq \abs {\reach B} < \abs B \lightning)$.

	Donc $B = \reach B$

	De plus, le morphisme $\phi$ est une bijection et $\quot {\phi}$ est un morphisme d'automates.
	Donc $B \cong A_L$.

\end{exercice}


\subsection{Minimisation d'automates}

\begin{rappel}
	Pour tout langage rationnel $L$, il existe un unique (à isomorphisme près) automate déterministe minimal
	(avec le plus petit nombre possible d'états) qui reconnait $L$.
\end{rappel}

\begin{remarque}
	Ce n'est pas vrai pour les automates non-déterministes.


	\begin{twoautomata}
		\digraph[scale=0.5]{minex310}{
			rankdir=LR;

			node [shape=circle, style=filled, color=lightblue];
			q0 [label="q_0"];
			q1 [label="q_1"];
			q2 [label="q_1"];

			start [shape=point];

			start -> q0;
			q0 -> q1 [label="b"];
			q1 -> q1 [label="a"];
			q1 -> q2 [label="a"];
			q2 -> q1 [label="b"];
			q2 [shape=doublecircle];
		}
		\caption*{$(ba^+)^+$}
	\end{twoautomata}
	\begin{twoautomata}
		\digraph[scale=0.5]{minex311}{
			rankdir=LR;

			node [shape=circle, style=filled, color=lightblue];
			q0 [label="q_0"];
			q1 [label="q_1"];
			q2 [label="q_1"];

			start [shape=point];

			start -> q0;
			q0 -> q1 [label="b"];
			q1 -> q2 [label="a"];
			q2 -> q2 [label="a"];
			q2 -> q1 [label="b"];
			q2 [shape=doublecircle];
		}
		\caption*{$(ba^+)^+$}

	\end{twoautomata}
\end{remarque}

L'automate des résiduels de L est l'automate minimal reconnaissant L.


Nous allons voir trois algorithmes de minimisation différents:
\begin{itemize}
	\item Algorithme de Brzozowski
	\item Algorithme de Moore
	\item Algorithme de Hopcroft
\end{itemize}


\subsubsection{Algorithme de Brzozowski}

\begin{prop}
	Si $A$ est un automate déterministe et accesible, \ie, $\forall q \in Q, \exists \text{ un chemin } q_0 \to q$,
	et  $A^ {\sim} = \detA (\mirr(A)) $ est l'automate obtenu en déterminisant l'automate miroir de $A$. Alors
	$A^{\sim}$ est l'automate minimal qui reconnait $\mirror {\lang A}$.
\end{prop}


\begin{proof}
	Soit $A = (Q, q_0, T)$, $L = \lang A$. $\mirr (A) = (Q, T, q_0), M = \lang {\mirr (A)} = \mirror L$.

	On note $X \cdot u$ et $\delta (X,u)$, $X$ étant un ensemble d'états et $u$ un mot, l'ensemble $\bigcup\limits_{q \in X} \delta (q,u)$.

	Nous allons montrer que $A^{\sim}$ est l'automate minimal pour $M$. Pour cela, il suffit de montrer que $A^{\sim}$
	est isomorphe à l'automate des résiduels de $M$.

	Pour cela, il faut \mq si $u$ et $v$ sont deux mots quelconques \tq $\quot u M = \quot v M$, alors $T \cdot u =  T \cdot v$ dans $A^{\sim}$.

	Soit $p$ un état de $T \cdot u$. Puisque $A$ est accesible, il existe un chemin de $q_0 \to p$ dans $A$, et donc il existe un chemin $p \to q_0$ dans $\mirr(A)$.
	Soit $w$ l'étiquette de ce chemin, alors le mot $uw$ est l'étiquette d'un chemin réussi de $\mirr (A)$. En conséquence, $uw \in M$.

	Mais $uw\in M \iff w \in \quot u M \iff w\in \quot v M \iff vw \in M$.

	Donc $vw$ est l'étiquette d'un chemin réussi $\Gamma$ dans $\mirr (A)$. Étant donné que $A$ est déterministe, tout chemin aboutissant dans $q_0$
	et dont l'étiquette a $w$ comme suffixe doit passer par $p$.

	On en  déduit que $p \in Tv$.
\end{proof}


\begin{coro}
	Soit $A$ un automate, alors l'automate $\detA (\mirr (\detA(\mirr (A))))$ est l'automate minimal pour $\lang A$.
\end{coro}

\begin{proof}
	En effet,

	$\underbrace{\detA (\mirr (
			\underbrace{\detA(\mirr (A))}_{\text{Un automate déterministe et accesible qui reconnait }\mirror {\lang A}}
			))}_{\text{la proposition précédente garantit que cet automate est minimal pour }  \mirror{\mirror {\lang A} } = \lang A}
	$
\end{proof}

\begin{complexite}
	La complexité de l'algorithme est en $O(2^n)$. En effet, la dernière détermination exécutée travaille sur un automate de taille au plus $2^n$
	et produit un automate de taille $2^n$. Sa complexité reste aussi en $O(2^n)$.
\end{complexite}


\begin{remarque}
	Alors que d'autres algorithmes, \tq Moore, ont une complexité polynomiale, Brzozowski à différence des autres, ne nécessite pas
	que l'automate donné soit déterministe.
\end{remarque}

\subsubsection{Algorithme de Moore}


\begin{definition}[Congruence d'automates]
	Soit $A = (Q,q_0,T,S)$ un automate déterministe et soit $\sim$ une relation d'équivalence définie sur l'ensemble $Q$. On dit que
	$\sim$ est une congruence si $\sim$ satisfait les conditions suivantes :
	\begin{enumerate}
		\item Compatibilité aves les transitions: Si $q \sim q'$ alors $\forall a \in \alphabet, \delta (q,a) \sim \delta (q',a)$
		\item Saturation de $A$: Si $q \sim q'$ alors $q \in T \iff q' \in T$
	\end{enumerate}
\end{definition}


\begin{definition}
	Si $A = \AFD$ est un AFD et $\sim$ une congruence définie sur $Q$. On définit l'automate quotient :
	$$ A/\sim = (Q',q_0',T',\delta') $$

	avec \begin{itemize}
		\item $Q' = \setdef {[q]} {q\in Q}$ (chaque état est étiqueté par une classe d'équivalence)
		\item $q_0' = [q_0]$
		\item $T' =\setdef  {[q]} {q \in T}$
		\item $\delta'([q], a) = [p]$ \ssi $p \in [qa]$
	\end{itemize}

\end{definition}

\begin{prop}
	Si $A$ est un automate déterministe et $\sim$ une congruence sur $A$, alors $\lang A = \lang {A/\sim}$.
\end{prop}


\begin{definition}[Congruence de Nerode]

	Si $A = (Q, \set {q_0}, T, \delta)$ un AFD, $\forall q, q' \in Q$ on définit $q \cong q' \iff L_q = L_{q'}$.
\end{definition}

\begin{rappel}
	$L_q = \setdef w {\deltaS q w \in T}$.
	Donc
	\begin{eqnarray*}
		q \cong q' \iff L_q = L_{q'} &\iff& \forall w \in \mots, w \in L_q \mssi w \in L_{q'} \\
		&\iff& \forall w, \deltaS q w \in T \mssi  \deltaS {q'} w \in T
	\end{eqnarray*}

	On en déduit que
	\begin{equation}\label{eq:congnot}
		q \not\cong q' \iff \exists w, \deltaS q w \in T \et \deltaS {q'} w \notin T
	\end{equation}
	où $\deltaS q w \notin T$ et $\deltaS{q'} w \in T$ sont deux états non équivalents, dits \emph{séparables}.
\end{rappel}

Si $q$ et $q'$ sont séparables et $w$ est un mot qui satisfait \ref{eq:congnot}, on dira que $w$ \emph{sépare} $q$ et $q'$.


Pour calculer la congruence de Nérode on introduit une famille de congruences :

\begin{equation}
	\cong_i, i  \in \N, q \cong_i q' \iff \forall w, \abs w \leq i \implies \deltaS q w \in T \iff  \deltaS {q'} w \in T
\end{equation}


\begin{definition}
	Définition alternative

	\begin{itemize}
		\item $q \cong_0 q' \mssi q\in T \iff q' \in T$
		\item $q \cong_{i+1} q' \mssi q \cong_{i} q' \et  \forall a \in \alphabet, \delta (q,a) \cong_i \delta (q', a)$
	\end{itemize}
\end{definition}

\begin{remarque}
	Par définition $\cong_{i+1}$ induit une partition de $Q$ au moins aussi fine que celle induite par $\cong_i$.

	Les classes de $\cong_{i+1}$ sont obtenues en partitionnant des classes de $\cong_i$.
\end{remarque}

\begin{remarque}
	Puisque $Q$ est fini, le processus de partitionnement doit se stabiliser. Autrement dit,

	$$\exists k \in \N^*, \text{\tq} \cong_{k} \ = \ \cong_{k+j} \forall j \in \N^* $$
\end{remarque}


\begin{prop}
	Si $k \in \N$ \tq $ \cong_k \ = \ \cong_{k+j} \forall j \in \N$, alors $\cong_k \ = \ \cong$
\end{prop}

\begin{proof}
	Soit $q,q' \in Q$, $q \cong_k q'$, \mq $L_q = L_{q'}$. Exercice.
\end{proof}

\begin{definition}[Algorithme de Moore]
	L'algorithme consiste à créer un automate à partir des classes d'équivalence sur $\cong$.
	Les états sont les classes d'équivalence, on obtient les transitions en regardant le comportement
	d'un représentant de la classe et les états finaux sont les classes qui ont un représentant qui est un état final.
\end{definition}


\begin{complexite}
	L'algorithme effectue $n$ étapes et chaque étape dépense $O(n)$, donc sa complexité est en $O(n^2)$ (au pire).

	En fait, en moyenne $O(n \log n)$ et meme $O(n \log \log n)$,  \cite{David2010TheAC}
\end{complexite}



\section{Monoïdes}

\begin{definition}[Monoïde]
	Un monoïde est un tuple $(M, \times, 1)$ où $M$ est un ensemble, $\times : M \times M \to M$ est une operation binaire sur $M$,
	$1\in M$ \tq
	\begin{enumerate}
		\item $\times$ est une opération  associative ($\forall x,y,z \in M , x \times (y \times z) = (x \times y) \times z$)
		\item $1$ est un élément neutre pout $\times$, \cad, $\forall x \in M, 1 \times x = x \times 1 = x$
	\end{enumerate}
\end{definition}

\begin{exemple}
	\begin{itemize}
		\item $(\N, +, 0)$
		\item $(\mots, \cdot, \motvide)$
		\item $(\Z, +, 0)$ est un monoïde, mais aussi un groupe car tout élément possède un symétrique.
		\item Si $Q$ est un ensemble, alors $(Q ^ Q, \circ, \text{id})$
	\end{itemize}
\end{exemple}


\begin{definition}[Morphisme de monoïdes]
	Soit $(M,\times_M, 1_M)$ et $(N,\times_N, 1_N)$ des monoïdes. Une fonction $h :  M \to N$ est appelée un morphisme de monoïdes \ssi
	\begin{enumerate}
		\item $h(1_M) = 1_N \reason{$h$ préserve l'élément neutre}$
		\item $\forall x, y \in M, h (x \times_M y) = h(x) \times_N h(y) \reason{$h$ préserve la multiplication}$
	\end{enumerate}
\end{definition}


\begin{prop}
	Soit $\alphabet$ un ensemble et $(M,\cdot, 1)$ un monoïde. Soit $f : \alphabet \to M$ une fonction. Il existe
	un unique morphisme de monoïdes $\bar f : \mots \to M$ \tq $\bar f (a) =  f (a),  \forall a \in \alphabet$.

	\begin{tikzcd}
		\alphabet \arrow[r, hook]
		\arrow[d, "f", name=A]
		& \mots \arrow[dl, dashrightarrow, "\exists ! \bar f", name=B] \\
		M
	\end{tikzcd}
	%TODO: Add circular arrow
\end{prop}


\begin{proof}
	\begin{itemize}
		\item Existence:

		      Soit $w\in \mots$ de la forme $\motDecomp w n$ où $w_i \in \alphabet$.

		      Soit $\bar f (w) =^{def} f(w_1) \cdot \ldots \cdot f(w_n)$ et $\bar f (\motvide) = 1_M$. Alors pour $a\in \alphabet, \bar f (a) = f (a)$.

		      Si $\motDecomp w n$ et $\motDecomp v m \in \mots$.
		      \begin{eqnarray*}
			      \bar f(wv) = \bar f(\decomp w n \decomp v m) &=& f(w_1) \cdot \ldots \cdot f(w_n) \cdot f(v_1) \cdot \ldots \cdot f(v_m) \\
			      &=& \left( f(w_1) \cdot \ldots \cdot f(w_n)\right)\cdot\left(f(v_1) \cdot \ldots \cdot f(v_m) \right)\\
			      &=& \bar f(\decomp w n) \cdot \bar f (\decomp v m) \\
			      &=& \bar f(w) \cdot \bar f (v)
		      \end{eqnarray*}

		      Donc $\forall w, v \in \mots, \bar f (wv) = \bar f (w)\bar f(v) \et \bar f (\motvide) = 1_M$ et ainsi, $\bar f$ est un morphisme de monoïdes.


		\item Unicité:

		      Si $h : \mots \to M$ est un morphisme de monoïdes \tq $h(a) = f(a), \forall a \in \mots$. Alors $h(\motvide) = 1_M$.

		      Comme $h$ préserve la multiplication, on a
		      $h (\decomp a n) = h(a_1) \ldots h(a_n) = f(a_1) \ldots f(a_n) = \bar f (\decomp a n)$.

		      Et donc $h = \bar f$
	\end{itemize}
\end{proof}

On dit que $\mots$ est le monoïde librement engendré sur $\alphabet$

\begin{definition}[Librement engendré]
	$H$ est librement engendré sur $\alphabet$ si
	$$\forall f : \alphabet \to M \text{ où } M  \text { est un monoïde}, \exists ! \bar f : H \to M \et i : \alphabet \to H, \bar f \circ i   = f $$

	\begin{tikzcd}
		\alphabet \arrow[r, hook, "i"]
		\arrow[d, "f"]
		& H  \arrow[dl, dashrightarrow, "\exists ! \bar f"] \\
		M
	\end{tikzcd}

    Et on a donc que $ H \cong \mots$.
\end{definition}

\begin{exemple}
Montrons que $(\N, + , 0)$ est librement engendré sur $\set 1$.

	\begin{tikzcd}
		\set 1 \arrow[r, hook]
		\arrow[d, "f"]
		&  (\N, +, 0) \arrow[dl, dashrightarrow, "\exists ! \bar f"] \\
		(M,\cdot, 1_A)
	\end{tikzcd}

	Alors on a que $f(1) = m \in M$ et donc $f(n) =  f(\underbrace{1+\ldots+1}_{n \text{ fois}}) = f(1)\cdot \ldots \cdot f(1) = m\cdot \ldots \cdot m = m^n$
    Donc, si on pose $i: \set 1 \to (\N, + 0)$ \tq $f(1) = 1$, alors quelque soit le choix de $m$, \ie, quelque soit la fonction $f$ choisie (car elle dépend juste du choix $m$), alors 
    on peut toujours poser $\bar f (\N, +, 0)$ avec $f(0) = 1_M$ et $f(1) = m$ qui vérifie bien que $\bar f \circ i = f$. Ainsi, $(\N, +, 0)$ est librement engendré sur $\set 1$
\end{exemple}

\begin{remarque}
	$\N \cong \set{1}^*$
\end{remarque}

\subsection{Reconnaissance par monoïdes}

\begin{definition}[Reconnaissance par monoïdes]
	Soit $\alphabet$ un alphabet fini. Soit $L \subseteq \mots$ un langage et soit $\phi : \mots \to M$ un morphisme de monoïdes où $(M, \circ, 1_M)$ est un monoïde fini. On dit que
	$\phi$ reconnait $L$ \ssi il existe $P \subseteq M$ \tq $L = \quot {\phi} (P)$, \cad $L = \setdef {w \in \mots} {\phi (w) \in P}$.
\end{definition}

\begin{exercice}
	$\phi$ reconnaît $L$ \ssi $L = \quot {\phi} (\phi (L))$.
\end{exercice}

\begin{proof}
    \begin{itemize}
        \item $\Leftarrow$

            Soit $P = \phi(L) \subseteq M$, alors on a que $\quot {\phi} (P) = \quot {\phi} (\phi (L)) = L$ et donc $\phi$ reconnait $L$.

        \item $\Rightarrow$

            On sait qu'il existe $P$ \tq $L = \setdef {w} {w \in \mots, \phi(w) \in P}$. Regardons qu'est que c'est $\phi (L)$.
            On a que $\phi (L) = \setdef {\phi (w)} {w \in \mots, \phi(w) \in P} = P$. Et donc on a bien $\quot {\phi} (\phi (L)) = \quot {\phi} (P) = L$.
    \end{itemize}
\end{proof}

\begin{prop}
	Soit $L \subseteq \mots$ un langage sur l'alphabet $\alphabet$. Alors $L$ est reconnaissable \ssi $L$ est reconnu par un morphisme de monoïdes $\phi : \mots \to M$ où $M$ est fini.
\end{prop}


\begin{proof}
	\begin{itemize}
		\item $\Leftarrow$

		      Soit $\phi : \mots \to M$ un morphisme de monoïdes avec $M$ fini qui reconnait le langage $L$.
		      $\exists P \subseteq M, L = \quot {\phi} (P)$.

		      Soit $A$ l'automate suivant : $\antuple {M, 1_M, P, \delta}$ où

		      $$ \begin{array}{rcl}
				      \delta : M \times \alphabet & \to     & M                \\
				      (m,a)                       & \mapsto & m \cdot \phi (a)
			      \end{array} $$

		      On peut montrer par induction sur la longueur du mot que

		      $$ \begin{array}{rcl}
				      \delta ^* : M \times \mots & \to     & M        \\
				      (1_M, w)                   & \mapsto & \phi (w)
			      \end{array} $$

		      \ssi
		      \begin{eqnarray*}
			      A \text{ accepte le mot } w &\iff&  \deltaS {1_M} w \in P \iff \phi (w) \in P \\
			      &\iff&  w \in \quot {\phi}(P) \iff w \in L.
		      \end{eqnarray*}

		      Donc $A$ accepte le langage $L$


		\item $\Rightarrow$

		      Soit $A = \AFD$ un automate déterministe complet qui accepte $L$.
		      Soit

		      $$ \begin{array}{rcl}
				      \phi : \mots & \to     & Q^Q        \reason{$(Q^Q, \circ,\text{id}_Q) $ est bien un monoïde} \\
				      \phi (w)(q)  & \mapsto & \deltaS q w
			      \end{array} $$

		      On a bien :
		      \begin{itemize}
			      \item $Q^Q$ est un monoïde fini.
			      \item $\phi (\motvide) (q) = \deltaS q {\motvide} = q$, donc $\begin{array}{rcl}
					            \phi (\motvide) : Q & \to     & Q \\
					            q                   & \mapsto & q
				            \end{array} $
			            est la fonction identité sur $Q$ et donc $\phi$ preserve l'identité (élément neutre).
			      \item $\phi (w) \circ \phi (w') = \phi (ww'), \ \forall w, w' \in \mots$ car
			            $\deltaS {\deltaS q w} {w'} = \deltaS q {ww'}$.

			            Soit $P \subseteq Q^Q$ défini par $P \setdef {f : Q \to Q} {f(q_0)\in F}$.

			            \begin{eqnarray*}
				            \quot {\phi} (P) &=& \setdef {w \in \mots} {\phi (w) \in P} \\
				            &=& \setdef {w \in \mots} {\phi (w) (q_0)\in F} \\
				            &=& \setdef {w \in \mots} {\deltaS {q_0} w \in F}\\
				            &=& \setdef {w \in \mots} {w \text { est accepté par } A}\\
				            &=& L
			            \end{eqnarray*}
		      \end{itemize}

		      Donc $L$ est reconnu par $\phi$.
	\end{itemize}
\end{proof}

\begin{definition}
	On va appeler $\phi (\mots)\subseteq Q^Q$ le \textbf{monoïde de transition de l'automate} $A$.
\end{definition}


\begin{definition}[Congruence]
	Soit $(M, \cdot, 1_M)$ un monoïde. Une congruence sur $M$ est une relation d'équivalence $\sim$ $\subseteq M \times M$ \tlq
	$$\forall m,m'\in M, \ m \sim m' \iff \forall n,r \in M, n\cdot m \cdot r \sim n \cdot m' \cdot r$$
\end{definition}

\begin{definition}[Congruence syntaxique d'un langage]
	Soit $L \subseteq \mots$ un langage. Soit $\sim_L \subseteq \mots \times \mots$ la relation d'équivalence définie par :
	$$ w \sim_L w' \mssi \forall u,v \in \mots, uwv \in L \iff u w' v \in L$$
\end{definition}

\begin{prop}
	$\sim_L$ est une congruence sur $\mots$.
\end{prop}

\begin{proof}
	Exercice.
\end{proof}


\begin{prop}
	Si $\monoide$ est un monoïde et $\sim$ $\subseteq M \times M$ est une congruence sur $M$. On note $M/\sim = \setdef {[m]} {m \in M}$, où $[m]$ note la classe d'équivalence de $m$ par rapport à $\sim$.

	Alors $r/\sim$ est un monoïde et $ \begin{array}{rcl}
			h : M & \to     & M/\sim \\
			m     & \mapsto & [m]
		\end{array} $ est un morphisme de monoïdes.
\end{prop}

\begin{proof}
	$(M/\sim, \cdot, [1_M])$ est un monoïde où $[m][m'] = [mm']$.

	À montrer que cette multiplication est bien définie, \cad si $m_2 \sim m \et m_2' \sim m'$ alors $m_2 m_2' \sim m m'$ :
	\begin{eqnarray*}
		m_2 &\sim& m \\
		m_2' &\sim& m' \\
		1_M \cdot  m_2 \cdot m_2' &\sim& 1_M \cdot m \cdot m_2' \reason{car $\sim$ est une congruence} \\
		m_2 \cdot m_2' &\sim&  m \cdot m_2' \\
		\text{Mais } m \cdot m_2' &\sim& m \cdot m' \reason{car $\sim$ est une congruence} \\
		\implies m_2 \cdot m_2' &\sim& m \cdot m'
	\end{eqnarray*}

	Donc la multiplication sur $M/\sim$ est bien définie.


	$[1_M]$ est un élément neutre car $[m]\cdot [1_M] = [m \cdot 1_M] = [m]$ et pareil pour $[1_M]\cdot [m] = [m]$.

	$h$ est donc un morphisme de monoïdes:

	\begin{itemize}
		\item  $h(1_M) = [1_M]$
		\item  $h(m\cdot n) = [m \cdot n] = [m] \cdot [n] = h(m) \cdot h(n)$
	\end{itemize}
\end{proof}

\begin{remarque}
	$h$ est un \textbf{morphisme surjectif} (aussi appelé un quotient).
\end{remarque}

\begin{definition}
	Soit $L \subseteq \mots$ un langage et soit $\sim_L$ la congruence syntaxique sur $\mots$.
	Le monoïde quotient $\mots / \sim_L$ est appelé le monoïde syntaxique de $L$.
\end{definition}

\begin{prop}
	Avec les notations ci-dessus, $L$ est reconnu par le morphisme
	$ \begin{array}{rcl}
			\phi : \mots & \to     & \mots/\sim_L \\
			w            & \mapsto & [w]
		\end{array} $.
\end{prop}


\begin{remarque}\label{rem:quot_mono}
	Si $w \in L$ et $w' \sim_L w$ alors $w' \in L$, parce que  $w' \sim_L w$.

	Comme $\motvide w \motvide \in L$ , $\motvide w' \motvide$ donc $w' \in L$.

\end{remarque}

\begin{proof}
	Soit $P \subseteq \mots / \sim_L$ donné par $P = \setdef {[w]} {w \in L}$.

	On doit \mq $L = \quot {\phi} (P)$, \cad, $L = \setdef {w \in \mots} {\phi (w) \in P}$

	En utilisant la remarque \ref{rem:quot_mono} on a que si $ w\in L$ alors $[w ] \subseteq L$.

	On en conclut que
	$$ L = \bigcup_{w \in L} [w]    \iff L = \quot {\phi} (P) $$
\end{proof}


\begin{prop}
	Le monoïde syntaxique a une propriété similaire à \ref{lem:reach}:
	$\forall \monoide$ monoïde qui reconnait un langage $L$, on a le diagramme suivant:

	\begin{tikzcd}[row sep=large]
		&(N, \cdot , 1_N) \arrow[dr, hook] \arrow[dl, twoheadrightarrow] \\
		(\mots / \sim_L, \cdot,  [\motvide]) & & (M, \cdot, 1_M)
	\end{tikzcd}
\end{prop}

\begin{terminologie}
	On dit que le monoïde syntaxique de $L$ divise tout monoïde qui accepte $L$.
\end{terminologie}




\begin{lemma}
	Si $N$ divise $M$ et $N$ reconnait un langage $L$ alors $M$ reconnait $L$.

\end{lemma}

\begin{proof}

	\begin{tikzcd}[row sep=large]
		&T \arrow[dr, hook] \arrow[dl, twoheadrightarrow] \\
		N & & M
	\end{tikzcd}
	$\phi : \mots \to N $ et $P \subseteq N$ \tq $L = \quot {\phi} (P)$

	Exercice : \mq on peut construire $\phi'' : \mots \to M$ qui reconnait $L$.
\end{proof}


\begin{prop}
	Un monoïde  $M$ reconnait  un langage $L$ \ssi le monoïde syntaxique $\mots / \sim_L$ divise $M$.
\end{prop}

\begin{proof}

	\begin{itemize}
		\item $\Leftarrow$

		      $\mots / \sim_L$ reconnait $L$. Par le lemme, si $\mots / \sim_L$ divise $M$, alors $M$ reconnait $L$.

		\item $\Rightarrow$

		      Supposons que $M$ reconnait $L$. Donc on a $\phi : \mots \to M$ et $P \subseteq M$ avec $\quot {\phi} (P) = L$.

		      Soit $N = \phi (\mots)$ l'image directe de $\mots$ par $\phi$.

		      Alors $N \hookrightarrow M$ est un sous-monoïde de $M$.

		      Je voudrais avoir un quotient $h : N \twoheadrightarrow \mots/\sim$.

		      Si $n \in N$ alors $\exists w \in \mots$ \tq $\phi(w) = n$. On définit $h(n) = [w]$.

		      Il faut \mq
		      \begin{enumerate}
			      \item $h$ est bien défini (si $\phi (w') = \phi (w) \implies [w] = [w']$). \label{prop:recon1}
			      \item $h$ est un morphisme de monoïdes.
			      \item $h$ est surjectif.
		      \end{enumerate}

		      Montrons cela:
		      \begin{itemize}
			      \item \ref{prop:recon1}: Supposons que $\phi(w) = \phi (w')$.

			            On veut \mq $[w]=[w']$ \cad $w \sim_L w'$, \cad $\forall u,v \in \mots, uwv \in L \iff uw'v \in L$.

			            Soient $u,v \in \mots$.
			            \begin{eqnarray*}
				            uwv \in L &\iff& \phi (uwv) \in P\\
				            &\iff& \phi (u)\phi(w) \phi(v) \in P \reason{car $\phi$ est un morphisme} \\
				            &\iff& \phi (u)\phi(w') \phi(v) \in P \reason{car $\phi (w) = \phi (w')$ } \\
				            &\iff& \phi (uw'v) \in P\\
				            &\iff& uw'v \in L
			            \end{eqnarray*}

			            Donc $w \sim_L w'$.
		      \end{itemize}
		      Exercice : finir la preuve.
	\end{itemize}
\end{proof}

\section{Logique monadique du second ordre (MSO)}

Est-ce qu'il existe $w \in \mots$ \tq $w {\models} \exists x \exists y (x < y)$ soit vraie ?
($x,y$ sont interpretés comme des positions dans le mot $w$.)
Par exemple, $a {\nvDash} \exists x \exists y (x < y)$
Ici, cette proposition est satisfaite par $w \in \mots$ \ssi $\abs w \geq 2$.


Pour une proposition $\phi$, on a un langage $\La_\phi = \setdef {w \in \mots} {w \models \phi}$.


\begin{exemple}
	$\Sigma = \set{a,b}$
	$\phi_2 = \exists x \exists y (\forall z (z \geq x) \land Q_a x \land \forall z (z \leq y) \land Q_b y)$
	où $\leq, \geq$ sont interpretés comme les relations habituelles sur $N$.
	$Q_a x$ signifie qu'en position $x$, on retrouve la lettre $a$.
	$La_{\phi_2} = \setdef {w \in \kleene{\set {a,b}}} {w \models \phi_2} = a(a+b)*b$
\end{exemple}


%TODO: Comment and add line breaks
\begin{exemple}
	$$\phi_3 = \exists X (\forall x (\forall z (z \geq x) \ra X(x))) \land \forall x (\forall z (z \leq x) \ra \lnot X(x)) \land
		\forall x \forall y (((x < y) \land \forall z(z > x) \ra (z \geq y)) \ra (X (x) \lra \lnot X(y)))$$


	$X \rightsquigarrow $ un ensemble de positions dans un mot


	$X (x)\rightsquigarrow$ vrai \ssi $x \in X$


	$\La_{\phi_3} = \setdef {w \in \mots} {\abs w \equiv 0 (\text { mod} 2)}$
\end{exemple}

\subsection{Syntaxe}

\begin{definition} [La logique du premier ordre]
	\begin{itemize}
		\item $\Vi$ un ensemble de variables : $\set {x,y,z,x_1,y_1,z_1, \ldots}$
		\item \underline{Prédicats numériques} :  $\PP = \set {R_i^j, i > 0, j \geq o}$
		\item \underline{Formules atomiques} :
		      \begin{syntaxdef}
			      \syntaxHeader {\alpha} {Q_a x} {TODO}
			      \syntax {R_i^j(x_1, \ldots, x_j)} {Prédicats numériques}
		      \end{syntaxdef}
		\item \underline{Formules du premier ordre} :
		      \begin{syntaxdef}
			      \syntaxHeader {\phi} {\alpha} {Formules atomiques}
			      \syntax {\phi \land \phi} {Conjonction}
			      \syntax {\lnot \phi} {Négation}
			      \syntax {\exists x \phi} {Quantificateur existentiel}
		      \end{syntaxdef}
	\end{itemize}
\end{definition}


\begin{definition} [La logique du second ordre]
	C'est une extension de la logique du premier ordre :
	\begin{itemize}
		\item un ensemble de variables du second ordre : $X,Y,Z,X_1,Y_1,Z_1 \cdots$
		\item \underline{Formules atomiques} :
		      \begin{syntaxdef}
			      \syntaxExtension{\alpha}
			      \syntax {X(x)} {Appartenance a $X$}
		      \end{syntaxdef}
		\item \underline{Formules du second ordre} :
		      \begin{syntaxdef}
			      \syntaxExtension{\phi}
			      \syntax {\exists X \phi} {Quantificateur existentiel sur des ensembles}
		      \end{syntaxdef}
	\end{itemize}
\end{definition}


\subsection{Sémantique}

\subsubsection{Sémantique de la logique monadique du premier ordre}

Pour $n \in \N$ la longueur d'un mot, $x$ est interpreté comme un élément de $\set {1, \cdots, n}$

$$ \interpret {R_i^j}_n \subseteq \set {1, \cdots, n}^j$$
est une interpretation de chaque prédicat $R_i^j$ pour tout $n \geq 1$.
$$ \interpret {<}_n \subseteq \set {1, \cdots, n} \times \set {1, \cdots, n} = \setdef {(i,j)} {i<j}$$


\begin{definition}
	Une $\V$-structure, pour $\V \subseteq \Vi$ est un mot de la forme $(a_1, U_1) \ldots (a_n, U_n)$ où $a_1, \ldots, a_n \in \alphabet$ et
	$U_1, \ldots, U_n \subseteq \Vi$ tels que

	\begin{enumerate}
		\item $U_i \cap U_j = \emptyset, \forall i,j i \neq j$
		\item  $\bigcup_{i=1}^n U_i = \V$
	\end{enumerate}
\end{definition}

\begin{definition}[relation de satisfaction]
	On définit la relation de satisfaction $w \models \phi$ où $w$ est une $\V-structure$ et $\phi$ une formule de premier ordre \tq
	\begin{enumerate}
		\item $\forall x \in FV(\phi) \implies x \in \V$ (FV = variables libres)
		\item Les quantificateurs distincts dans $\phi$ lient des variables distinctes
	\end{enumerate}
\end{definition}

Si $\phi$ est un proposition, \cad, $FV(\phi) = \emptyset$ alors on a $\La_{\phi} = \setdef {w \text { les } \emptyset-\text{structures}} {w \models \phi} \subseteq \mots$.


Pour une interprétation $\interpret {R_i^j}_{{i,j}}$ de prédicats numériques, on définit la relation $w \models \phi$ par induction sur la structure de la formule $\phi$.

\begin{itemize}
	\item $w \models Q_a x$ ssi $w$ contient une lettre $(a,U_i)$ avec $x \in U_i$
	\item $w \models R_i^j(x_1, \ldots, x_j)$ ssi $\interpret {R_i}_{\abs w}(k_1, \ldots, k_j)$ est vrai où les $k1, \ldots, k_j$ sont les positions dans $w$ où les variables
	      $x_1, \cdots x_j$ apparaissent.
	\item $w \models \phi_1 \land \phi_2$ ssi $w \models \phi_1$ et $w \models \phi_2$
	\item $w \models \lnot \phi$ ssi $w \nvDash \phi$
	\item si $w = (a_1, U_1) \ldots (a_n, U_n)$ est une $\V$-structure.
	      $w \models \exists \phi$ ssi $\exists i \in \set {1, \ldots, n}$
	      $(a_1, U_1) \ldots  (a_i, U_i \cup \set x)  \ldots (a_n, U_n) \models \phi$
	      (il s'agit d'une $\V \cup {x}$-structure.)
\end{itemize}

\begin{exemple}
	$$abc \models_? \exists x \exists y (x < y)$$

	$(a, \set x)(b, \emptyset)(c, \emptyset) \models_? \exists y (x \leq y)$ est une $\set x $-structure

	$(a, \set x)(b, \set y)(c, \emptyset) \models_? (x < y) \iff (1,2) \in \interpret {<}_3 \iff 1 < 2$

	Version 2

	%TODO: petit dessin Photo
\end{exemple}


Soit $\phi$ et $\psi$ deux formules, $\phi \iff \psi$ \ssi $\La_{\phi} = \La_{\psi}$.

\begin{abbreviation}
	$\forall x \phi = \neg \exists x (\neg \phi)$
\end{abbreviation}

\begin{exemple}
	TODO: add it (important)
\end{exemple}


\subsubsection{Sémantique de la logique monadique du second ordre}


$V_1$ les variables du premier ordre

$V_2$ les variables du second ordre


\begin{definition}
	Une $(\V_1,\V_2)$-structure est un mot sur $\alphabet \times 2^{\V_1} \times  2^{\V_2}$, \cad un mot de la forme
	$(a_1,U_1,U_1) \cdots (a_n,U_n,V_n)$ où
	\begin{itemize}
		\item Les $U_i$ sont des ensembles de variables du premier ordre.
		\item Les $V_i$ sont des ensembles de variables du second ordre.
		\item $(a_1,U_1) \cdots (a_n,U_n)$ est une $\V_1$-structure.
	\end{itemize}
\end{definition}

\begin{definition}
	On définit la relation de satisfaction $w \models \phi$ où $w$ est une $(\V_1, \V_2)-structure$ et $\phi$ une formule de
	MSO avec les memes contraintes sur les variables que pour la logique du premier ordre %TODO: Add reference
	plus $w \models X(x)$ et $w \models \exists X \phi$
\end{definition}

L'induction est la même que pour le premier ordre, avec ces cas en plus :
\begin{itemize}
	\item $w \models X (x)$ ssi $w$ contient une lettre $(a,S,T)$ où $x \in S$ et $X \in T$
	\item $w \models \exists X \phi$ ssi $\exists J$ un ensemble de positions dans $w$ avec la proprieté:
	      la $(\V_1, \V2)$-structure $w'$ obtenue en remplaçant $(a_i, S_i,T_i)$ pour $i \in J$ par $(a_i, S_i, T_i \cup \set X)$ satisfait $\phi$.
\end{itemize}


\begin{exemple}
	TODO
\end{exemple}


\begin{definition}
	Un langage $L$ est définissable dans MSO[<] (la logique monadique du second ordre avec un prédicat binaire <).
	\ssi il existe une formule $\phi \in MSO[<]$ \tlq $L = \setdef {w\in \mots} {w \models \phi}$
\end{definition}

\begin{definition}
	Un langage $L \subseteq \mots$ est définissable dans $FO[<]$ (la logique du premier ordre avec <)
	\ssi $\exists \phi \in FO[<]$ \tlq $L = \setdef {w \in \mots} {w \models \phi}$
\end{definition}


\subsection{Relation avec les expressions rationnelles}

\begin{theorem}
	Un langage est définissable dans $MSO[<]$ ssi $L$ est régulier.
\end{theorem}


\begin{proof}
	%TODO: use \setminus instead of \
	\begin{itemize}
		\item $\Leftarrow$

		      Soit $A = \AFD$ un automate fini déterministe qui accepte un langage $L$.  On peut supposer que $L \subseteq \mots$ (
		      car $\motvide \in L$, on va prendre la disjonction de la formule obtenue pour $L \setminus \set {\motvide}$ avec $\forall x \neg (x = x)$)

		      Soit $w \in \mots$, Alors $w$ est reconnu par l'automate $A$ ssi $\exists X_0,\cdots,X_{k-1} \subseteq \set{1, \cdots , \abs w}$
		      tels que les propriéts suivantes soient vérifiés :
		      \begin{enumerate}
			      \item $\bigcup_{i=0}^{k - 1} X_i = \set{1, \cdots, \abs w}$
			      \item $\forall i < j, X_i \cap X_j = \emptyset$
			      \item $1 \in X_0$
			      \item $\forall j \in \set{1, \cdots, \abs w}$ si $j \in X_i \et j+i \in X_e$ et si $a$ est la lettre en position $j$ dans le mot
			            $w$, alors $\delta (q_i,a) = q_e$.

			      \item Si $\abs w \in X$ et $a$ est ;a dernière lettre de $w$, alors $\delta (q_j,a) \in F$.
		      \end{enumerate}

		      Supposons que $w = \decomp a n$ est accepté par $A$. On construit les ensembles $(X_i)_{i \in \enum 0 {k-1}}$
		      tel que $i \in X_j$ ssi après avoid lu les premiers $i-1$ lettres de $w$ on arrive a l'état $q_j$
		      %TODO: Diagram ? 

		      Les ensembles $X_0, \cdots, X_{k-1}$ satisfait les 5 propriétés.
		      \begin{enumerate}
			      \item $X_0 \cup \cdots \cup X_{k-1} = \enum i {\abs w}$. L'inclusion a gauche est vraie par définition. Montrons l'autre inclusion.
			            On considère l'unique chemin dans l'automate $A$ obtenu en lisant les premieres $i-1$ lettres de $w$ à partir de l'etat $q_0$.
			            Supposons qu'on arrive dans l'état $q_j$. Par définitions $u \in X_j$.
			      \item $\forall i < j, X_i \cap X_j = \emptyset$ est vrai car l'automate est déterministe.
			      \item $1 \in X_0$ car si on lit les premieres $0$ lettres de $w$ on reste a l'état $q_0$.
			      \item Après les premieres $j-1$ lettres on arrive a $q_i$. $a$ est la lettre en position $j$. Donc après les
			            premieres $j$ lettres on arrive dans l'état $\delta (q_i, a)$. Si après les premières $j$ lettres on arrive dans l'état $q_l$, alors
			            $\delta (q_i, a) = q_l$.
			      \item $w$ est accepté par $A$. Donc si $\abs w \in X_j$ alors après avoir lu les premieres $\abs w - 1$ lettres on arrive dans $q_j$.
			            Si $a$ est la dernière lettre de $w$, alors $\delta (q_j,a)$ est un tat acceptant car $w$ est accepté.
		      \end{enumerate}
		      On construit la formule
		      $$ \phi = \exists X_0 \cdots \exists X_{k-1} (\phi_1 \land \cdots \land \phi_5)$$
		      où
		      \begin{itemize}
			      \item $\phi_1 = \forall x (X_0(x) \lor \cdots \lor X_{k-1}) = \forall x \bigvee_{i=1}^{k-1} X_i(x)$
			      \item $\phi_2 = \forall x \bigwedge_{0\leq i < j \leq k-1}\lnot (X_i(x) \land X_j(x))$
			      \item $\phi_3 = \exists x \left((\forall y, x \leq y) \land X_0(x) \right)$
			      \item $\phi_4 = \forall x \left( \forall y (y = x + 1)  \ra \bigwedge_{0\leq i < l < k} ( (X_i (x) \land X_l(y) )\ra \bigvee_{S_l} Q_a(x))\right)$\\
			            Où $S_l = \setdef {a \in \alphabet} {\delta (q_i,a) = a_l}$
			      \item $\phi_5 = \forall x \left( \forall y (x \geq y)  \ra \bigwedge_{i = 0}^{k-1} ( X_i(x) \ra \bigvee_{T_i} Q_a(x)))\right)$\\
			            Où $T_i = \setdef {a \in \alphabet} {\delta (q_i,a) \in F}$

		      \end{itemize}

		\item $\Rightarrow$

		      On suppose que $L$ est défini par une formule de $MSO[<]$.
		      La preuve est par induction sur la structure de la formule.

		      $L \subseteq (\alphabet \times 2^{\V_1} \times 2 ^ {\V_2})$.

		      %TODO: Rappel ?
		      $\La$ esl l'ensemble de toutes les $(\V_1, \V_2)$-structures, pour $\V_1$ en ensemble de variables du premier
		      ordre et $\V_2$ un ensemble de variables du second ordre.

		      \begin{exercice}
			      Trouver un automate sur l'alphabet $\alphabet \times 2^{\V_1} \times 2 ^ {\V_2}$ qui accepte $\La$
		      \end{exercice}

		      \begin{exercice}
			      Trouver un automate qui accepte les $(\V_1, \V_2)$-structures \tq une variable de premier ordre $x$ apparait
			      dans une lettre de la forme $(a,S,T)$. Le but est de \mq le $\lang {Q_a(x)}$ est régulier.
		      \end{exercice}

		      \begin{exercice}
			      Montrer que $\La_{X(x)}$ est régulier.
		      \end{exercice}

		      \begin{exercice}
			      Montrer que $\La_{x< y}$ est régulier.
		      \end{exercice}



		      Soit $\phi = \exists x \psi$ et supposons que $L_{\psi}$ est régulier et donc accepté par un automate $A = \AFD$ sur l'alphabet
		      $\alphabet \times 2^{\V_1} \times 2 ^ {\V_2}$.

		      On définit l'automate $A' = \antuple{Q\times \set {0,1}, (q_0,0), F \times \set 1 , \delta'}$
		      où $\delta' = \setdef {((q,u),(a,S,T),(q',u))} {u \in \set {0,1}, x \notin S, (q,(a,S,T),q')\in \delta}
			      \cup
			      \setdef {((q,0),(a,S\ \set x,T),(q',1))} { x \in S, (q,(a,S,T),q')\in \delta}$
		      %TODO: diagram ???

		      $A'$ est un automate sur l'alphabet $\alphabet \times 2^{\V_1\setminus \set x} \times 2 ^ {\V_2}$.

		      %TODO: add final step
	\end{itemize}
\end{proof}



\begin{theorem}
	Un langage est définissable dans $FO[<]$ ssi il existe une expression rationnelle sans étoile $r$ définissant un langage
	ssi
	le monoïde syntaxique du langage est apériodique (\cad il ne contient aucun sous-groupe non-trivial).
\end{theorem}






\section{Apprentissage de langages rationnels par queries et contrexemples}

Nous allons étudier un algorithme qui permet de générer un automate pour un langage rationnel $L$.
L'idée est que le ""learner"" (ou "étudiant"), une entité dont l'objectif est de retrouver le langage, pose des
questions à un ""oracle"" (ou "teacher"). Les questions sont de deux types :

\begin{itemize}
	\item Les ""membership queries"" : Est-ce que le "mot" $w$  appartient au langage $L$ ?
	\item Les ""conjectures"" : On présente un automate "conjecture". Si cet automate reconnaît $L$, alors il retourne \emph{Ok}.
	      Sinon, il retourne un "mot" ""contre-exemple"" $w$. Ce "mot" vérifie soit $w \in L$ (il n’est pas accepté par notre
	      automate), soit $w \notin L$ (il est accepté par notre automate).
\end{itemize}

Cet algorithme découle du travail de \cite{angluinLearning}.


\subsection{Définitions}


\begin{definition}[Clôture par préfixe / (\blue{suffixe})]
	Un ensemble $S$ est ""clos par préfixe"" (\blue{suffixe}), si
	$$\forall w \in S, \text{ tout préfixe (\blue{suffixe}) de $w$ est dans } S$$
\end{definition}

\begin{exercice}
	Indiquez lesquels des ensembles suivants sont "clos par préfixe" :
	\begin{enumerate}
		\item $\set{0,2,10,010} \ \times$
		\item $\set{110,1,0,\motvide,11} \ \checkmark$
		\item $\set{1110,10,1} \ \times$
		\item $\set{011,0,\motvide,11,01} \ \times$
		\item $\set{111,\motvide,11,1,0} \ \checkmark$
	\end{enumerate}
\end{exercice}

Le learner maintient une table $T$ avec:

\begin{itemize}
	\item Les colonnes de $T$ qui sont indexées par un ensemble $E$ de "mots" "clos par suffixe".
	\item Les lignes sont indexées par un ensemble de "mots" $S \cup S\cdot \alphabet$, ou $S$ est
	      un ensemble "fermé par préfixe".
\end{itemize}


\begin{definition}
	Si $s \in S \cup S\alphabet$, soit $n = \abs E$, on définit ""row""$(s) \in [0,1]^n$ la suite suivante :
	$$ \row s = (T[s,e_1], \ldots, T[s, e_n])$$
\end{definition}

\begin{definition}
	Une table est \emph{""close""} si
	$$ \forall t = sa \in S\alphabet, \exists s'\in S, \row {s'} = \row s$$
\end{definition}

\begin{definition}
	Une table est \emph{""cohérente""} si
	$$ \forall s_1, s_2 \in S, \row{s_1} = \row{s_2} \implies \forall a \in \alphabet, \row{s_1a} = \row{s_2a}$$
\end{definition}


\begin{exemple} \label{ex:tables}
	\ \newline
	\begin{itemize}
		\item La table suivante n'est pas "close" car $\nexists s \in S, \row s = \row {ab}$
		      \begin{center}
			      \begin{tabular}{c|c|c}
				                 & $\motvide$ & a \\ \hline
				      $\motvide$ & 0          & 1 \\
				      a          & 1          & 1 \\
				      b          & 0          & 0 \\ \hline

				      aa         & 0          & 0 \\
				      ab         & 1          & 0 \\
				      bb         & 0          & 1 \\
				      bb         & 1          & 1 \\
			      \end{tabular}
		      \end{center}

		\item La table suivante est "close" et "cohérente"
		      \begin{center}
			      \begin{tabular}{c|c|c}
				                 & $\motvide$ & a \\ \hline
				      $\motvide$ & 0          & 0 \\
				      a          & 1          & 1 \\ \hline

				      b          & 1          & 1 \\
				      aa         & 0          & 0 \\
				      ab         & 0          & 0 \\
			      \end{tabular}
		      \end{center}
		\item La table suivante n'es pas "cohérente" car $\row {\motvide} = \row a$ mais
		      $\row {\motvide a} = \row a \neq \row {aa}$

		      \begin{center}
			      \begin{tabular}{c|c|c}
				                 & $\motvide$ & a \\ \hline
				      $\motvide$ & 0          & 1 \\
				      a          & 0          & 1 \\
				      b          & 0          & 0 \\ \hline

				      aa         & 0          & 0 \\
				      ab         & 0          & 1 \\
				      bb         & 1          & 1 \\
				      bb         & 1          & 0 \\
			      \end{tabular}
		      \end{center}
	\end{itemize}
\end{exemple}


\subsection{L'algorithme}

\begin{theorem}\label{thm:automata-tables}
	À partir d'une table "close" et "cohérente", on peut construire un automate $A = (Q, I, F, \delta)$ où :
	$$
		\begin{aligned}
			Q                   = & \setdef{ \row{s}} {s \in S }  \\
			I                   = & \set{ \row{\motvide} }        \\
			F                   = & \setdef{ \row{s} }{T(s) = 1 } \\
			\delta(\row{s}, a)  = & \row{sa}.
		\end{aligned}
	$$
\end{theorem}

\begin{exemple}
	Nous allons voir comment passer de l'une des tables de l'exemple \ref{ex:tables} vers son automate associé.\\
	\begin{minipage}{0.5\textwidth}
		\centering
		\begin{tabular}{c|c|c}
			           & $\motvide$ & a \\ \hline
			$\motvide$ & 0          & 0 \\
			a          & 1          & 1 \\ \hline

			b          & 1          & 1 \\
			aa         & 0          & 0 \\
			ab         & 0          & 0 \\
		\end{tabular}
	\end{minipage}
	\begin{minipage}{0.5\textwidth}
		\centering
		\begin{automata}
			\digraph[scale=0.5]{automateTable}{
				rankdir=LR;

				node [shape=circle, style=filled, color=lightblue];
				q0 [label="00"];
				q1 [label="11"];

				start [shape=point];
				start -> q0;

				q0 -> q1 [label="a,b"];
				q1 -> q0 [label="a,b"];

				q1 [shape=doublecircle];
			}
		\end{automata}
	\end{minipage}
\end{exemple}

Initialement on propose Figure \ref{fig:algo-inL} ou Figure \ref{fig:algo-notinL}, selon si $\epsilon \in L$ ou non.

\begin{twoautomata}
	\centering
	\begin{automata}
		\digraph[scale=0.5]{automateBase1}{
			rankdir=LR;

			node [shape=circle, style=filled, color=lightblue];
			q0 [label=" "];

			start [shape=point];
			start -> q0;

			q0 -> q0 [label="Σ"];

			q0 [shape=doublecircle];
		}
	\end{automata}
	\caption{Automate pour $\motvide \in L$}\label{fig:algo-inL}
\end{twoautomata}
\begin{twoautomata}
	\centering
	\begin{automata}
		\digraph[scale=0.5]{automateBase2}{
			rankdir=LR;

			node [shape=circle, style=filled, color=lightblue];
			q0 [label=" "];

			start [shape=point];
			start -> q0;

			q0 -> q0 [label="Σ"];
		}
	\end{automata}
	\caption{Automate pour $\motvide \notin L$}\label{fig:algo-notinL}
\end{twoautomata}

Tant que l'"oracle" donne un "contre-exemple"
\begin{itemize}
	\item $w$ = "contre-exemple".
	\item $w$ et tous ses préfixes sont ajoutés à $S$.
	\item Les nouvelles lignes sont remplies en posant des "membership queries".
	\item Tant que $T$ n'est pas "cohérente" ou n'est pas close:
	      \begin{itemize}
		      \item Si $T$ n'est pas "cohérente" (\cad, $\exists s_1, s_2 \in S, \exists a \in \alphabet, \row{s_1} = \row{s_2} \et \row{s_1a} \neq \row{s_2a}$
		            \cad $\exists e \in E, T (s_1ae) \neq T(s_2ae)$)

		            On ajoute $ae$ et tous ses suffixes à $E$.
		      \item Si $T$ n'est pas "close" alors ($\exists t \in S \alphabet, \forall s \in S, \row t \neq \row s$)

		            Ajouter $t$ à $S$.
	      \end{itemize}
	\item On propose à l'"oracle" l'automate "conjecture" associé à $T$
\end{itemize}

\subsection{Propriétés des tables closes et cohérentes}

\begin{notation}
	On note $(S,E,T)$ une table, où $S$ est l'ensemble des lignes. $E$ l'ensemble des "mots" des colonnes et $T$ la fonction de vérité.
\end{notation}

\begin{definition}
	Un automate $(Q, q_0, F, S)$ est ""en accord"" avec une fonction $T$, si
	$$ \forall s \in S \cup S\alphabet, \forall e \in E, \delta(q_0,se) \in F \iff T(se) = 1$$
\end{definition}

\begin{theorem}
	Si $A$ est l'automate issu d'une table $(S,E,T)$ alors $A$ est "en accord" avec $T$ et tout autre
	automate "en accord" avec $T$ et non isomorphe à $A$ possède au moins un état de plus que $A$.
\end{theorem}

\begin{lemma} \label{lem:learning-9}
	Si $(S,E,T)$ est "close" et "cohérente" alors dans l'automate $A = \mathcal A (S,E,T)$
	$$\forall s \in S \cup S\alphabet, \delta(q_0,s) = \row s$$
\end{lemma}

\begin{proof}
	Induction sur la taille de $s$.
	\begin{itemize}
		\item Si $\len s = 0$, alors $s = \motvide$ et donc $\delta(q_0, \motvide) = \row \motvide$
		      est vrai par construction de l'automate.
		\item Supposons l'énoncé vrai pour tout "mot" appartenant à $S \cup S\alphabet$ de longueur inférieure ou égale a $n$ et
		      soit $t$ un "mot" de longueur $n +1$, alors $ t  = s \cdot x,$ avec $s \in S$, car
		      si $t \in S\alphabet$, $s \in S$ trivialement et si $t \in S$, comme $S$ est clos par préfixe, $s \in S$.

		      \begin{eqnarray*}
			      \delta(q_0, t) &=& \delta(\delta(q_0, s), x) \\
			      &=& \delta(\row s, x)  \reason {Par hypothèse d'induction} \\
			      &=& \row {sx}  \reason {Par la définition de $\delta$ dans \ref{thm:automata-tables}} \\
			      &=& \row t
		      \end{eqnarray*}
	\end{itemize}
\end{proof}


\begin{lemma}
	Si $(S,E,T)$ est "close" et "cohérente" alors l'automate $A = \mathcal A (S,E,T)$
	est "en accord" avec la fonction $T$, \ie
	$$\forall s \in S \cup S\alphabet, \forall e \in E, \delta(q_0,se) \in F \iff T(se) =1$$
\end{lemma}

\begin{proof}
	Induction sur la taille de $e$.

	\begin{itemize}
		\item Si $\len e = 0$, alors $e = \motvide$.
		      Soit $s \in S \cup S \alphabet$, et donc $se = s$.
		      \begin{itemize}
			      \item Si $s \in S$, alors par le lemme précédent,
			            $\delta(q_0, s) = \row {s}$ et
			            comme $\row {s} \in F \iff T(s) = 1$ par construction de l'automate on retrouve le résultat cherché.
			      \item Si $s \in S\alphabet$, comme la table est "close", il existe $s' \in S$, \tq $\row {s'} = \row s$ qui vérifie
			            $\row {s'} \in F \iff T(s') = 1$ par construction de l'automate.

		      \end{itemize}
		\item Supposons l'énoncé vrai pour tout "mot" de $E$ de longueur inférieure ou égale a $n$. Soit $e \in E$ tel que $\len e = n + 1$,
		      et on pose $e = x e'$, avec $x \in \alphabet$. Soit $s \in S \cup S\alphabet$, il existe
		      $s'$ \tq $\row s = \row {s'}$ car la table est "close".
		      \begin{eqnarray*}
			      \delta(q_0, se) &=& \delta(\delta(q_0,s),e) \\
			      &=& \delta(\delta(q_0,s),xe') \\
			      &=& \delta(\row s,xe') \reason{par le lemme \ref{lem:learning-9}} \\
			      &=& \delta(\row {s'},xe') \reason{car $\row s = \row {s'}$} \\
			      &=& \delta(\delta (\row {s'}, x), e') \\
			      &=& \delta(\row {s'x}, e') \reason {par définition de $\delta$}\\
			      &=& \delta(\delta{q_0, s'x}, e') \\
			      &=& \delta(q_0, s'x e')
		      \end{eqnarray*}
		      Comme $\len {e'} = n$ on peut appliquer l'hypothèse d'induction et on a
		      $$ \delta(q_0, se) = \delta(q_0, s'x e') \in F \iff T(s'x e') = 1$$

		      Comme $\row s = \row {s'}$  et $e = xe'$ on a que $T(s'xe') = T(sxe') = T(se)$.
	\end{itemize}
\end{proof}

\begin{lemma}
	Si $(S,E,T)$ est "close" et "cohérente" alors si automate $A = \mathcal A (S,E,T)$
	a $n$ états, tout automate $A'$ "en accord" avec $T$ et ayant au plus $n$ états est isomorphe à $A$.
\end{lemma}

\begin{proof}
	Soit $A' = (Q', q_0', F', \delta')$ un autre automate "en accord" avec $T$ et ayant au plus $n$ états.

	L'objectif est d'exhiber un isomorphisme entre les deux automates. La construction et la vérification
	de cet isomorphisme suivent les arguments détaillés dans la preuve du Lemme 4 de \cite{angluinLearning}.
\end{proof}

\subsection{Terminaison de l'algorithme}

\begin{lemma}
	Si $(S,E,T)$ est une table alors tout automate "en accord" avec $T$ possède au
	moins autant d'états que le nombre de valeurs distinctes de l'ensemble $\setdef {\row s} {s \in S}$.
\end{lemma}

\begin{proof}
	Soit $A = (Q,q_0,F,\delta)$ un automate compatible avec $T$ et définissons
	$f: R \to Q$, $f(\row s) = \delta(q_0, s)$.

	Soient $s_1 \et s_2$ \tq $\row {s_1} \neq \row{s_2}$, donc $\exists e \in E, T(s_1,e) \neq T(s_2,e)$.
	Comme $A$ est "en accord" avec $T$, seulement l'un des deux "mots" est reconnu.

	Sans perte de généralité, on suppose $s_1e$ reconnu et $s_2e$ pas reconnu, donc
	$\delta (q_0, s_1, e) \in F$, alors que $\delta (q_0,s_2,e) \notin F$, cela implique que
	$\delta (q_0,s_1) \neq \delta (q_0,s_2)$.

\end{proof}

Soit $n$ le nombre d'états de l'"automate minimal" reconnaissant le "langage" cherché.

\begin{remarque}
	Le nombre d'états des différents automates "conjecture" ne peut qu'augmenter.
	\begin{itemize}
		\item Si on ajoute un "mot" à $E$ parce que la table n'était pas "cohérente", le nombre de "rows" distincts augmente
		      d'une unité.
		\item Si on ajoute un "mot" à $S$ parce que la table n'est pas "close", le nombre de "rows" distincts augmente d'une unité.
	\end{itemize}
	On déduit qu'on ne peut pas trouver une table non "close" ou non "cohérente" plus de $n-1$ fois tout au long de l'algorithme.

	En particulier, on déduit qu'après un nombre fini d'étapes de l'algorithme on arrive toujours à trouver
	une table "close" et "cohérente" et à émettre une "conjecture".
\end{remarque}


Combien de "conjectures" le "learner" émet-il avant de donner la bonne ?

Soit $(S,E,T)$ une table "close" et "cohérente" et $A(S,E,T)$ l'automate associé.
Supposons que $A$ soit une "conjecture" fausse, et donc l'"oracle" fournit un "contre-exemple" $t$.

L'automate $S(S,E,T)$ et l'automate du langage cherché sont en désaccord sur le "mot" $t$, donc ces deux automates
ne sont pas équivalents, et donc $A(S,E,T)$ a au moins un état de moins que l'automate cherché.
Donc $A(S,E,T)$ a au plus $n-1$ états. On en déduit que le nombre de fausses "conjectures" émises pas le "learner" est $n-1$.

On en déduit que l'algorithme s'arrête toujours, au pire après avoir rendu la table "close" et
"cohérente" $n-1$ et après avoir émis au plus $n-1$ fausses "conjectures".


\subsection{Analyse de la complexité}

Elle dépend du nombre $n$ et de la longueur du plus long "contre-exemple" fourni par l'oracle, notée $m$.


\begin{itemize}
	\item A chaque fois qu'on trouve une table non "cohérente", on ajoute un "mot" à $E$.
	\item A chaque fois qu'on trouve une table non "close", on ajoute un "mot" à $S$.
	\item A chaque fois qu'on émet une ""conjecture"" fausse, on ajoute au plus $m$ "mots" à $S$
\end{itemize}

On en déduit que $\abs E \leq n$.

La longueur maximale des "mots" de $E$ augmente au plus d'une unité à chaque fois qu'on rend la table "cohérente"
(on ajoute à $E$ un "mot" $xe$ avec $e \in E$). On déduit que la longueur maximale d'un "mot" de $E$ est toujours $<n$.

De même, on a que  $\abs S \leq n + m(n-1)$. %TODO: explanation of each number.

La longueur maximale des "mots" de $s$ est inférieure à $m - n -1$.


On déduit que le cardinal maximal de $(S \cup S \Sigma) \times E$ (le nombre maximal de classes de la table) est : $(k+1)(n+m(n-1))n \in O(mn^2)$.


Considérons le coût de chaque type d'opérations :
\begin{itemize}
	\item Vérifier si une table est "close" et "cohérente" se fait en temps polynomial relativement à la taille de la table.
	\item Ajouter un "mot" à $S$ ou à $E$ nécessite au plus $nm$ "membership queries".
	\item Construire une ""conjecture"" à partir d'une table "cohérente" et "close" est faisable en temps polynomial relativement à
	      la taille de la table.
	\item Un "contre-exemple" nécessite l'addition d'au plus $m$ "mots" à $S$ et cette opérations est effectuée au plus $n-1$ fois.
\end{itemize}

\begin{theorem}
	L'algorithme produit toujours un automate isomorphe à l'"automate minimal" du language cherché, de plus, le temps de calcul peut être
	exprimé par un polynôme en $n$ et en $m$.
\end{theorem}

\begin{remarque}
	Si l'"oracle" fournit toujours un "contre-exemple" de taille $\leq n$ (ce qui est toujours possible), alors le polynôme en question dépend de $n$ seulement.
\end{remarque}


\section{String matching}

\begin{definition}
	Étant donné deux chaines de caractères, $T$ le texte et $M$ le motif, on dit que $T$ présente une occurrence de $M$ si
	$$ \exists i, 0 \leq i \leq \len T - \len M, \forall j \in \enum 1 {\len M -1}, T[i + j] = M[j]$$
\end{definition}

\begin{remarque}
	Le texte $T$ contient le facteur $u$ \ssi $T$ a un préfixe qui appartient à $\mots u$.
\end{remarque}

\begin{remarque}
	L'algorithme trivial qui teste toutes les positions de $T$ et vérifie s'il y a une occurrence de $u$ en position $i$ est en $O(\abs T \abs u)$.

	\begin{algorithmic}[lines]
		\Function{Naif}{$T,u$}
		\For{$i$ in $0, \ldots , \len T - \len u + 1$}
		\State $j = 0$
		\While  {$j  <  \len u \et (u[j] == t[i+j])$}
		\State j ++
		\EndWhile
		\If {$j == \abs u$}
		\State Afficher l'occurrence de $u$ en position $i$
		\EndIf
		\EndFor
		\EndFunction
	\end{algorithmic}
\end{remarque}

\begin{remarque}(Un premier algorithme meilleur que le naïf)
	Pour chercher les occurrences de $u$ dans $T$, on peut construire un automate qui reconnait $\mots u$ et lui faire analyser $T$.
	À chaque fois qu'on passe par un état final, on vient de voir une occurrence de $M$.
\end{remarque}

\begin{remarque}
	Avec l'algorithme naïf, chaque caractère de $T$ pourrait être analysé $\abs u$ fois, car le motif est décalé d'une position à chaque fois.
	Avec les automates, chaque caractère est analysé une seule fois, ce qui explique l'amélioration.
\end{remarque}

\begin{remarque}
	L'approche par automates présente cependant quelques problèmes d'efficacité :
	\begin{itemize}
		\item Si l'automate est non déterministe, le temps de calcul peut devenir très lourd : il faut analyser un nombre potentiellement exponentiel de chemins.
		\item Si on le "déterminise" avant, alors on risque de trouver un automate de taille exponentielle.
	\end{itemize}
\end{remarque}

\subsection{Knuth-Morris-Pratt}


L'algorithme Knuth-Morris-Pratt, \cite{kmp}, améliore l'algorithme naïf en introduisant des décalages d'amplitude
$> 1$ et fait en sorte que chaque caractère de $T$ soit analysé une seule fois.

\begin{definition}
	Si $w$ est un mot, on appelle ""bord"" de $w$ le mot le plus long qui est en même temps préfixe et suffixe propre de $w$ .
\end{definition}

\begin{exemple}
	\quotes A et  \quotes{ABA} sont les "bords" de \quotes{ABABA}.
\end{exemple}


Pour tout mot $w \in \mots$, on note $\Bord w$ le plus long "bord" de $w$,
\cad le plus long préfixe propre de  $w$ qui est aussi un suffixe propre de $w$.
On note également $b(w)$ la longueur de $\Bord w$. De plus, pour tout $j$ tel que
$0 \leq j \leq \len w $, on note $w_j$ le préfixe de $w$ de longueur $j$.

La fonction KMP repose sur le calcul préalable de $\Bord{w_j}$ et  $b(w_j)$ pour tous les
$j$, dans un temps polynomial en $\len w$. Cette phase de prétraitement permet d’optimiser les décalages
à effectuer lors de la recherche d’un motif $u$ dans un texte $T$.

Supposons qu’au cours de la recherche de $u$ dans $T$, un échec se produit au niveau de la position $j$
du motif $u$ alors que l'on examine le texte à partir de la position $i$. Cet échec indique que le caractère
$T[i+j]$ ne correspond pas à $u[j]$.

Pour déterminer le prochain décalage efficace de $u$, on observe que :
\begin{itemize}
	\item Décaler $u$ d’une amplitude $k$, $1 \leq k \leq j$, n’a de sens que si, après ce décalage,
	      un préfixe de $u$ peut s’aligner avec la portion correspondante de $T$.
	\item Il est inutile de considérer des décalages $k < b(u_{j-1})$, car $\Bord{u_{j-1}}$
	      est précisément le plus long suffixe de $u_{j-1}$ qui est aussi un préfixe de $u$.
\end{itemize}

Ainsi, après un échec à la position $j$, le décalage optimal consiste à aligner $\Bord{u_{j-1}}$
au début de $u$ avec sa précédente occurrence dans $T$. Autrement dit, on recule le curseur
$j$ à $b(u_{j-1})$ sans avoir à comparer à nouveau les $b(u_{j-1})$ premiers caractères,
car ils sont garantis égaux en raison de la définition du "bord".

Ce mécanisme permet de limiter les comparaisons inutiles et assure que chaque caractère de $T$
est comparé à un nombre constant de caractères de $u$ en moyenne, rendant ainsi le nombre total
de comparaisons linéaire en $\len T + \len u$.


\begin{exemple}
    \todo{explain}
	\begin{eqnarray*}
		T &=& ABABAABCBABABACAB \\
		u &=& ABABACA
	\end{eqnarray*}
\end{exemple}


L'algorithme effectue un pré-traitement dans lequel pour tout $j$ avec $0 \leq j \leq \len u - 1$
calcule le plus grand "bord" du préfixe de $u$ de longueur $j$.

\begin{exemple}
	Pour le motif \quotes{ABABACA}:

	\begin{itemize}
		\item $\Bord {\motvide} = \motvide$
		\item $\Bord A = \motvide$
		\item $\Bord {AB} = \motvide$
		\item $\Bord {ABA} = A$
		\item $\Bord {ABAB} = AB$
		\item $\Bord {ABABA} = ABA$
		\item $\Bord {ABABAC} = \motvide$
		\item $\Bord {ABABACA} = A$
	\end{itemize}
\end{exemple}

Il suffit de calculer une fonction qui à tout $j \in \enum 0 {\len u - 1}$ associe
la longueur du plus long "bord" du préfixe de longueur $j$ du motif, c'est la fonction dite préfixe.


Cet algorithme peut aussi se retrouver comme un algorithme issu des automates.

Soit $u$ le motif à chercher.

\begin{definition}
	Pour tout $w \in \mots$ on défini
	$q(w) = \setdef {\text{ le plus long } X} {X \text{ est suffixe de } w \et X \text{ est préfixe de } u}$
\end{definition}

\begin{prop}
	Soit $\congN$ la congruence de Nérode induite par le langage $\mots u$, alors
	$$ w_1 \congN w_2 \iff q(w_1) = q(w_2)$$
\end{prop}

\begin{proofI}
	\item \bimpLR \\
	Supposons $w_1 \cong w_2$ ($\forall y \in \mots, w_1y \in \mots u \iff w_2y\in \mots u$)

	Notons $z$ le plus long suffixe de $w_1$ qui est aussi préfixe de  $u$ et notons $y$ le mot $z^{-1}u$ (le mot obtenu de $u$ en effaçant $z$ au début)
	alors $w_1y$ se termine par $u$, c'est-à-dire est  dans $\Sigma^*u$.


	Le mot $y$ est le mot de longueur minimale tel que $w_1y$ se termine par $u$, car s'il existait un mot $y'$ plus court que $y$  tel que $w_1y'$
	se termine par $u$ alors $z$ ne serait pas le plus long suffixe de $w_1$ qui est aussi préfixe de $u$.


	Puisque $w_1 \cong  w_2$ on a que   $w_2y$  est  dans $\Sigma^*u$ aussi, c'est-à-dire  se termine par $u$, de plus  $y$ est aussi le mot de
	longueur minimale  tel que $w_2y$  est  dans $\Sigma^*u$. En effet  si un mot $y''$ tel que $w_2y''$  est  dans $\Sigma^*u$  existait alors
	$w_1 y''$ serait aussi dans $\Sigma^*u$  à cause de l'équivalence, en contradiction avec la minimalité de $y$.


	Il en découle que $z$ est aussi un suffixe de $w_2$ et que c'est le plus long suffixe de $w_2$ qui aussi préfixe de $u$, donc $q(w_1)=q(w_2)$.

	\item \bimpRL \\
	Supposons que $q(w_1) = q(w_2)$ et nous voulons montrer que
	$$\forall y, \quad w_1 y \in \mots u \iff w_2 y \in \mots u$$

	Ceci est trivialement vrai si $\len y \geq \len u$,
	dans ce cas

	\begin{eqnarray*}
		w_1 y \text{ se termine par } u & \text{ ssi }& y \text{ se termine par } u \\
		& \text{ ssi }& w_2y \text{ se termine par } u
	\end{eqnarray*}

	Supposons donc que $\len y < \len u$ et considérons le mot $w_1 y$

	$w_1y \in \mots u \implies \exists$ un suffixe $z$ de $w_1$ qui est préfixe de $u$ et $z$ est un suffixe de $q(w_1) = q(w_2)$
	donc $z$ est aussi un suffixe de $q(w_2)$ et en particulier il est un suffixe de $w_2 \implies w_2 y $ se décompose en
	$w_2'zy = w_2 y \in \mots u$
\end{proofI}

Cette proposition suggère une nouvelle manière pour calculer l'automate minimal pour $\mots u$

\begin{eqnarray*}
	Q  &=& \set {\text {les préfixes de } u} \\
	q_0 &=& <\motvide> \\
	F &=& \set{<u>} \\
	\delta(<v>,a) &=&  \left\{ \begin{array}{cc}
		<v \cdot a>  & \text{si } v\cdot a \text{ est un préfixe de } u \\
		q(v \cdot a) & \text{sinon }
	\end{array}\right.
\end{eqnarray*}

\begin{exemple}
	Pour le mot \quotes{ABABACA} on construit l'automate suivant : \\
	\begin{automata}
		\digraph[scale=0.50]{automateStringMatch}{
			rankdir=LR;

			node [shape=circle, style=filled, color=lightblue];
			0;
			1;
			2;
			3;
			4;
			5;
			6;

			start [shape=point];
			start -> 0;

			0 -> 0 [label="X ≠ A"];
			0 -> 1 [label="A"];

			1 -> 1 [label="A"];
			1 -> 2 [label="B"];

			2 -> 3 [label="A"];

			3 -> 4 [label="B"];
			3 -> 1 [label="A"];

			4 -> 5  [label="A"];

			5 -> 6  [label="C"];
			5 -> 4  [label="B"];
			5 -> 1  [label="A"];

			6 -> 7  [label="A"];

			7 -> 2  [label="B"];
			7 -> 1  [label="A"];

			7 [shape=doublecircle];
		}
	\end{automata}

	L'automate est complet et tout transition \quotes{manquante} va vers l'état 0.
\end{exemple}


\begin{remarque}
	Si dans cet automate on remplace les préfixes de $u$ par leur longueur, on se rend compte que cet automate
	contient la même information que la fonction préfixe.
\end{remarque}

\section{Évaluation de fonctions booléennes}

\begin{notation}
	On note $\B = \set {0,1}$.
\end{notation}

On veut évaluer les fonctions $f : \B ^ n \to \B$

\begin{exemple}
	Une telle fonction est $f(x_1,x_2,x_3) = (x_1 \land x_2) \lor x_3 : \B^3 \to \B$
\end{exemple}

Une fonction $f : \B ^ n \to B$ est représentable avec $2^k$ bits
(on donne les images des $2^n$ $n$-uplets sur $\B$).
Donc au total il y a $2^{2^n}$ fonctions $\B^n \to \B$.

\begin{idee}
	Utiliser des automates qui reconnaissent le langage des $n$-uplets
	$(x_1,x_2,\ldots, x_n)$ pour lesquels $f(x_1, x_2, \ldots, x_n) = 1$.
\end{idee}

Un automate qui fait cela est l'arbre de décision de $f$.

\begin{exemple}
	Si $f(x_1,x_2,x_3) = (x_1 \land x_2) \lor x_3$, $Sol(f) = 111,110,001,011,101$

	TODO: image
\end{exemple}


Pour éviter cet inconvénient, plutôt que de travailler avec les automates génériques
on travaille avec des BDD (Binary Decision Diagrams).

\begin{definition}
	Un BDD est un graphe avec une seule racine, acyclique, il possède deux feuilles,
	une étiquetée par 0 (false) et une étiquetée par 1 (true).

	Les nœuds internes (nœuds de décision) on toujours deux fils. Notamment, si $n$ est un nœud interne
	et $x_i$ la variable relative à ce nœud, on nomme $\low n$ le fils de $n$ correspondant à l'assignation
	$x_i = 0$ et $\high n$ celui correspondant à l'assignation $x_i = 1$.

	Chaque nœud a une étiquette qui est un entier $\geq 0$ et on a toujours $\etiquette n > \etiquette {\low n},  \etiquette {\high n}$.
\end{definition}


\begin{definition}
	Un BDD est dit réduit si :
	\begin{enumerate}
		\item On n'a jamais $\low n = \high n$
		\item Il n'y a pas deux sous arbres isomorphes.
	\end{enumerate}
\end{definition}

A partir d'un arbre de décision on peut obtenir un BDD réduit par la méthode suivante :
\begin{itemize}
	\item Les feuilles qui ne se correspondent pas à des états terminaux ont comme étiquette $0$.
	\item Les feuilles qui se correspondent à des états terminaux ont comme étiquette $1$.
	\item Si $n$ est un nœud interne :
	      \begin{itemize}
		      \item si $\etiquette {\low n} = \etiquette {\high n} = e$, alors $\etiquette n = e$.
		      \item s'l existe un nœud $n'$ déjà étiqueté et \tq
		            \begin{eqnarray*}
			            \etiquette {\low {n'}} &=& \etiquette {\low {n}} \\
			            \etiquette {\high{n'}} &=& \etiquette {\high{n}}
		            \end{eqnarray*}
		            Alors on pose $\etiquette n = \etiquette {n'}$
		      \item Sinon l'étiquette de $n$ est le plus petit entier qui n'a pas encore été utilisé.
	      \end{itemize}
\end{itemize}

Une fois qu'on a attribué une étiquette à tous les nœuds, on identifie les états ayant la même étiquette.






\newpage

\bibliographystyle{alpha}
\bibliography{automates}


\end{document}

