
\section{Automates finis}

\subsection{Automates finis déterministes}

\begin{definition}[automate fini déterministe]
	\AP
	Soit $\alphabet$ un "alphabet", un ""automate fini déterministe"" ("AFD") est un tuple $\AFD$ où
	\begin{itemize}
		\item $Q$ est un ensemble fini d'états
		\item $q_0 \in Q$ est appelé l'état initial
		\item $F \subseteq Q$ est l'ensemble des états finaux / acceptants
		\item $\delta : Q \times \alphabet \to Q$ est la fonction de transition
	\end{itemize}
\end{definition}


\begin{exemple} Un "automate fini déterministe" :
	\vspace{0.5cm}
	\begin{minipage}{0.5\textwidth}
		\begin{itemize}
			\item $\alphabet = \set{a, b}$
			\item $Q = \set{q_0, q_1}$
			\item $q_0$ est l'état initial
			\item $F = \set{q_0, q_1}$
			\item $\delta :
				      \left\{
				      \begin{array}{cc}
					      (q_0, a) & \mapsto q_0 \\
					      (q_0, b) & \mapsto q_1 \\
					      (q_1, a) & \mapsto q_0 \\
					      (q_1, b) & \mapsto q_1 \\
				      \end{array}
				      \right.$
		\end{itemize}
	\end{minipage}
	\begin{minipage}{0.5\textwidth}
		\begin{automata}
			\digraph[scale=0.75]{automateDet}{
				rankdir=LR;

				node [shape=circle, style=filled, color=lightblue];
				q0 [label="q_0"];
				q1 [label="q_1"];

				start [shape=point];
				start -> q0;

				q0 -> q0 [label="a"];
				q0 -> q1 [label="b"];
				q1 -> q1 [label="b"];
				q1 -> q0 [label="a"];

				q0 [shape=doublecircle];
				q1 [shape=doublecircle];
			}
		\end{automata}
	\end{minipage}
\end{exemple}

\begin{definition}[lecture d'un mot par un AFD]
	\AP
	Soit $A = \AFD$ un "AFD" qui a pour "alphabet" $\alphabet$. On définit la fonction $\kleene \delta$ par induction sur la longueur du "mot" $w$ :
	$$ \begin{array}{rcl}
			\kleene \delta : Q \times \mots & \to     & Q                               \\
			(q, \motvide)                   & \mapsto & q                               \\
			(q, wa)                         & \mapsto & \delta(\kleene \delta(q, w), a)
		\end{array} $$
	On a alors que $\kleene \delta(q, w)$ est l'état atteint par $A$ après avoir lu le "mot" $w$ depuis l'état $q$.
\end{definition}

\begin{definition}[langage reconnu par un AFD]
	\AP
	Le ""langage reconnu"" / accepté par un "AFD" $A = \AFD$ est le "langage"
	$$ \lang A = \setdef w {w \in \mots, \kleene \delta(q_0, w) \in F} $$
\end{definition}

\begin{exemple}
	Pour l'automate $A$ suivant, avec $\alphabet = \set {a,b}$, le "langage reconnu" est
	$\lang A = \lang {\kleene {(a + b)} b} = \lang {\kleene {(a + b)} b \kleene b} = \setdef {wb} {w \in \mots}$.

	\begin{center}
		\begin{automata}
			\digraph[scale=0.75]{automateDet2}{
				rankdir=LR;

				node [shape=circle, style=filled, color=lightblue];
				q0 [label="q_0"];
				q1 [label="q_1"];

				start [shape=point];
				start -> q0;

				q0 -> q0 [label="a"];
				q0 -> q1 [label="b"];
				q1 -> q1 [label="b"];
				q1 -> q0 [label="a"];

				q1 [shape=doublecircle];
			}
		\end{automata}
	\end{center}
\end{exemple}


\begin{definition}[chemin]
	\AP
	Soit $A = \AFD$ un "AFD", et $p,q \in Q$, un ""chemin"" $p \cheminD q$ est une suite
	$(p, a_0, q_1), (q_1, a_1, q_2), \ldots, (q_{n-1}, a_{n-1}, q)  \in Q^n \times \alphabet^n \times Q^n$
	telle que
	$\forall i \in \enum 1 {n-2}, \delta(q_i, a_i) = q_{i+1}$ et $\delta(p, a_0) = q_1$ et $\delta(q_{n-1}, a_{n-1}) = q$.
\end{definition}

\begin{definition}
	\AP
	Un automate $A = \AFD$ est ""accesible"" si, pour tout état de l'automate, il existe un "chemin" partant de
	$q_0$ qui mène a lui.
\end{definition}

\subsection{Automates finis non déterministes + $\motvide$-transitions}

Dans cette partie, on traite en meme temps les cas avec et sans $\motvide$-transitions. Les additions en
\darkgreen{vert} correspondent à la définitions avec $\motvide$-transitions. En ignorant cela, on retrouve la
definition d'un "automate fini non déterministe".

\begin{definition}[automate fini non déterministe \darkgreen{+ $\motvide$-transitions}]
	\AP
	Soit $\alphabet$ un alphabet, un ""automate fini non déterministe"" ("AFN") est un tuple $\AFN$ où
	\begin{itemize}
		\item $Q$ est un ensemble fini d'états
		\item $I \subseteq Q$ est l'ensemble des états initiaux
		\item $F \subseteq Q$ est l'ensemble des états finaux / acceptants
		\item $\delta : Q \times (\alphabet \darkgreen {\ \cup \set\motvide}) \to \mathcal{P}(Q)$ est la fonction de transition
	\end{itemize}
\end{definition}


\begin{exemple}
	Un "automate fini non déterministe" avec $\alphabet = \set{a, b}$ :
	\begin{center}
		\begin{automata}

			\digraph[scale=0.65]{automateNonDet}{
				rankdir=LR;

				node [shape=circle, style=filled, color=lightblue];
				q0 [label="q_0"];
				q1 [label="q_1"];
				q2 [label="q_2"];

				start [shape=point];
				start -> q0;

				q0 -> q0 [label="b"];
				q0 -> q1 [label="a"];
				q1 -> q1 [label="b"];
				q1 -> q0 [label="b"];
				q0 -> q2 [label="a"];
				q2 -> q2 [label="b"];

				q1 [shape=doublecircle];
				q2 [shape=doublecircle];
			}
		\end{automata}
	\end{center}
\end{exemple}


\begin{definition}[lecture d'un mot par un AFN]
	Soit $A = \AFN$ un "AFN" qui a pour "alphabet" $\alphabet$. On définit la fonction $\kleene \delta$ par induction sur la longueur du "mot" $w$ :
	$$ \begin{array}{rcl}
			\kleene \delta : Q \times \mots & \to     & \parts Q                                                  \\
			(q, \motvide)                   & \mapsto & \set q                                                    \\
			(q, wa)                         & \mapsto & \bigcup\limits_{p \in \kleene \delta (q, w)} \delta(p, a)
		\end{array} $$
\end{definition}


\begin{definition}[langage reconnu par un AFN]
	Le langage reconnu / accepté par un "AFN" $A = \AFN$ est le langage
	$$ \lang A = \setdef {w \in \mots} {\exists q_0 \in I, \kleene \delta(q_0, w) \cap F \neq \emptyset} $$

\end{definition}

\subsection{Déterminisation d'un AFN}
Soit $A = \AFN$ un AFN, on considère l'"AFD" $A' = \antuple{\parts Q, I, F', \delta'}$ où
\begin{itemize}
	\item $F' = \setdef {q \in \parts Q} {q \cap F \neq \emptyset}$
	      \vspace{0.25cm}
	\item $ \begin{array}{rcl}
			      \delta' : \parts Q \times \alphabet & \to     & \parts Q                              \\
			      (Q, a)                              & \mapsto & \bigcup\limits_{p \in Q} \delta(p, a)
		      \end{array} $
\end{itemize}

\vspace{0.25cm}

\AP
Ce processus est appelé ""déterminisation"" d'un "AFN" et nous permet de transformer un "AFN" en un "AFD" équivalent.

\begin{exemple}
	"Déterminisation" d'un "automate non déterministe" :

	\begin{minipage}{0.4\textwidth}
		\begin{center}
			Automate non déterministe :

			\begin{automata}
				\digraph[scale=0.65]{automateNonDet2}{
					rankdir=LR;

					node [shape=circle, style=filled, color=lightblue];
					q0 [label="q_0"];
					q1 [label="q_1"];

					start [shape=point];
					start -> q0;

					q0 -> q0 [label="a"];
					q0 -> q1 [label="a,b"];
					q1 -> q1 [label="a,b"];

					q1 [shape=doublecircle];
				}
			\end{automata}
		\end{center}
	\end{minipage}
	\begin{minipage}{0.6\textwidth}
		\begin{center}
			Automate déterminisé :

			\begin{automata}
				\digraph[scale=0.5]{automateDet3}{
				rankdir=LR;

				node [shape=circle, style=filled, color=lightblue];
				q0 [label="{q_0}"];
				q0q1 [label="{q_0, q_1}"];
				q1 [label="{q_1}"];

				start [shape=point];
				start -> q0;

				q0 -> q0q1 [label="a"];
				q0q1 -> q0q1 [label="q"];
				q0q1 -> q1 [label="b"];
				q0 -> q1 [label="b"];
				q1 -> q1 [label="a,b"];

				q0q1 [shape=doublecircle];
				q1 [shape=doublecircle];
				}
			\end{automata}
		\end{center}
	\end{minipage}
\end{exemple}


\begin{theorem}
	Soit $A$ un "automate fini non déterministe" avec des $\motvide$-transitions, alors il existe un "automate fini déterministe" $A'$, \tq $\lang A = \lang {A'}$.
\end{theorem}

\subsection{Équivalence entre expressions rationnelles et automates finis déterministes}

\begin{theorem}[Injection des expressions rationnelles vers les automates finis déterministes]
	Soit $r \in \erat$, alors il existe un "automate fini déterministe" $N(r)$ \tq $\lang A = \lang r$.
\end{theorem}

\begin{proof}
	Nous allons construire un tel automate par induction sur la structure de $r$ cependant la preuve du fait que cet
	automate reconnaît le langage associé à $r$ est omise. Cette construction est appelée \textbf{construction de Thompson}.

	\begin{twoautomata}
		\digraph[scale=0.5]{thompson1}{
			rankdir=LR;
			node [shape=circle, style=filled, color=lightblue];
			q0 [label=" "];
			q1 [label=" "];
			start [shape=point];
			start -> q0; q0 -> q1 [label="ε"];
			q1 [shape=doublecircle];
		}
		\caption*{$N(\motvide)$}
	\end{twoautomata}
	\begin{twoautomata}
		\digraph[scale=0.5]{thompson2}{
			rankdir=LR;
			node [shape=circle, style=filled, color=lightblue];

			start [shape=point];
			q0 [label=" "];
			q1 [label=" ", shape=doublecircle];

			start -> q0;
			q0 -> q1 [style=invis];
		}
		\caption*{$N(\emptyset)$}
	\end{twoautomata}

	\begin{twoautomata}
		\digraph[scale=0.5]{thompson3}{
			rankdir=LR;
			node [shape=circle, style=filled, color=lightblue];
			q0 [label=" "];
			q1 [label=" "];
			start [shape=point];
			start -> q0; q0 -> q1 [label="a"];
			q1 [shape=doublecircle];
		}
		\caption*{$N(a), a \in \alphabet$}
	\end{twoautomata}
	\begin{twoautomata}
		\digraph[scale=0.5]{thompson4}{
		rankdir=LR;

		node [shape=circle, style=filled, color=lightblue];

		subgraph cluster_r {
		style=rounded;
		label="N(r)";
		color=lightcoral;

		nrin [label=" "];
		nrout [label=" "];
		nrin -> nrout [style=invis];
		}


		subgraph cluster_s {
		style=rounded;
		label="N(s)";
		color=lightgreen;

		nsin [label=" "];
		nsout [label=" "];
		nsin -> nsout [style=invis];
		}


		q0 [label=" "];
		q1 [label=" "];
		start [shape=point];

		start -> q0;
		q0 -> nrin [label="ε"];
		q0 -> nsin [label="ε"];

		nrout -> q1 [label="ε"];
		nsout -> q1 [label="ε"];

		q1 [shape=doublecircle];
		}

		\caption*{$N(r + s)$}
	\end{twoautomata}

	\begin{twoautomata}
		\digraph[scale=0.5]{thompson5}{
		rankdir=LR;

		node [shape=circle, style=filled, color=lightblue];

		subgraph cluster_r {
		style=rounded;
		label="N(r)";
		color=lightcoral;

		nrin [label=" "];
		nrout [label=" "];
		nrin -> nrout [style=invis];
		}

		subgraph cluster_s {
		style=rounded;
		label="N(s)";
		color=lightgreen;

		nsin  [label=" "];
		nsout [label=" "];
		nsin -> nsout [style=invis];
		}

		start [shape=point];

		start -> nrin;
		nrout -> nsin [label="ε"];
		nsout [shape=doublecircle];
		}
		\caption*{$N(rs)$}
	\end{twoautomata}
	\begin{twoautomata}
		\digraph[scale=0.5]{thompson6}{
		rankdir=LR;

		node [shape=circle, style=filled, color=lightblue];

		subgraph cluster_r {
		style=rounded;
		label="N(r)";
		color=lightcoral;

		nrin [label=" "];
		nrout [label=" "];
		nrin -> nrout [style=invis];
		}

		start [shape=point];

		q0 [label=" "];
		q1 [label=" "];
		start [shape=point];
		start -> q0;
		q0 -> q1 [label="ε"];
		q0 -> nrin [label="ε"];
		nrout -> nrin [label="ε"];
		nrout -> q1 [label="ε"];

		q1 [shape=doublecircle];
		}

		\caption*{$N(\kleene r)$}
	\end{twoautomata}

	En combinant ces constructions, on peut construire un "automate fini non déterministe" pour n'importe quelle
	"expression rationnelle" qui peut être "determinisé" pour obtenir un "automate fini déterministe".

\end{proof}
