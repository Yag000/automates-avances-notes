\section{Évaluation de fonctions booléennes}

\begin{notation}
	On note $\B = \set {0,1}$.
\end{notation}

On veut évaluer les fonctions $f : \B ^ n \to \B$

\begin{exemple}
	Une telle fonction est $f(x_1,x_2,x_3) = (x_1 \land x_2) \lor x_3 : \B^3 \to \B$
\end{exemple}

Une fonction $f : \B ^ n \to B$ est représentable avec $2^k$ bits
(on donne les images des $2^n$ $n$-uplets sur $\B$).
Donc au total il y a $2^{2^n}$ fonctions $\B^n \to \B$.

\begin{idee}
	Utiliser des automates qui reconnaissent le langage des $n$-uplets
	$(x_1,x_2,\ldots, x_n)$ pour lesquels $f(x_1, x_2, \ldots, x_n) = 1$.
\end{idee}

Un automate qui fait cela est l'arbre de décision de $f$.

\begin{exemple}
	Si $f(x_1,x_2,x_3) = (x_1 \land x_2) \lor x_3$, $Sol(f) = 111,110,001,011,101$

	TODO: image
\end{exemple}


Pour éviter cet inconvénient, plutôt que de travailler avec les automates génériques
on travaille avec des BDD (Binary Decision Diagrams).

\begin{definition}
	Un BDD est un graphe avec une seule racine, acyclique, il possède deux feuilles,
	une étiquetée par 0 (false) et une étiquetée par 1 (true).

	Les nœuds internes (nœuds de décision) on toujours deux fils. Notamment, si $n$ est un nœud interne
	et $x_i$ la variable relative à ce nœud, on nomme $\low n$ le fils de $n$ correspondant à l'assignation
	$x_i = 0$ et $\high n$ celui correspondant à l'assignation $x_i = 1$.

	Chaque nœud a une étiquette qui est un entier $\geq 0$ et on a toujours $\etiquette n > \etiquette {\low n},  \etiquette {\high n}$.
\end{definition}


\begin{definition}
	Un BDD est dit réduit si :
	\begin{enumerate}
		\item On n'a jamais $\low n = \high n$
		\item Il n'y a pas deux sous arbres isomorphes.
	\end{enumerate}
\end{definition}

A partir d'un arbre de décision on peut obtenir un BDD réduit par la méthode suivante :
\begin{itemize}
	\item Les feuilles qui ne se correspondent pas à des états terminaux ont comme étiquette $0$.
	\item Les feuilles qui se correspondent à des états terminaux ont comme étiquette $1$.
	\item Si $n$ est un nœud interne :
	      \begin{itemize}
		      \item si $\etiquette {\low n} = \etiquette {\high n} = e$, alors $\etiquette n = e$.
		      \item s'l existe un nœud $n'$ déjà étiqueté et \tq
		            \begin{eqnarray*}
			            \etiquette {\low {n'}} &=& \etiquette {\low {n}} \\
			            \etiquette {\high{n'}} &=& \etiquette {\high{n}}
		            \end{eqnarray*}
		            Alors on pose $\etiquette n = \etiquette {n'}$
		      \item Sinon l'étiquette de $n$ est le plus petit entier qui n'a pas encore été utilisé.
	      \end{itemize}
\end{itemize}

Une fois qu'on a attribué une étiquette à tous les nœuds, on identifie les états ayant la même étiquette.



