\section{Langages de mots infinis et automates de Büchi}

\subsection{Langages de mot infinis}

\begin{notation}
	Soit $A$ un alphabet.

	On note $\aomega$ comme l'ensemble des mots infinis sur $A$
	(un mot infini est une fonction $: \N \to A$).

	On note $\ainf = A^* \cup \aomega$.

	On étend la définition de certaines notions:

	Soit $L \subseteq A^*$ un langage (de mots finis), on définit
	$\lomega = \setdef {x_0x_1\cdots} {x_i \in L \setminus \set {\motvide}}$
	( généralisation de l'opérateur $*$ à une concaténation d'un nombre infini de mots de $L$.)
\end{notation}


\begin{definition}[Langages $\omega$-rationnels]
	La classe des langages $\omega$-rationnels de $\ainf$ est la \emph{plus petite}
	classe $R$ de sous-ensembles de $\ainf$ qui satisfait :
	\begin{enumerate}
		\item $\emptyset \in R \et \forall a \in \Sigma, \set a \in R$;
		\item Close par union finie;
		\item $\forall X \in R \cap A^*, \forall Y \in R \cap \ainf, X \cdot Y \in R$; \label{omega-clot-fi}
		\item $\forall X \in R \cap A^*, X^* \in R \et X^{\omega} \in R$.\label{omega-clot-op}
	\end{enumerate}
\end{definition}

\begin{remarque}
	La classe $R$ contient en particulier tous les langages rationnels de $A^*$.
\end{remarque}

\begin{exemple}
	$A = \set {a,b}$.

	\begin{itemize}
		\item $L = \setdef {w \in \aomega} {w \text{ a un nombre fini de } b}$, on peut le définir par l'expression
		      $\omega$-rationnelle $(a + b)^* a^{\omega}$ ou $(a^*b)^* a^{\omega}$,
		\item $L = \setdef {w \in \aomega} {w \text{ commence par } a \et \text{a un nombre infini de } b}$,
		      avec l'expression : $a(a^*b)^{\omega}$.
	\end{itemize}

\end{exemple}

\begin{prop}[Caractérisation des langages $\omega$-rationnels]
	Un langage $L \subseteq \ainf$ est $w$-rationnel \ssi $L$ est l'union finie de langages
	de la forme $XY^{\omega}$ avec $X,Y \in R \cap A^*$ ($X,Y$ sont des langages rationnels \quotes{classiques}).
\end{prop}

\begin{proofI}
	Soit $\Delta$ la classe des langages obtenue comme union finie de langages de la forme $XY^{\omega}, X,Y \in Rat(A^*)$.
	\begin{itemize}
		\item \fbox{$\Delta \subseteq R$} \\
		      Par définition de $R$
		      $$Y^{\omega} \in R \reason {par \ref{omega-clot-op}}$$
		      et
		      $$XY^{\omega} \in R \reason{par \ref{omega-clot-fi}}$$
		      et puisque $R$ est fermé relativement à l'union finie, on a que tous les langages de $\Delta$ sont dans $R$,
		      on a $\Delta \subseteq R$.
		\item \fbox{$R \subseteq \Delta$} \\

		      Pour montrer l'inclusion inverse, on définit une classe $\mathcal E$.
		      Avec
		      \begin{equation*}
			      L \in \mathcal E \iff
			      \left\{
			      \begin{array}{c}
				      L \cap A^* \in \text{Rat}(A^*) \\
				      L \cap \aomega \in \Delta
			      \end{array}
			      \right.
		      \end{equation*}
		      et on montre que $R \subseteq \mathcal E \subseteq \Delta$.
		      Il faut vérifier que
		      \begin{enumerate}
			      \item $\emptyset \in \mathcal{E}$
			      \item $\forall a \in A, \set a \in  \mathcal{E}$
			      \item $\mathcal{E}$ est fermé par union finie

			      \item $\forall X \in \mathcal{E} \cap A^*, \forall Y \in \mathcal{E} \cap \ainf, X \cdot Y \in \mathcal{E}$
			      \item $\forall X \in \mathcal{E} \cap A^*, X^* \in \mathcal{E} \et X^{\omega} \in \mathcal{E}$
		      \end{enumerate}
		      La vérification de ces propriétés est laissée en exercice.

		      Puisque $R$ est définie comme la \underline{plus petite} classe qui satisfait
		      les propriétés précédentes, on en déduit que $R \subseteq \mathcal{E}$.
	\end{itemize}
\end{proofI}


\subsection{Automates de Büchi}

\begin{definition}
	Un ""automate de Büchi"" est un quadruplet $(Q,I,T,\delta)$, avec $\abs Q < \infty$, où
	$Q,I,T,\delta$ ont la même signification que pour les automates finis.
\end{definition}

\begin{definition}
	Un chemin infini est une suite infinie $(q_i,a_i,q_{i+1})$ tel que $\forall i, (q_i,a_i,q_{i+1}) \in \delta$ et le mot
	infini $a_0a_1\ldots$ est dit l'étiquette du chemin.
\end{definition}


\begin{definition}
	Un chemin $(q_0, a_0, q_1), (q_1, a_1, q_2), \ldots $ est ""réussi"" (ou acceptant) si $q_0 \in I$ et si $\exists \text{ une suite infinie d'indices } q_{i_J} \in T$.

	Donc un chemin est "réussi" si et seulement s'il passe un nombre infini de fois par des états terminaux.

	On note en particulier qu'il doit exister un état $t \in T$ par lequel le chemin passe un nombre infini de fois.
\end{definition}

\begin{definition}
	Un mot $w \in \aomega$ est accepté par un "automate de Büchi" si et seulement s'il est l'étiquette d'un chemin "réussi".
\end{definition}


\begin{definition}
	Un langage $L$ de $\aomega$ est reconnaissable s'il existe un "automate de Büchi" qui accepte exactement les mots de $L$.
\end{definition}


\begin{exemple}
	L'automate suivant reconnait les mots qui commencent par $a$ et qui ont un nombre infini de $b$ :

	\begin{automata}
		\digraph[scale=0.50]{buchi1}{
			rankdir=LR;

			node [shape=circle, style=filled, color=lightblue];
			0;
			1;

			start [shape=point];
			start -> 0;

			0 -> 0 [label="a"];
			0 -> 1 [label="a"];
			1 -> 1 [label="b"];
			1 -> 0 [label="b"];

			1 [shape=doublecircle];
		}
	\end{automata}
\end{exemple}


\begin{exercice}
	Écrire un "automate de Büchi" qui reconnait les mots infinis sur $\set {a,b}$ ayant un nombre fini de $b$.

	Trois automates possibles : \\
	\begin{minipage}{0.3\textwidth}
		\begin{automata}
			\digraph[scale=0.50]{buchi2}{
				rankdir=LR;

				node [shape=circle, style=filled, color=lightblue];
				0;
				1;

				start [shape=point];
				start -> 0;

				0 -> 0 [label="a,b"];
				0 -> 1 [label="a"];
				1 -> 1 [label="a"];

				1 [shape=doublecircle];
			}
		\end{automata}
	\end{minipage}
	\begin{minipage}{0.3\textwidth}
		\begin{automata}
			\digraph[scale=0.50]{buchi3}{
				rankdir=LR;

				node [shape=circle, style=filled, color=lightblue];
				0;
				1;

				start [shape=point];
				start -> 0;

				0 -> 0 [label="a,b"];
				0 -> 1 [label="a,b"];
				1 -> 1 [label="a"];

				1 [shape=doublecircle];
			}
		\end{automata}
	\end{minipage}
	\begin{minipage}{0.3\textwidth}
		\begin{automata}
			\digraph[scale=0.50]{buchi4}{
				rankdir=LR;

				node [shape=circle, style=filled, color=lightblue];
				0;
				1;

				start [shape=point];
				start -> 0;

				0 -> 0 [label="a,b"];
				0 -> 1 [label="b"];
				1 -> 1 [label="a"];

				1 [shape=doublecircle];
			}
		\end{automata}
	\end{minipage}
\end{exercice}


\begin{definition}
	Un automate est dit émondé si tous ses états sont accessibles (on peut y accéder à partir d'un état initial)
	et co-accessibles (à partir de n'importe quel état, on peut arriver à un état final).
\end{definition}

Les notions d'automate émondé, complet, déterministe se généralisent aux "automates de Büchi".


Mais alors que les algorithmes pour rendre émondé ou complet un automate marchent aussi
pour les "automates de Büchi", il n'existe pas d'algorithme pour "déterminiser" un "automate de Büchi".

En fait, il existe des langages reconnaissables de $\aomega$ qui ne sont pas reconnus par un "automate de Büchi" déterministe.


La classe des langages reconnaissables par un "automate de Büchi" est fermée relativement à l'union finie
et la méthode pour construire l'automate qui reconnait l'union est la même que pour les automates finis (on \quotes{met ensemble}
les automates).

On verra que cette classe est aussi fermée relativement au complémentaire et à l'intersection.


\begin{definition}
	Un automate est dit normalisé s'il a

	\begin{itemize}
		\item 1 seul état initial sans flèches rentrantes
		\item 1 seul état final sans flèches sortantes
	\end{itemize}
\end{definition}

\begin{theorem}[lien avec le théorème de Kleene]
	$L \subseteq \aomega$ est $\omega$-rationnel si et seulement s'il est reconnu par un "automate de Büchi".
\end{theorem}

\begin{proofI}
	\begin{itemize}
		\item \bimpRL \\
		      Soit $L$ reconnaissable, il existe $A = (Q,I,T,\delta)$ tel que $L = \La^{\omega}(A)$.

		      Soient $q,q' \in Q$. Pour chaque couple $(q,q')$, on définit $A_{qq'} = (Q,q,q',\delta)$

		      Soit $\La^*(A_{qq'})$ le langage reconnu par $A_{qq'}$ et $\La^+(A_{qq'}) = \La^*(A_{qq'}) \setminus \set {\motvide}$

		      Alors $L = \bigcup_{(q,q') \in I \times T} \La^*(q,q')(\La^+(q',q'))^{\omega}$. Montrons cette égalité.

		      \begin{eqnarray*}
			      w \in \bigcup_{(q,q') \in I \times T} \La^*(q,q')(\La^+(q',q'))^{\omega} &\iff& \exists i_0 \in I, \exists t_0 \in T, \text{ tels que } w \in \La^*(i_0,t_0)(\La^+(t_0,t_0))^{\omega} \\
			      &\iff& w \text{ est l'étiquette d'un chemin qui passe } \\
			      & & \text{ un nombre infini de fois par l'état final } t_0 \\
			      &\iff& w\in \La^w (A) = L
		      \end{eqnarray*}

		\item \bimpLR \\
		      Au vu de la caractérisation et du fait que la classe Rec est fermée relativement à l'union finie, il suffit de \mq si
		      $X,Y \in \text{Rat}(A^+)$, alors $XY^{\omega} \in \text{Rec}$.

		      D'après le théorème de Kleene, $X,Y \in \text{Rat} (A^*)$, alors $X,Y$ sont reconnaissables par des automates finis, donc

		      \begin{eqnarray*}
			      &\exists A_1 \text{ \tq } X = \lang {A_1} \\
			      &\exists A_2 \text{ \tq } Y = \lang {A_2}
		      \end{eqnarray*}

		      De plus on peut choisir $A_1$ et $A_2$ normalisés.

		      Soit $A_1 = (Q_1, i_1, f_1, \delta_1)$ et $A_2 = (Q_2, i_2, f_2, \delta_2)$.
		      Donc les automates ont la forme suivante :

		      \begin{automata}
			      \digraph[scale=0.50]{buchithm1}{
				      rankdir=LR;

				      node [shape=circle, style=filled, color=lightblue];
				      i1;
				      f1;
				      i2;
				      f2;

				      node [shape=box, style=filled, color=lightblue];
				      a1 [label = "A1"]
				      a2 [label = "A2"]

				      start [shape=point];
				      start -> i1;


				      start2 [shape=point];
				      start2 -> i2;

				      i1 -> a1;
				      i1 -> a1;
				      a1 -> f1;
				      a1 -> f1;


				      i2 -> a2;
				      i2 -> a2;
				      a2 -> f2;
				      a2 -> f2;

				      f1 [shape=doublecircle];
				      f2 [shape=doublecircle];
			      }
		      \end{automata}

		      Supposons d'abord que $\motvide \notin X$.

		      Alors on identifie les états $f_1,i_2$ et $f_2$ :

		      \begin{automata}
			      \digraph[scale=0.50]{buchithm2}{
				      rankdir=LR;

				      node [shape=circle, style=filled, color=lightblue];
				      i1;
				      end;

				      node [shape=box, style=filled, color=lightblue];
				      a1 [label = "A1"]
				      a2 [label = "A2"]

				      start [shape=point];
				      start -> i1;

				      i1 -> a1;
				      i1 -> a1;
				      a1 -> end;
				      a1 -> end;


				      end -> a2;
				      end -> a2;
				      a2 -> end;
				      a2 -> end;

				      end [shape=doublecircle];
			      }
		      \end{automata}

		      L'automate obtenu reconnait bien les mots $XY^{\omega}$ et il es défini formellement comme suit :
		      $$ A = ( Q_1 \cup Q_2 \setminus \set {i_2,f_2}, \set {i_1}, \set {f_1}, \delta_1 \cup \delta_2 \cup \eta_0 \cup \eta_1 \cup \eta_2)$$
		      où :
		      \begin{eqnarray*}
			      \eta_0 &=& \setdef {(f_1,a,f_1)} {(i_2,a,f_2) \in \delta_2} \\
			      \eta_1 &=& \setdef {(f_1,a,q)} {(i_2,a,q) \in \delta_2} \\
			      \eta_2 &=& \setdef {(q,a,f_1)} {(q,a,f_2) \in \delta_2}
		      \end{eqnarray*}


		      Dans le cas où $\motvide \in L_1$, alors on rend $f_1$ aussi initial.
	\end{itemize}
\end{proofI}

\begin{exemple}
	$L_1 = \motvide, L_2 = \lang {a^*b}$

	L'automate classique pour $L_2$ :

	\begin{automata}
		\digraph[scale=0.50]{buchi5}{
			rankdir=LR;

			node [shape=circle, style=filled, color=lightblue];
			i [label="i"];
			f [label="f"];

			start [shape=point];
			start -> i;

			i -> i [label="a"];
			i -> f [label="b"];

			f [shape=doublecircle];
		}
	\end{automata}

	On normalise l'automate en ajoutant un nouvel état initial $qnew$ avec les mêmes transitions sortantes que l'ancien état initial.

	\begin{automata}
		\digraph[scale=0.50]{buchi6}{
			rankdir=LR;

			node [shape=circle, style=filled, color=lightblue];
			q [label="qnew"];
			i [label="i"];
			f [label="f"];

			start [shape=point];
			start -> q;

			q -> i [label="a"];
			q -> f [label="b"];
			i -> i [label="a"];
			i -> f [label="b"];

			f [shape=doublecircle];
		}
	\end{automata}

	Ensuite on identifie les états $qnew$ et $f$ :

	\begin{automata}
		\digraph[scale=0.50]{buchi7}{
			rankdir=LR;

			node [shape=circle, style=filled, color=lightblue];
			q [label="qnew"];
			i [label="i"];

			start [shape=point];
			start -> q;

			q -> q [label="b"];
			q -> i [label="a"];

			i -> i [label="a"];
			i -> q [label="b"];

			q [shape=doublecircle];
		}
	\end{automata}
\end{exemple}


\begin{definition}
	Soit $L \subseteq \mots$ un langage de mots finis. On note :
	$$\larr L = \setdef {w \in \alphabet^ {\omega}} {w \text{ a une infinité de préfixes dans } L}$$
\end{definition}


\begin{exemple}
	$ $ \par\nobreak\ignorespaces
	\begin{enumerate}
		\item $L = a^*b, \larr L = \emptyset$ (aucun mot infini)
		\item $L = (ab)^*, \larr L = (ab)^{\omega}$ (un seul mot)
		\item $L = (a^*b)^+, \larr L =  (a^*b)^{\omega}$
	\end{enumerate}
\end{exemple}


\begin{prop}[Admis]
	Soit $A = (Q,I,T,\delta)$ un "automate de Büchi" déterministe, alors $\La^{\omega} (A) = \larr {\La^+ (A)}$
	où $\La^+ (A)$ est le langage des mots finis reconnus par $A$ comme un automate fini traditionnel.
\end{prop}


\begin{coro}
	$X \subseteq \mots$ est reconnu par un "automate de Büchi" déterministe \ssi $\exists L \in \text{Rat}(\mots)$
	\tq $X = \larr L$.
\end{coro}

\begin{prop}
	Le langage $X = (a+b)^*a^{\omega}$ ($ {\abs w}_b < \infty $) ne peut pas être reconnu par un "automate de Büchi" déterministe.
\end{prop}

\begin{proof}
	Par l'absurde, s'il y a un automate déterministe $\exists  L \in \text{Rat}(\alphabet)$ \tq tout mot de $X$ a un nombre infini de préfixes dans $L$.

	\begin{itemize}
		\item Prenons le mot $ba^{\omega}$, il a un préfixe $ba^{i_1} \in L$.
		\item Prenons le mot $ba^{i_1}ba^{\omega}$, il a un préfixe $ba^{i_1}ba^{i_2} \in L$.
		\item Prenons le mot $ba^{i_1}ba^{i_2}a^{\omega}$, il a un préfixe $ba^{i_1}ba^{i_2}ba^{i_3} \in L$.
	\end{itemize}

	Alors on peut construire un mot infini :
	$$ ba^{i_1}ba^{i_2}\ldots ba^{i_k}\ldots$$
	qui est l'étiquette d'un chemin qui passe un nombre infini de fois par un état final.

	Ce mot serait reconnu par l'automate, or il contient un nombre infini de $b$.  $\contradict$
\end{proof}


\subsection{Clôture sous intersection et complémentaire}

\begin{prop}
	La famille des langages reconnaissables par un "automate de Büchi" est fermée relativement à l'intersection.
\end{prop}

\begin{proof}
	Soit $L_1$ et $L_2$ deux langages reconnaissable et soient
	\begin{eqnarray*}
		A_1 &=& (Q_1,I_1,T_1.\delta_1) \\
		A_2 &=& (Q_2,I_2,T_2.\delta_2)
	\end{eqnarray*}
	tels que $L_1 = \La ^ {\omega} (A_1) \et  L_2 = \La ^ {\omega} (A_2)$.

	Alors l'automate $A = (Q,I,T,\delta)$ :
	\begin{eqnarray*}
		Q &=& Q_1 \times Q_2 \times \set{1,2}  \\
		I &=& \setdef {(i_1,i_2,1)} {i_1 \in I_1,i_2 \in I_2 } \\
		T &=&  \setdef {(t,q,1)} {t \in T_1, q \in Q} \\
		\delta ((q_1,q_2,x), a) &=&
		\left\{
		\begin{array}{cc}
			(\delta_1(q_1,a), \delta_2 (q_2,a), 2) & \text {si } q_1 \in T_1 \et x = 1 \\
			(\delta_1(q_1,a), \delta_2 (q_2,a), 1) & \text {si } q_2 \in T_2 \et x = 2 \\
			(\delta_1(q_1,a), \delta_2 (q_2,a), x) & \text {sinon}
		\end{array}
		\right.
	\end{eqnarray*}
\end{proof}

\begin{exemple}

	Soit

	\begin{itemize}
		\item $L_1 = \setdef {u \in {\set {a,b}}^{\omega}} {{\abs u}_a = \infty}$

		      \begin{automata}
			      \digraph[scale=0.50]{buchi8}{
				      rankdir=LR;

				      node [shape=circle, style=filled, color=lightblue];
				      0 [label="0"];
				      1 [label="1"];

				      start [shape=point];
				      start -> 0;

				      0 -> 0 [label="a"];
				      0 -> 1 [label="b"];

				      1 -> 1 [label="b"];
				      1 -> 0 [label="a"];

				      0 [shape=doublecircle];
			      }
		      \end{automata}

		\item $L_2 = \setdef {u \in {\set {a,b}}^{\omega}} {{\abs u}_b = \infty}$

		      \begin{automata}
			      \digraph[scale=0.50]{buchi9}{
				      rankdir=LR;

				      node [shape=circle, style=filled, color=lightblue];

				      0 [label="2"];
				      1 [label="3"];

				      start [shape=point];
				      start -> 0;

				      0 -> 0 [label="b"];
				      0 -> 1 [label="a"];

				      1 -> 1 [label="a"];
				      1 -> 0 [label="b"];

				      0 [shape=doublecircle];
			      }
		      \end{automata}

		\item On cherche $L_3 =  L_1 \cap L_2$

		      \begin{automata}
			      \digraph[scale=0.50]{buchi10}{
				      rankdir=LR;

				      node [shape=circle, style=filled, color=lightblue];

				      021 [label = "0,2,1"];
				      032 [label = "0,3,2"];
				      122 [label = "1,2,2"];
				      031 [label = "0,3,1"];
				      121 [label = "1,2,1"];

				      start [shape=point];
				      start -> 021;

				      021 -> 032 [label="a"];
				      021 -> 122 [label="b"];

				      032 -> 032 [label="a"];
				      032 -> 122 [label="b"];


				      122 -> 031 [label="a"];
				      122 -> 121 [label="b"];


				      031 -> 032 [label="a"];
				      031 -> 122 [label="b"];

				      121 -> 031 [label="a"];
				      121 -> 121 [label="b"];

				      021 [shape=doublecircle];
				      031 [shape=doublecircle];
			      }
		      \end{automata}
	\end{itemize}
\end{exemple}

\begin{remarque}
	Le "langage" $L_3 = \setdef {u \in {\set {a,b}}^{\omega}} {{\abs u}_a = \infty \et {\abs u}_b = \infty}$, \ie,
	le "langage" de mots de ${\set {a,b}}^{\omega}$ avec un nombre infini de $a$ et de $b$, reconnu par l'automate suivant :

	\begin{automata}
		\digraph[scale=0.50]{buchi123}{
			rankdir=LR;

			node [shape=circle, style=filled, color=lightblue];

			0 [label=" "];
			1 [label=" "];

			start [shape=point];
			start -> 0;

			0 -> 0 [label="b"];
			0 -> 1 [label="a"];

			1 -> 1 [label="a"];
			1 -> 0 [label="b"];

			1 [shape=doublecircle];
		}
	\end{automata}
\end{remarque}

\begin{notation}
	Soit $w \in \mots$, on note $\inf (w) = \setdef {x \in \alphabet} {{\abs w}_x = \infty}$
\end{notation}

\begin{theorem}[Factorisation Ramseyenne]
	Soit $f : \N \times \N \to C $, où $C$ est un ensemble fini, alors il existe une suite infinie d'entiers $E = i_0, i_1, \ldots, i_n, \ldots$
	telle que $f(E \times E)$ soit un singleton, \ie $\forall i_{j_1},i_{e_1},i_{j_2},i_{e_2}$ dans $E$, $f(i_{j_1},i_{e_1})=f(i_{j_2},i_{e_2})$.
\end{theorem}

\begin{proof}
	On construit une suite de sous-ensembles de $\N$, $T_0, T_1, \ldots$, de la maniere suivante :
	$T_0 = \N$. Si on a déjà $T_i$, on construit $T_{i+1}$ comme suit.

	Puisque les couleurs sont en nombre fini, il existe une couleur $c_0$ et une suite
	$u_0, v_1, v_2, \ldots$ tel que $f(u_0, v_i) = c_0 \forall i \geq 1$ (il existe un nombre infini d'arcs sortant de $u_0$ de couleur $c_0$)
	alors $T_{i+1} = \set {v_1, v_2, \ldots}$


	Soit $u_i = \min (T_1)$ et considérons $u_0, u_1, \ldots$. Par construction $\forall i \in \N, \forall j \in \N, f(u_i, u_{i + j}) = c_i$
	puisque les couleurs sont en nombre fini, il doit exister une suite $i_0, i_1, \ldots$ telle que $f(u_{i_J}, u_{i_{J+1}}) = c_{i_0}$
	et donc la suite $u_{i_0},u_{i_1}, \ldots$ est la suite cherchée.
\end{proof}


\begin{notation}
	Soit $A = \AFN$ un automate, alors pour un d'états $p,q \in Q$ et un mot $w \in \mots$, alors
	on \intro[cheminB]{note} $\chemin p q w$ s'il existe un chemin de $p$ à $q$ avec $w$ comme étiquette et on note
	$\cheminT p q w$ si le chemin passe par au moins un état terminal.
\end{notation}


\begin{definition}
	Soit $X$ un langage reconnaissable et $A = (Q,I,T,\delta)$ un "automate de Büchi" qui reconnait $X$.

	On définit une relation sur $\mots$:
	\begin{eqnarray*}
		w \sim w' \iff &\forall p,q \in Q, \chemin p q w \iff \chemin p q {w'} \\
		&\et \cheminT p q w \iff \cheminT p q {w'}
	\end{eqnarray*}
\end{definition}

\begin{prop}
	\begin{enumerate}
		\item La relation $\sim$ est une relation d'équivalence.
		\item La relation $\sim$ est une relation d'équivalence d'index fini, son nombre de classes ne peut pas dépasser $3^{{\abs Q}^2}$.
		\item La relation $\sim$ est une congruence.
	\end{enumerate}
\end{prop}

\begin{proofI}
	\begin{enumerate}
		\item  trivial
		\item Chacune des ${\abs Q}^2$ coupes d'états distingue 3 types de mots :
		      \begin{itemize}
			      \item les mots qui ne sont pas l'étiquette d'un chemin de $p$ à $q$,
			      \item les mots qui sont l'étiquette d'un chemin $s$ de $p$ à $q$ qui ne passe par aucun état terminal,
			      \item les mots qui sont l'étiquette d'un chemin $s$ de $p$ à $q$ qui passe par au moins 1 un état terminal.
		      \end{itemize}
		\item $\sim$ est une congruence.

		      Si $w \sim w'$ et $u,v \in \mots$ alors $uwv \sim uw'v$

		      TODO
	\end{enumerate}
\end{proofI}

\begin{remarque}
	$\mots / \sim$ est un monoïde fini.

	Et donc $\forall w \in \mots, [w]$ est un langage rationnel, en effet, $[w]$ est l'image inverse d'un singleton de $\mots / \sim$
	par le morphisme de projection $\phi : \mots \to \mots / \sim$ tel que $\forall w \in \mots,\ f (w) = [w] $. Un singleton est rationnel et
	l'image inverse d'un rationnel par un morphisme est aussi un rationnel.
\end{remarque}


On considère la famille de langages de $\Sigma^{\omega}$ de la forme $[u][v]^{\omega}$ où $[u] \et [v]$ sont des classes d'équivalence de $\sim$.

On va \mq les langages $[u][v]^{\omega}$ constituent une partition de $\Sigma^{\omega}$ plus fine que la partition $<X,\compl{C}(X)>$,
donc que $X$ et $\compl C (X)$ sont l'union finie d'ensembles de la forme $[u][v]^{\omega}$.

\begin{prop}\label{prop:buchi1}
	Soit $Y = [u][v]^{\omega}$ pour un choix quelconque de $u \et v$, alors $Y \cap X = \emptyset$ ou $Y \subseteq X$.
\end{prop}


\begin{proof}
	Supposons $Y \cap X \neq \emptyset$ et soit $x \in Y \cap X$, alors $x$ est l'étiquette d'un chemin "réussi" de $A$
	de la forme
	$$q_0 \cheminArrow {u_0} q_1 \cheminArrow {v_1} q_2 \cheminArrow {v_2} q_3 \ldots$$
	avec $u_0 \in [u]$ et $\forall i, v_i \in [v]$, de plus,
	une infinité de chemins $\chemin {q_i} {q_{i_1}} {v_i}$ passent par au moins un état terminal.

	Soit $y \in Y = [u][v]^{\omega}$. On peut donc écrire $y$ comme :
	$$ y = u_0'v_1'v_2'\ldots$$
	avec $u_0' = [u]$ et $\forall i, v_i' = [v]$.

	Mais alors $u_0 \sim u_0'$ et $\forall i, v_i \sim v_i'$ et donc

	$$q_0 \cheminArrow {u_0'} q_1 \cheminArrow {v_1'} q_2 \cheminArrow {v_2'} q_3 \ldots$$
	avec une infinité de chemins $\chemin {q_i} {q_{i_1}} {v'_i}$ passent par au moins un état terminal grâce à l'équivalence
	$v_i \sim v_i'$.

	Ainsi $y$ est aussi l'étiquette d'un chemin "réussi" de $A$, \cad $A$ reconnait $y$, donc $y \in X$, donc $Y \subseteq X$.
\end{proof}



\begin{prop}\label{prop:buchi2}
	$\forall x \in \Sigma^{\omega}, \exists u,v \in \Sigma^*, x \in [u][v]^{\omega}$.
\end{prop}

\begin{proof}
	Écrivons $x = a_0a_1 \ldots a_n \ldots$ où $a_i \in \Sigma$.

	On définit une fonction $f : \N \times \N \to \Sigma/\sim$
	$$f(i,j) = [a_ia_{i+1}\ldots a_{j-1}], i < j$$
	par le théorème de Ramsey, il existe une suite $I = i_0,i_1,\ldots$ \tlq $f(I)$ est un singleton.
	En particulier,
	$$f(i_0,i_1) = f(i_1,i_2) = f(i_2,i_3) \ \ldots$$
	Mais alors, soit $u - q_0 \ldots q_{i_0-1}$, $v_j = a_{i+j} \ldots a_{{i_{j+1} -1}}$ et $x = a_0a_ 1\ldots a_{i_0-1}a_{i+0}$
	mais
	\begin{eqnarray*}
		f(i_0,i_1) &=& [a_{i_0} a_{i_0+1}\ldots a_{i_1-1}] \\
		f(i_1,i_2) &=& [a_{i_1} a_{i_1+1}\ldots a_{i_2-1}] \\
		f(i_2,i_3) &=& [a_{i_2} a_{i_2+1}\ldots a_{i_3-1}]
	\end{eqnarray*}
	en fait, on a que $v_j \sim v_1 \forall j \geq 1$ et donc $x \in [u][v_1]^{\omega}$.
\end{proof}

\begin{theorem}
	$\forall X$ $\omega$-rationnel, on a
	$$ \compl C(X) = \bigcup_{ \substack{ (u,v) \mid \\ [u][v]^{\omega} \cap X = \emptyset} } [u][v]^{\omega}$$
\end{theorem}

\begin{proofI}
	\begin{itemize}
		\item \fbox{$\supseteq$} \\
		      Évidente car on fait une union d'ensembles qui sont tous contenus dans $\compl C (X)$.
		\item \fbox{$\subseteq$} \\
		      Soit $y \in \compl C (X)$.
		      $$\exists u,v,\text{ tels que } y \in [u][v]^{\omega} \reason{Propositon \ref{prop:buchi2}}$$
		      alors $[u][v]^{\omega} \cap \compl C (X) \neq \emptyset$ et donc par la  Proposition \ref{prop:buchi1}
		      on a $[u][v]^{\omega} \cap X = \emptyset$ et donc $y$ est bien dans l'un des ensembles de l'union.
	\end{itemize}
\end{proofI}

\begin{coro}
	La classe des langages reconnaissables par un "automate de Büchi" est fermée relativement au passage au complémentaire.
\end{coro}


\subsection{Automates de Muller}

Pour résoudre le problème de l'absence d'un algorithme de détermination des "automates de Büchi" on introduit des
généralisations comme les automates de Muller.

\begin{definition}
	Un automate de Muller est un quadruplet $(Q,I, \mathcal T, \sigma)$ où $\mathcal T$ (la table de l'automate) est un ensemble de sous-ensembles
	de $Q$ : $\mathcal T \subseteq \parts Q$.

	Soit $w \in \Sigma^{\omega}$, $w$ est accepté par $A$ si $w$ est l'étiquette d'un chemin $\gamma$ \tq $\inf ({\gamma}) \in \mathcal T$ où
	$\inf ({\gamma})$ est l'ensemble d'états de $Q$ par qui $\gamma$ passe infiniment souvent.
\end{definition}


\begin{exemple}
	L'automate suivant

	\begin{automata}
		\digraph[scale=0.75]{muller1}{
			rankdir=LR;

			node [shape=circle, style=filled, color=lightblue];

			0 [label="1"];
			1 [label="2"];

			start [shape=point];
			start -> 0;

			0 -> 0 [label="b"];
			0 -> 1 [label="a"];

			1 -> 1 [label="a"];
			1 -> 0 [label="b"];

		}
	\end{automata}

	avec la table $\mathcal T = \set {\set 2}$ est un automate de Muller déterministe qui reconnait les mots avec un nombre
	fini de $b$ (ceci n'est pas possible avec un automate déterministe de Büchi).
\end{exemple}
