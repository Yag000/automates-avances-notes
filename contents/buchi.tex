\section{Langages de mots infinis et automates de Büchi}


\begin{notation}
	Soit $A$ un alphabet.

	On note $\aomega$ comme l'ensemble des mots infinis sur $A$
	(un mot infini est une fonction $: \N \to A$)

	On note $\ainf = A^* \cup \aomega$.

	On étend la définition de certaines notions:

	Soit $L \subseteq A^*$ un langage (de mots finis), on définit
	$\lomega = \setdef {x_0x_1\cdots} {x_i \in L \setminus \set {\motvide}}$
	( généralisation de l'opérateur $*$ à une concaténation d'un nombre infini de mots de $L$.)
\end{notation}


\begin{definition}[Langages $\omega$-rationnels]
	La classe des langages $\omega$-rationnels de $\ainf$ est la \emph{plus petite}
	classe $R$ de sous-ensembles de $\ainf$ qui satisfait
	\begin{enumerate}
		\item $\emptyset \in R \et \forall a \in \Sigma, \set a \in R$
		\item Close par union finie
		\item $\forall X \in R \cap A^*, \forall Y \in R \cap \ainf, X \cdot Y \in R$ \label{omega-clot-fi}
		\item $\forall X \in R \cap A^*, X^* \in R \et X^{\omega} \in R$\label{omega-clot-op}
	\end{enumerate}
\end{definition}

\begin{remarque}
	La classe $R$ contient en particulier tous les langages rationnels de $A^*$.
\end{remarque}

\begin{exemple}
	$A = \set {a,b}$

	$L = \setdef {w \in \aomega} {w \text{ a un nombre fini de } b}$

	On peut le définir par l'expression $\omega$-rationnelle $(a + b)^* a^{\omega}$ ou $(a^*b)^* a^{\omega}$


	$L = \setdef {w \in \aomega} {w \text{ commence par } a \et \text{ a un nombre infini de } b}$

	Avec l'expression : $a(a^*b)^{\omega}$.
\end{exemple}

\begin{prop}[Caractérisation des langages $\omega$-rationnels]
	Un langage $L \subseteq \ainf$ est $w$-rationnel \ssi $L$ est l'union finie de langages
	de la forme $XY^{\omega}$ avec $X,Y \in R \cap A^*$ ($X,Y$ sont des langages rationnels "classiques").
\end{prop}

\begin{proofI}
	\begin{itemize}
		\item \fbox{$\Delta \subseteq R$} \\
		      Soit $\Delta$ la classe des langages obtenue comme union finie de langages de la forme $XY^{\omega}, X,Y \in Rat(A^*)$
		      mais alors par définition de $R$
		      $$Y^{\omega} \in R \reason {par \ref{omega-clot-fi}}$$
		      et
		      $$XY^{\omega} \in R \reason{par \ref{omega-clot-op}}$$
		      et puisque $R$ est fermé relativement à l'union finie, on a que tous les langages de $\Delta$ sont dans $R$,
		      donc $\Delta \subseteq R$.
		\item \fbox{$R \subseteq \Delta$} \\

		      Pour montrer l'inclusion inverse, on définit une classe $\mathcal{E}$.
		      Avec
		      \begin{eqnarray*}
			      L \in \mathcal{E} &\iff& L \cap A^* \in \text{Rat}(A^*) \\
			      &\iff& L \cap \aomega \in \Delta
		      \end{eqnarray*}
		      et on montre que $R \subseteq \mathcal{E} \subseteq \Delta$.
		      Il faut vérifier que
		      \begin{enumerate}
			      \item $\emptyset \in \mathcal{E}$
			      \item $\forall a \in A, \set a \in  \mathcal{E}$
			      \item $\mathcal{E}$ est fermé par union finie

			      \item $\forall X \in \mathcal{E} \cap A^*, \forall Y \in \mathcal{E} \cap \ainf, X \cdot Y \in \mathcal{E}$
			      \item $\forall X \in \mathcal{E} \cap A^*, X^* \in \mathcal{E} \et X^{\omega} \in \mathcal{E}$
		      \end{enumerate}
		      La vérification de ces propriétés est laissé en exercice.

		      Comme $R$ est définie comme la plus petite classe qui satisfait
		      les propriétés precedentes, on en déduit que $R \subseteq \mathcal{E}$
	\end{itemize}
\end{proofI}

\begin{definition}
	Un automate de Büchi est un quadruplet $(Q,I,T,\delta)$, avec $\abs Q < \infty$, où
	$Q,I,T,\delta$ ont la même signification que pour les automates finis.
\end{definition}

\begin{definition}
	Un chemin infini est une suite infinie $(q_i,a_i,q_{i+1})$ tel que $\forall i, (q_i,a_i,q_{i+1}) \in \delta$ et le mot
	infini $a_0a_1\cdots$ est dit l'étiquette du chemin.
\end{definition}


\begin{definition}
	Un chemin est réussi (ou acceptant) si $\exists q_0 \in I, \exists \text{ une suite infinie d'indices}, q_{i_J} \in T$,
	donc un chemin est réussi \ssi il passe un nombre infini de fois par des états terminaux.
\end{definition}

\begin{definition}
	Un mot $w \in \aomega$ est accepté par un automate de Büchi \ssi il est l'étiquette d'un chemin réussi.
\end{definition}


\begin{definition}
	Un langage de $\aomega$ est reconnaissable s'il existe un automate de Büchi qui accepte exactement les mots de $L$.
\end{definition}


\begin{exemple}
	L'automate suivant reconnait les mots qui commencent par $a$ et qui ont un nombre infini de $b$ :

	\digraph[scale=0.50]{buchi1}{
		rankdir=LR;

		node [shape=circle, style=filled, color=lightblue];
		0;
		1;

		start [shape=point];
		start -> 0;

		0 -> 0 [label="a"];
		0 -> 1 [label="a"];
		1 -> 1 [label="b"];
		1 -> 0 [label="b"];

		1 [shape=doublecircle];
	}
\end{exemple}


\begin{exercice}
	écrire un automate de Büchi qui reconnait les mots infinis sur $\set {a,b}$ ayant un nombre fini de $b$. \\
	\begin{minipage}{0.3\textwidth}
		\digraph[scale=0.50]{buchi2}{
			rankdir=LR;

			node [shape=circle, style=filled, color=lightblue];
			0;
			1;

			start [shape=point];
			start -> 0;

			0 -> 0 [label="a,b"];
			0 -> 1 [label="a"];
			1 -> 1 [label="a"];

			1 [shape=doublecircle];
		}
	\end{minipage}
	\begin{minipage}{0.3\textwidth}
		\digraph[scale=0.50]{buchi3}{
			rankdir=LR;

			node [shape=circle, style=filled, color=lightblue];
			0;
			1;

			start [shape=point];
			start -> 0;

			0 -> 0 [label="a,b"];
			0 -> 1 [label="a,b"];
			1 -> 1 [label="a"];

			1 [shape=doublecircle];
		}
	\end{minipage}
	\begin{minipage}{0.3\textwidth}
		\digraph[scale=0.50]{buchi4}{
			rankdir=LR;

			node [shape=circle, style=filled, color=lightblue];
			0;
			1;

			start [shape=point];
			start -> 0;

			0 -> 0 [label="a,b"];
			0 -> 1 [label="b"];
			1 -> 1 [label="a"];

			1 [shape=doublecircle];
		}
	\end{minipage}
\end{exercice}


\begin{definition}
	Un automate est dit émondé si tous ses états sont accesibles (on peut y accéder à partir d'un état initial)
	et co-accessibles (à partiré de n'importe quel état, on peut arriver à un état final).
\end{definition}

Les notions d'automate émondé, complet, déterministe se généralisent aux automates de Büchi.


Mais alors que les algorithmes pour rendre émondé ou complet un automate marchent aussi
pour les automates de Büchi, il n'existe pas d'algorithme pour "déterminiser" un automate de Büchi.

En fait, il existe des langages reconnaissables de $\aomega$ qui ne sont pas reconnus par un automate déterministe.



La classe des langages reconnaissables par un automate de Büchi est fermée relativement à l'union finie
et la méthode pour construire l'automate qui reconnait l'union est la même que pour les automates finis (on "met ensemble"
les deux automates).

On verra que cette classe est aussi fermée relativement au complémentaire et à l'intersection.


\begin{definition}
	Un automate est dit normalisé s'il a

	\begin{itemize}
		\item 1 seul état initial sans flèches rentrantes
		\item 1 seul état final sans flèches sortantes
	\end{itemize}
\end{definition}

\begin{theorem}[lien avec le théorème de Kleene]
	$L \subseteq \aomega$ est $\omega$-reconnaissable \ssi il est reconnu par un automate de Büchi.
\end{theorem}

\begin{proofI}
	\begin{itemize}
		\item \bimpRL \\
		      Soit $L$ reconnaissable, il existe $A = (Q,I,T,\delta)$ tel que $L = \La^{\omega}(A)$.

		      Soient $q,q' \in Q$.Pour chaque couple $(q,q')$, on définit $A_{qq'} = (Q,q,q',\delta)$

		      Soit $\La^*(A_{qq'})$ le langage reconnu par $A_{qq'}$ et $\La^+(A_{qq'}) = \La^*(A_{qq'}) \setminus \set {\motvide}$

		      Alors $L = \bigcup_{(q,q') \in I \times T} \La^*(q,q')(\La^+(q',q'))^{\omega}$, en effet :

		      \begin{itemize}
			      \item \fbox{$\supseteq$} \\
			            En effet, si $w \in \bigcup_{(q,q') \in I \times T} \La^*(q,q')(\La^+(q',q'))^{\omega}$, alors
			            $$\exists i_0 \in I, \exists t_0 \in T, \text{ tels que } w \in \La^*(i_0,t_0)(\La^+(t_0,t_0))^{\omega}$$
			            donc $w$ est l'étiquette d'un chemin qui passe un nombre infini de fois par l'état final $t_0$, et donc
			            $w\in \La^w (a) = L$
			      \item \fbox{$\subseteq$} \\
			            Analogue à la contraposée.
		      \end{itemize}

		\item \bimpLR \\
		      Au vu de la caractérisation et du fait que la classe Rec est fermée relativement a l'union finie, il suffit de \mq que si
		      $X,Y \in \text{Rat}(A^+)$, alors $XY^{\omega} \in \text{Rec}$.

		      D'après le théorème de Kleene, $X,Y \in \text{Rat} (A^*)$, alors $X,Y$ sont reconnaissables par des automates finis, donc

		      \begin{eqnarray*}
			      &\exists A_1 \text{ \tq } X = \lang {A_1} \\
			      &\exists A_2 \text{ \tq } Y = \lang {A_2}
		      \end{eqnarray*}

		      De plus on peut choisir $A_1$ et $A_2$ normalisés.

		      Soit $A_1 = (Q_1, i_1, f_1, \delta_1)$ et $A_2 = (Q_2, i_2, f_2, \delta_2)$.
		      Donc les automates ont la forme suivante :
		      \digraph[scale=0.50]{buchithm1}{
			      rankdir=LR;

			      node [shape=circle, style=filled, color=lightblue];
			      i1;
			      f1;
			      i2;
			      f2;

			      node [shape=box, style=filled, color=lightblue];
			      a1 [label = "A1"]
			      a2 [label = "A2"]

			      start [shape=point];
			      start -> i1;


			      start2 [shape=point];
			      start2 -> i2;

			      i1 -> a1;
			      i1 -> a1;
			      a1 -> f1;
			      a1 -> f1;


			      i2 -> a2;
			      i2 -> a2;
			      a2 -> f2;
			      a2 -> f2;

			      f1 [shape=doublecircle];
			      f2 [shape=doublecircle];
		      }

		      Supposons que $\motvide \notin X$.

		      Alors on identifie les états $f_1,i_2$ et $f_2$ :

		      \digraph[scale=0.50]{buchithm2}{
			      rankdir=LR;

			      node [shape=circle, style=filled, color=lightblue];
			      i1;
			      end;

			      node [shape=box, style=filled, color=lightblue];
			      a1 [label = "A1"]
			      a2 [label = "A2"]

			      start [shape=point];
			      start -> i1;

			      i1 -> a1;
			      i1 -> a1;
			      a1 -> end;
			      a1 -> end;


			      end -> a2;
			      end -> a2;
			      a2 -> end;
			      a2 -> end;

			      end [shape=doublecircle];
		      }

		      L'automate obtenu reconnait bien les mots $XY^{\omega}$ et il se correspond à:

		      $$ A = ( Q_1 \cup Q_2 \setminus \set {i_2,f_2}, \set {i_1}, \set {f_1}, \delta_1 \cup \delta_2 \cup \eta_0 \cup \eta_1 \cup \eta_2)$$
		      où
		      \begin{eqnarray*}
			      \eta_0 &=& \setdef {(f_1,a,f_1)} {(i_2,a,f_2) \in \delta_2} \\
			      \eta_1 &=& \setdef {(f_1,a,q)} {(i_2,a,q) \in \delta_2} \\
			      \eta_2 &=& \setdef {(q,a,f_1)} {(q,a,f_2) \in \delta_2}
		      \end{eqnarray*}


		      Si $\motvide \in L_1$, alors on rend $f_1$ initial.
	\end{itemize}
\end{proofI}

\begin{exemple}
	$L_1 = \motvide, L_2 = \lang {a^*b}$

	L'automate classique pour $L_2$ :

	\digraph[scale=0.50]{buchi5}{
		rankdir=LR;

		node [shape=circle, style=filled, color=lightblue];
		i [label="i"];
		f [label="f"];

		start [shape=point];
		start -> i;

		i -> i [label="a"];
		i -> f [label="b"];

		f [shape=doublecircle];
	}

	on normalise l'automate

	\digraph[scale=0.50]{buchi6}{
		rankdir=LR;

		node [shape=circle, style=filled, color=lightblue];
		q [label="qnew"];
		i [label="i"];
		f [label="f"];

		start [shape=point];
		start -> q;

		q -> i [label="a"];
		q -> f [label="b"];
		i -> i [label="a"];
		i -> f [label="b"];

		f [shape=doublecircle];
	}

	On applique utilise la construction de la proposition \ref{prop:l1l2o} et on obtient :

	\digraph[scale=0.50]{buchi7}{
		rankdir=LR;

		node [shape=circle, style=filled, color=lightblue];
		q [label="qnew"];
		i [label="i"];
		f [label="f"];

		start [shape=point];
		start -> q;

		q -> q [label="b"];
		q -> i [label="a"];
		q -> f [label="b"];

		i -> i [label="a"];
		i -> f [label="b"];
		i -> q [label="b"];

		q [shape=doublecircle];
	}

	Et parfois certains états sont inutiles :

	\digraph[scale=0.50]{buchi8}{
		rankdir=LR;

		node [shape=circle, style=filled, color=lightblue];
		q [label="qnew"];
		i [label="i"];

		start [shape=point];
		start -> q;

		q -> q [label="b"];
		q -> i [label="a"];

		i -> i [label="a"];
		i -> q [label="b"];

		q [shape=doublecircle];
	}
\end{exemple}


\begin{definition}
	Soit $L \subseteq \mots$ un langage de mots finis. On note
	$$\larr L = \setdef {w \in \alphabet^ {\omega}} {w \text{ a une infinité de préfixes dans } L}$$
\end{definition}


\begin{exemple}
	\begin{enumerate}
		\item $L = a^*b, \larr L = \emptyset$
		\item $L = (ab)^*, \larr L = (ab)^{\omega}$
		\item $L = (a^*b)^+, \larr L =  (a^*b)^{\omega}$
	\end{enumerate}
\end{exemple}


\begin{prop}[Admis]
	Soit $A = (Q,I,T,\delta)$ un automate de Büchi déterministe alors $\La^{\omega} (A) = \larr {\La^+ (A)}$
	où $\La^+ (A)$ est le langage des mots finis reconnus par $A$ comme un automate fini "traditionnel".
\end{prop}


\begin{coro}
	$X \subseteq \mots$ est reconnu par un automate de Büchi déterministe \ssi $\exists L \in \text{Rat}(\mots)$
	\tq $X = \larr L$.
\end{coro}

\begin{prop}
	Le langage $X = (a+b)^*a^{\omega}$ ($ {\abs w}_b < \infty $) ne peut pas être reconnu par un automate de Büchi déterministe.
\end{prop}

\begin{proof}
	Par l'absurde, s'il y a un automate déterministe $\exists  L \in \text{Rat}(\alphabet)$ \tq tout mot de $X$ a un nombre infini de préfixes dans $L$.

	\begin{itemize}
		\item Prenons le mot $ba^{\omega}$, il a un préfixe $ba^{i_1} \in L$
		\item Prenons le mot $ba^{i_1}ba^{\omega}$, il a un préfixe $ba^{i_1}ba^{i_2} \in L$
		\item Prenons le mot $ba^{i_1}ba^{i_2}a^{\omega}$, il a un préfixe $ba^{i_1}ba^{i_2}ba^{i_3} \in L$
	\end{itemize}

	Alors on peut construire un mot infini :
	$$ ba^{i_1}ba^{i_2}\ldots ba^{i_k}\ldots$$
	qui est l'étiquette d'un chemin qui passe un nombre infini de fois par un état final.
	Donc ce mot serait reconnu par l'automate, or il contient un nombre infini de $b$.  $\contradict$
\end{proof}


\begin{prop}
	La famille des langages reconnaissables par un automate de Büchi est fermée relativement à l'intersection.
\end{prop}

\begin{proof}
	Soit $L_1$ et $L_2$ deux langages reconnaisable ey soeint
	\begin{eqnarray*}
		A_1 &=& (Q_1,I_1,T_1.\delta_1) \\
		A_2 &=& (Q_2,I_2,T_2.\delta_2)
	\end{eqnarray*}
	tels que $L_1 = \La ^ {\omega} (A_1) \et  L_2 = \La ^ {\omega} (A_2)$.

	Alors l'automate $A = (Q,I,T,\delta)$ :
	\begin{eqnarray*}
		Q &=& Q_1 \times Q_2 \times \set{1,2}  \\
		I &=& \setdef {i_1,i_2,1} {i_1 \in I_1,i_2 \in I_2 } \\
		T &=&  \setdef {t,q,1} {t \in T_1, q \in Q} \\
		\delta ((q_1,q_2,x), a) &=&
		\left\{
		\begin{array}{cc}
			(\delta_1(q_1,a), \delta_2 (q_2,a), 2) & \text {si } q_1 \in T_1 \et x = 1 \\
			(\delta_1(q_1,a), \delta_2 (q_2,a), 1) & \text {si } q_2 \in T_2 \et x = 2 \\
			(\delta_1(q_1,a), \delta_2 (q_2,a), x) & \text {sinon}
		\end{array}
		\right.
	\end{eqnarray*}
\end{proof}

\begin{exemple}
	TODO
\end{exemple}

\begin{notation}
	Soit $w \in \mots$, not note $\inf (w) = \setdef {x \in \alphabet} {{\abs w}_x = \infty}$
\end{notation}



\begin{theorem}{Factorisation Ramsayenne}
	Soit $f : \N \times \N \to C $, où $C$ est un ensemble fini, alors il existe une suite infinie d'entiers $E = i_0, i_1, \ldots, i_n, \ldots$
	telle que $f(E \times E)$ soit un singleton, \ie $\forall i_{j_1},i_{e_1},i_{j_2},i_{e_2}$ dans $E$, $f(i_{j_1},i_{e_1})=f(i_{j_2},i_{e_2})$.
\end{theorem}

\begin{proof}
	On construit une suite de sous-ensembles de $\N$, $T_0, T_1, \ldots$, de la maniere suivante :
	$T_0 = \N$. Si on a déjà $T_i$, on construit $T_{i+1}$.

	Comme les couleurs sont au nombre fini, il existe une couleur $c_0$ et une suite
	$u_0, v_1, v_2, \ldots$ tel que $f(u_0, v_i) = c_0 \forall i \geq 1$ (il existe un nombre infini d'arcs sortant de $u_0$ de couleur $c_0$)
	alors $T_{i+1} = \set {v_1, v_2, \ldots}$


	Soit $u_i = \min (T_1)$. et considérons $u_0, u_1, \ldots$. Par construction $\forall i \in \N, \forall j \in \N, f(u_i, u_{i + j}) = c_i$
	puisque les couleurs sont en nombre fini, il doit exister une suite $i_0, i_1, \ldots$ telle que $f(u_{i_J}, u_{i_{J+1}}) = c_{i_0}$
	et donc la suite $u_{i_0},u_{i_1}, \ldots$ est la suite cherchée.
\end{proof}


Soit $X$ un langage reconnaisable et $A = (Q,I,T,\delta)$ un automate de Büchi qui reconnait $X$.

%TODO: Make pretty

On défini une relation sur $\mots$:
\begin{eqnarray*}
	w \sim w' \iff \forall p,q \in Q, & \exists \text{un chemin de $p$ à $q$ étiquetté par } w \\
	&\et \exists \text{ un chemin de $p$ à $q$ d'étiquette $w$ qui passe par un état terminal}
\end{eqnarray*}


\begin{itemize}
	\item $\sim$ est d'équivalence : trivial
	\item $\sim$ est une relation d'équivalence d'index fini, son nombre de classe ne peut pas dépasser $3^{{\abs Q}^2}$
	      car chacune des ${\abs Q}^2$ coupes d'états distingue 33types de mots :
	      \begin{itemize}
		      \item Les mots qui ne sont pas l'étiquette d'un chemin de $p$ à $q$.
		      \item Les mots qui sont l'étiquette d'un chemin $s$ de $p$ à $q$ dont aucun ne passe par un état terminal.
		      \item Les mots qui sont l'étiquette d'un chemin $s$ de $p$ à $q$ dont au moins 1 par par un état terminal.
	      \end{itemize}
	\item $\sim$ est une congruence.

	      Si $w \sim w'$ et $u,v \in \mots$ alors $uwv \sim uw'v$

	      TODO
\end{itemize}

Alors $\mots / \sim$ est un monoïde fini.

Et donc $\forall w \in \mots, [w]$ est un langage rationnel, en effet, $[x]$ est l'image inverse d'un singleton de $\mots / \sim$
par le morphisme de projection $\phi : \mots \to \mots / \sim$ tel que $f (x) = [w] \forall x \in \mots$.


\begin{prop}
	La classe des langages reconnaissables pas un automate de Büchi est fermée relativement au passage au complémentaire.
\end{prop}
