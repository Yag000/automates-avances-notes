\section{Langages de mots infinis et automates de Büchi}


\begin{notation}
	Soit $A$ un alphabet.

	On note $\aomega$ comme l'ensemble des mots infinis sur $A$
	(un mot infini est une fonction $: \N \to A$)

	On note $\ainf = A^* \cup \aomega$.

	On étend la définition de certaines notions:

	Soit $L \subseteq A^*$ un langage (de mots finis), on définit
	$\lomega = \setdef {x_0x_1\cdots} {x_i \in L \setminus \set {\motvide}}$
	( généralisation de l'opérateur $*$ à une concaténation d'un nombre infini de mots de $L$.)
\end{notation}


\begin{definition}[Langages $\omega$-rationnels]
	La classe des langages $\omega$-rationnels de $\ainf$ est la \emph{plus petite}
	classe $R$ de sous-ensembles de $\ainf$ qui satisfait
	\begin{enumerate}
		\item $\emptyset \in R \et \forall a \in \Sigma, \set a \in R$
		\item Close par union finie
		\item $\forall X \in R \cap A^*, \forall Y \in R \cap \ainf, X \cdot Y \in R$ \label{omega-clot-fi}
		\item $\forall X \in R \cap A^*, X^* \in R \et X^{\omega} \in R$\label{omega-clot-op}
	\end{enumerate}
\end{definition}

\begin{remarque}
	La classe $R$ contient en particulier tous les langages rationnels de $A^*$.
\end{remarque}

\begin{exemple}
	$A = \set {a,b}$

	$L = \setdef {w \in \aomega} {w \text{ a un nombre fini de } b}$

	On peut le définir par l'expression $\omega$-rationnelle $(a + b)^* a^{\omega}$ ou $(a^*b)^* a^{\omega}$ 


	$L = \setdef {w \in \aomega} {w \text{ commence par } a \et \text{ a un nombre infini de } b}$

	Avec l'expression : $a(a^*b)^{\omega}$.
\end{exemple}

\begin{prop}[Caractérisation des langages $\omega$-rationnels]
	Un langage $L \subseteq \ainf$ est $w$-rationnel \ssi $L$ est l'union finie de langages
	de la forme $XY^{\omega}$ avec $X,Y \in R \cap A^*$ ($X,Y$ sont des langages rationnels "classiques").
\end{prop}

\begin{proofI}
	%TODO: Make pretty

	\begin{itemize}
		\item \fbox{$\Delta \subseteq R$} \\
		      Soit $\Delta$ la classe des langages obtenue comme union finie de langages de la forme $XY^{\omega}, X,Y \in Rat(A^*)$
		      mais alors par définition de $R$
		      $$Y^{\omega} \in R \reason {par \ref{omega-clot-fi}}$$
		      et
		      $$XY^{\omega} \in R \reason{par \ref{omega-clot-op}}$$
		      et puisque $R$ est fermé relativement à l'union finie, on a que tous les langages de $\Delta$ sont dans $R$, 
			  donc $\Delta \subseteq R$.
		\item \fbox{$R \subseteq \Delta$} \\

		      Pour montrer l'inclusion inverse, on définit une classe $\mathcal{E}$.
		      TODO
	\end{itemize}
\end{proofI}

\begin{definition}
	Un automate de Büchi est un quadruplet $(Q,I,T,\delta)$, avec $\abs Q < \infty$, où
	$Q,I,T,\delta$ ont la même signification que pour les automates finis.
\end{definition}

\begin{definition}
	Un chemin infini est une suite infinie $(q_i,a_i,q_{i+1})$ tel que $\forall i, (q_i,a_i,q_{i+1}) \in \delta$ et le mot
	infini $a_0a_1\cdots$ est dit l'étiquette du chemin.
\end{definition}


\begin{definition}
	Un chemin est réussi (ou acceptant) si $\exists q_0 \in I, \exists \text{ une suite infinie d'indices}, q_{i_J} \in T$,
	donc un chemin est réussi \ssi il passe un nombre infini de fois par des états terminaux.
\end{definition}

\begin{definition}
	Un mot $w \in \aomega$ est accepté par un automate de Büchi \ssi il est l'étiquette d'un chemin réussi.
\end{definition}


\begin{definition}
	Un langage de $\aomega$ est reconnaissable s'il existe un automate de Büchi qui accepte exactement les mots de $L$.
\end{definition}


\begin{exemple}
	TODO
\end{exemple}


\begin{exercice}
	TODO
\end{exercice}


\begin{definition}
	Un automate est dit émondé si tous ses états sont accesibles (on peut y accéder à partir d'un état initial)
	et co-accessibles (à partiré de n'importe quel état, on peut arriver à un état final).
\end{definition}

Les notions d'automate émondé, complet, déterministe se généralisent aux automates de Büchi.


Mais alors que les algorithmes pour rendre émondé ou complet un automate marchent aussi
pour les automates de Büchi, il n'existe pas d'algorithme pour "déterminiser" un automate de Büchi.

En fait, il existe des langages reconnaissables de $\aomega$ qui ne sont pas reconnus par un automate déterministe.



La classe des langages reconnaissables par un automate de Büchi est fermée relativement à l'union finie
et la méthode pour construire l'automate qui reconnait l'union est la même que pour les automates finis (on "met ensemble"
les deux automates).

On verra que cette classe est aussi fermée relativement au complémentaire et à l'intersection.


\begin{theorem}[lien avec le théorème de Kleene]
	$L \subseteq \aomega$ est $\omega$-reconnaissable \ssi il est reconnu par un automate de Büchi.
\end{theorem}

\begin{proofI}
	\begin{itemize}
		\item \bimpRL

		      Soit $L$ reconnaissable, il existe $A = (Q,I,T,\delta)$ tel que $L = \La^{\omega}(A)$.

		      Soient $q,q' \in Q$.Pour chaque couple $(q,q')$, on définit $A_{qq'} = (Q,q,q',\delta)$

		      Soit $\La^*(A_{qq'})$ le langage reconnu par $A_{qq'}$ et $\La^+(A_{qq'}) = \La^*(A_{qq'}) \setminus \set {\motvide}$


		      Alors $L = \bigcup_{(q,q') \in I \times T} \La^*(q,q')(\La^+(q',q'))^{\omega}$, en effet :

		      \begin{itemize}
			      \item \fbox{$\supseteq$} \\
			            TODO

			      \item \fbox{$\subseteq$} \\
			            Analogue à la contraposée.
		      \end{itemize}


		\item \bimpLR
            TODO
	\end{itemize}
\end{proofI}
