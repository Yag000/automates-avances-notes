\section{Automates cellulaires}

\begin{definition}
	Un automate cellulaire est défini par :
	\begin{itemize}
		\item L'entier positif $d$ représentant la dimension de l'espace $(\Z^d)$
		\item Un ensemble fini $S$ d'états
		\item Un ""voisinage"", \ie un ensemble de vecteurs de $\Z^d (\vec{v_1},  \vec{v_1}, \ldots ,\vec{v_m})$
		      qui représente la position relative des voisins (relative à la position de la cellule). Les voisins de la
		      cellule $\vec r$ sont $\vec r + \vec {v_1},  \ldots , \vec r + \vec {v_m}$.
		\item Une fonction $f : S ^ m \to S$ de transition locale qui définit l'état d'une cellule à l'instant $t + 1$
		      en fonction des états de ses voisins à l'instant $t$.
	\end{itemize}
\end{definition}

Les automates cellulaires sont synchrones et homogènes dans le temps et dans l'espace.

\begin{definition}
	Une configuration est une fonction $\Z ^ d \to S$ qui décrit l'état courant de chaque cellule.

	Si $C$ est l'ensemble des configurations, alors un automate cellulaire définit une fonction $G: C \to C$ où si $c \in C$, alors
	$G(x) = c'$ est la configuration obtenue de $c$ en appliquant la règle $f$ à toutes les cellules.

	En général, on itère la fonction $G$ et pour une configuration $c$ on s'intéresse à la suite :
	$$ G^0 (c) = c, G^1 (c), G^2 (c), \text{ où } G^i (c) = G ( G^{i - 1 }(c))$$
\end{definition}

\begin{definition}
	Un point fixe (ou nature morte) est une configuration $c$ \tlq $\forall k, G^k(c) = c$.
\end{definition}


\begin{definition}
	Une configuration $c$ est dite périodique (dans le temps), ou oscillateur, si $\exists k$ \tq $G^k(c) = c$. Le plus petit
	petit $k$ \tq $G^k(c) = c$ est dite la période d'oscillation.
\end{definition}

\begin{definition}
	Une configuration est dite ultimement fixe si $\exists k$ \tq $\forall j, \ G^{k+j}(c) =  G^{k}(c)$.
\end{definition}


\begin{definition}
	Une configuration est dite ultimement périodique si $\exists k, \exists l>0$ \tq $ G^{k+l}(c) =  G^{k}(c)$
\end{definition}


\begin{exemple}
	TODO
\end{exemple}


\subsection{Unicité des automates cellulaires}


\begin{theorem}
	Si deux automates cellulaires $A \et B$ réalisent la même fonction $G$, alors $A \et B$ ne peuvent différer que par
	leur "voisinages" et les deux "voisinages" ne peuvent être différents que par la présence (ou pas) de voisins inutiles (\cad
	des voisins dont l'état n'a aucune influence sur la valeur de $f$).
\end{theorem}


\begin{exemple}
	TODO
\end{exemple}

\subsection{Le jeu de la vie}

\begin{exemple}
	Pour $d = 2$, Game of Life de Conway \cite{conwayGOL} est défini par :
	\begin{itemize}
		\item $S = \set {0,1}$
		\item $V = ((0,1), (1,0), (0,-1), (-1,0), (1,1), (1,-1), (-1,1), (-1,-1))$
		\item Fonction de transition :
		      \begin{itemize}
			      \item Une cellule vivante reste vivante \ssi elle a 2 ou 3 voisins vivant.
			      \item Une cellule morte devient vivante \ssi elle a 3 voisins vivants.
		      \end{itemize}
	\end{itemize}


	Une implémentation sur navigateur est disponible sur: \url{https://conwaylife.com/}.
\end{exemple}

\begin{definition}

	On apelle automates cellulaires élémentaires les automates cellulaires qui respectent :
	\begin{itemize}
		\item $d = 1$
		\item $S = \set{0,1}$
		\item $d = (-1,0,1)$
	\end{itemize}

	Il y a exactement $2^8 = 256$ automates élémentaires.
\end{definition}

\begin{notation}
	Wolfgram a introduit une numérotation \cite{wolfgramClass} pour les automates cellulaires élémentaires où chaque automate
	est associé à un entier entre 0 et 255.

	Comme on peut définir l'automate juste en étudiant son comportement sur trois cellules voisines, il suffit d'associer une
	étiquette à ce comportement pour caractériser tout l'automate cellulaire. Ainsi, on commence par trier
	les 8 configurations possibles pour 3 cellules voisines ainsi :

	% From : https://tex.stackexchange.com/questions/320195/is-there-an-implemented-way-to-draw-wolframs-elementary-cellular-automata
	\[ \LARGE
		\substack{\blacksquare \blacksquare \blacksquare } \enspace
		\substack{\blacksquare \blacksquare \square } \enspace
		\substack{\blacksquare \square \blacksquare } \enspace
		\substack{\blacksquare \square \square } \enspace
		\substack{\square \blacksquare \blacksquare } \enspace
		\substack{\square \blacksquare \square } \enspace
		\substack{\square \square \blacksquare } \enspace
		\substack{\square \square \square }
	\]

	Si on considère ces trois cellules comme un nombre binaire encodé sur 3 bits, ce tri correspond à un tri décroissant sur la
	valeur de ces configurations. On peut donc regarder quel sera l'état de la cellule du milieu, par exemple pour la
	fonction \textbf{xor} :


	\[ \LARGE
		\substack{\blacksquare \blacksquare \blacksquare \\ \square } \enspace
		\substack{\blacksquare \blacksquare \square \\ \blacksquare} \enspace
		\substack{\blacksquare \square \blacksquare \\ \square} \enspace
		\substack{\blacksquare \square \square \\ \blacksquare} \enspace
		\substack{\square \blacksquare \blacksquare \\ \blacksquare} \enspace
		\substack{\square \blacksquare \square \\ \square} \enspace
		\substack{\square \square \blacksquare \\ \blacksquare} \enspace
		\substack{\square \square \square \\ \square}
	\]

	Maintenant, on peut voir ceci comme un nombre binaire sur 8 bits ou le i-ème bit correspond à la
	nouvelle valeur de la cellule à la i-ème position. Ainsi, pour cet automate cellulaire, on obtient :


	\[ \LARGE
		\substack{\blacksquare \blacksquare \blacksquare \\ \square \\ 0 } \enspace
		\substack{\blacksquare \blacksquare \square \\ \blacksquare \\ 1} \enspace
		\substack{\blacksquare \square \blacksquare \\ \square \\ 0} \enspace
		\substack{\blacksquare \square \square \\ \blacksquare \\ 1} \enspace
		\substack{\square \blacksquare \blacksquare \\ \blacksquare \\ 1} \enspace
		\substack{\square \blacksquare \square \\ \square \\ 0} \enspace
		\substack{\square \square \blacksquare \\ \blacksquare \\ 1} \enspace
		\substack{\square \square \square \\ \square \\ 0}
	\]

	Ainsi cet automate est numéroté par $01011010_2 = 90_{10}$.
\end{notation}

\subsection{Configurations finies et configuérations périodiques}

\begin{definition}
	Soit $s \in S$ un état, on appelle $s$-support l'ensemble des cellules qui ont
	un état différent de $s$.
\end{definition}

\begin{notation}
	Souvent, on impose qu'un état $q_0$ soit un état ""quiescent"" (généralement, c'est l'état 0).

	Il est stable par $f$, \ie $f(\underbrace{q_0,\ldots, q_0}_m) = q_0$.
\end{notation}

\begin{definition}
	On appelle configurations finies ($q_0$-finies) les configurations ayant un $q_0$-support fini.
\end{definition}

\begin{definition}
	Soit $\vec v$ un vecteur de $\Z^d$, la translation $\tau_{\vec v}$ est une fonction d'automate cellulaire où le "voisinage" contient seulement des $\vec v$,
	autrement dit, la valeur d'une cellule $\vec r$ à l'instant $t+1$ est simplement la valeur de la cellule $\vec r + \vec v$ à l'instant $t$.


	$\vec v = (0,\ldots,0, 1, 0, \ldots, 0) = \sigma_i$ est nommé shift élémentaire, où le 1 est à la position $i$.

	Toute transition peut être exprimée en fonction de shifts élémentaires.
\end{definition}

\begin{definition}
	Si $\vec v$ est un vecteur, on dit qu'une configuration $c$ est $\vec v$-périodique si $c(\vec n) = c (\vec n + \vec v)$.
\end{definition}

\begin{exemple}
	TODO
\end{exemple}

\begin{definition}
	Une configuration est dite spatialement périodique si $\exists \vec v$ \tq $c$ est $\vec v$-périodique.
\end{definition}

\begin{definition}
	Une configuration est dite totalement périodique si $\exists k$ \tq $\forall i \in \enum 1 d$, $c$ est $\sigma_i^k$-périodique.
\end{definition}

\begin{prop}
	Si $G$ est la fonction d'un automate cellulaire et $\tau$ une translation, alors $G \circ \tau = \tau \circ G$.
\end{prop}

\begin{definition}
	Une configuration est dite totalement périodique si $\forall i \in \enum 1 d\exists k_i, $ \tq $c$ est $k_i\sigma_i$-périodique.
\end{definition}

\begin{remarque}
	Puisque les fonctions d'un automate cellulaire commutent avec les translations, il en découle que si $c$ est totalement périodique, alors $G(c)$
	est aussi totalement périodique.

	Donc si $P$ représente l'ensemble de toutes les configurations totalement périodiques, on considère $\restr G P : P \to P$, notée $G_P$.
	Les deux fonctions $G_F$ et $G_P$ donnent des renseignements sur la surjectivité et l'injectivité de $G$.
\end{remarque}

\subsection{Une topologie sur l'espace des configurations $S^{\Z^d}$}


\begin{definition}
	Soit $c_1,c_2, \ldots, c_n, \ldots$ une suite de configurations, on dit que la suite converge vers
	la configuration $c$ (ou qu'elle a comme limite la configuration $c$) si
	$$ \forall \vec n \in \Z^d, \exists i, c_j(\vec n) = c(\vec n) \forall j > i$$
\end{definition}

\begin{prop}(Compacité de $S^{\Z^d}$)
	Toute suite de configurations possède une sous-suite convergente.
\end{prop}

\begin{proof}
	La démonstration est une généralisation de la factorisation de Ramsay à un nombre $d$ quelconque de dimension.
	(Pour une configuration $c : \Z^d \to S$ avec $S$ fini, Ramsay le fait pour les fonctions $\N ^ 2 \to M$, avec $M$ fini).
\end{proof}

\begin{prop}\label{prop:continues}
	Les fonctions d'automates cellulaires sont continues.

	Soit $c_1,c_2, \ldots, c_n, \ldots$ une suite de configurations qui converge vers $c$,
	alors $G(c_1),G(c_2), \ldots, G(c_n), \ldots$ est une suite de configurations qui converge vers $G(c)$.
\end{prop}

\begin{proof}
	La démonstration n'est pas compliquée et elle est laissée en exercice.
\end{proof}

\begin{prop}\label{prop:dense}
	Les configurations finies et totalement périodiques sont denses.

	$$\forall c \in S^{\Z^d}, \exists c_1,c_2, \ldots,\forall c_i, c_i \in F, \exists p_1,p_2, \ldots \forall c_j, c_j \in P$$
	$$ \lim_{i \to \infty} c_i = c \et \lim_{j \to \infty} p_j = c $$
\end{prop}

\begin{proof}
	Soit $c$ une configuration. Soit $\vec {r_0}, \vec {r_1},\ldots$ un dénombrement de cellules de $\Z^d$ et pour $i \in \N$,
	$c_i$ la configuration définie par :

	$$ c_i(r_j) = \left\{ \begin{array}{cc}
			c(r_j) & \text{ si } j \leq i \\
			q      & \text{ sinon }
		\end{array}
		\right.$$
	où $q$ est un état "quiescent".

	Il est évident que $\forall i, c_i \in F$ et que $\lim_{i \to \infty} c_i = c$.

	Pour obtenir une suite de configurations périodiques qui convergent vers $c$, on peut obtenir une suite de configurations
	$p_1, p_2, \ldots $ où $p_i$ coïncide avec $c$ dans l'hypercube centré à l'origine et de coté $2i + 1$ et a une
	période $2i + 1$ dans toutes les directions.

	Les configurations $p_i$ sont évidement périodiques et $\lim_{i \to \infty} p_i = c$.
\end{proof}

\subsection{Injectivité et surjectivité des automates cellulaires}

\begin{definition}
	Un automate cellulaire est dit
	$\begin{array}{c}
			\text{injectif}  \\
			\text{surjectif} \\
			\text{bijectif}
		\end{array}$
	si sa fonction $G$ est
	$\begin{array}{c}
			\text{injective}  \\
			\text{surjective} \\
			\text{bijective}
		\end{array}$
	.
\end{definition}


\begin{theorem}
	\begin{enumerate}
		\item $G$ injective $\implies G_F, G_P$ injectives \label{thm:big1}
		\item Si $G_F$ est surjective ou $G_P$ est surjective $\implies G$ surjective \label{thm:big2}
		\item Si $G_P$ est injective $\implies G_P$ surjective \label{thm:big3}
	\end{enumerate}
\end{theorem}

\begin{proof}
	\begin{enumerate}
		\item Trivial, la restriction d'une fonction injective est toujours injective.
		\item Dans le cas où $G_F$ est surjective :

		      Soit $c$ une configuration quelconque, on sait par la Proposition \ref{prop:dense} qu'il existe
		      une suite de configurations finies $c_1, c_2, \ldots , c_m$ \tlq $\lim_{i \to \infty} c_i = c$.
		      Puisque chaque $c_i$ est une configuration finie et comme $G_F$ est surjective,
		      $$\forall i \exists e_i \in F, G_F(e_i) = c_i = G(e_i)$$
		      Soit la suite $e_1, e_2, \ldots e_i$  doit posséder une sous-suite convergente.
		      $$ e_{i_1}, e_{i_2}, \ldots , e_{i_j} \ldots \to e $$
		      $G$ est continue par la proposition \ref{prop:continues}, donc
		      $G(e_{i_1}), G(e_{i_2}), \ldots , G(e_{i_j}) \ldots  \to G(e)$.

		      Mais par ailleurs,
		      \begin{eqnarray*}
			      \lim_{j \to \infty} (G(e_{i_j})) &=& \lim_{j \to \infty} (c_{i_j}) \\
			      &=& \lim_{i \to \infty} (c_i) \\
			      &=& c
		      \end{eqnarray*}

		      On a donc $G(e) = c$ et $c$ possède donc une anti-image.
		\item Soit $c$ une configuration périodique.
		      $\exists k_1, k_2, \ldots, k_d$ \tlq
		      \begin{equation}
			      \forall i \in \enum 1 d, \ c \text{ est }\sigma_i^{k_i}\text{-périodique} \et
			      c(\vec n) = c(\vec n + \sigma_i^{k_i})
			      \label{eq:tmhc}
		      \end{equation}

		      Soit $K$ l'ensemble de toutes les configurations qui satisfont (\ref{eq:tmhc}). On a que $K \subseteq P$ et $K$ est fini,
		      car chaque configuration de $K$ est uniquement identifiée par les cellules dans un hypercube de cotés $k_1, k_2, \ldots, k_d$
		      ($\abs K = {\abs S}^{k_1k_2\ldots k_d}$)

		      Puisque $G$ commute avec les translations, on a que $G(K) \subseteq K$.

		      Considérons $\restr G K$. Cette fonction est injective car c'est une restriction de $G$ à un
		      sous-ensemble de $P$ et $G_P$ est injective par hypothèse.
		      Mais toute fonction injective sur un ensemble fini est aussi surjective,
		      \cad chaque élément de $K$ (dont $c$) a un antécédent dans $K$,
		      donc $\exists e \in K \subsetneq P, G(e) = c$, et donc $G_P$ est surjective.
	\end{enumerate}
\end{proof}

\begin{coro} \label{coro:inj-surj}
	Si un automate cellulaire est injectif, alors il est surjectif (et donc bijectif).
\end{coro}

\begin{proof}
	\begin{eqnarray*}
		G \text{ injective } &\implies& G_P \text{ injective} \reason{\ref{thm:big1} } \\
		&\implies& G_P \text{ surjective} \reason{\ref{thm:big3} } \\
		&\implies& G \text{ surjective} \reason{\ref{thm:big2} }
	\end{eqnarray*}
\end{proof}


\begin{definition}
	On dit qu'un automate cellulaire est réversible s'il est injectif et si $G^{-1}$ est une fonction d'automate cellulaire.
\end{definition}

\begin{remarque}
	Par définition, si un automate cellulaire est réversible alors il est bijectif.
\end{remarque}

\begin{prop}\label{prop:bij-rev}
	Si $A$ est un automate cellulaire bijectif, alors il est réversible.
\end{prop}

\begin{proof}
	La preuve montre que la fonction $G^{-1}$ est aussi une fonction d'automate cellulaire
	(définit la règle locale d'évolution en fonction de l'état des cellules appartenant à un voisinage \emph{borné}).
\end{proof}

\begin{coro}
	$G$ injective $\implies \ G_F$ surjective.
\end{coro}


\begin{proof}
	\begin{eqnarray*}
		G \text{ injective} &\implies& G \text{ bijective} \reason{\ref{coro:inj-surj} }\\
		&\implies& G \text{ réversible} \reason{\ref{prop:bij-rev} }
	\end{eqnarray*}
	Si $q$ est l'état "quiescent" pour $G$, alors $q$ est aussi l'état "quiescent" de $G^{-1}$.
	(Si toutes les cellules du voisinage sont à $q$, alors $G^{-1}$ met la cellule à l'état $q$.)

	Donc si $c$ est une configuration finie,  $e = G^{-1}(c)$ (la configuration obtenue en appliquant
	l'automate inverse) est aussi finie, et alors $G(e) = c$ et $c$ a bien une anti-image finie.
\end{proof}

TODO add graph

\subsubsection{Surjectivité et balance}

\begin{exemple}
	$W_{110}$ n'est pas surjectif. TODO
\end{exemple}

\begin{definition}
	Une configuration qui n'a pas d'anti-image est dite un ""Jardin d'Éden"" ("GOE", "JDE").
\end{definition}

\begin{exercice}
	Montrer que toute configuration contenant le motif $01010$ est un Jardin d'Éden pour $W_{110}$.
\end{exercice}

\begin{remarque}
	Si un automate cellulaire n'est pas surjectif, alors il existe des configurations qui n'ont pas d'anti-image, dites "Jardins d'Éden".
\end{remarque}

\begin{definition}
	Une configuration finie qui n'a pas d'anti-image est dite un \emph{""orphelin""}.
\end{definition}

\begin{prop}
	Il existe un "JDE" \ssi il existe un "orphelin".
\end{prop}


\begin{proofI}
	\begin{itemize}
		\item \bimpRL\\
		      S'il existe un "orphelin" $D$, alors toute configuration qui inclut l'"orphelin" est un "JDE".
		\item \bimpLR\\
		      Si un "JDE" existe, alors la fonction $G$ de l'automate n'est pas surjective, et donc $\restr G F$ n'est pas surjective,
		      \cad, $\exists$ une configuration finie qui n'a pas d'anti-image, \ie un "orphelin".
	\end{itemize}
\end{proofI}

\begin{definition}
	Deux motifs finis $J_1$ et $J_2$ sont dits jumeaux si, en remplaçant $J_1$ par $J_2$ dans
	n'importe quelle configuration qui contient $J_1$, on génère la même configuration.
\end{definition}

\begin{theorem}[du Jardin d'Éden \cite{mooreGOE, myhillGOE}]
	$G_F \text{ inj} \iff G \text{ surj}$, autrement dit, un automate cellulaire possède un "JDE" \ssi il possède deux jumeaux.
\end{theorem}



\begin{proofI}
	\begin{itemize}
		\item \bimpLR\\
		      Supposons que $G$ ne soit pas surjective, alors il existe un "JDE" et donc un "orphelin".
		      A partir de l'existence de cet "orphelin", on doit montrer l'existence d'un couple de jumeaux.
		      Prenons comme exemple $W_{110}$. On a montré que \emph{01010} est un "orphelin". Soit $k$ un entier
		      et considérons $C_k$ l'ensemble de toutes les configurations finies qui ont un support de taille
		      $5k-2$, alors $\abs {C_k} = 2^{5k-2} - \frac {32^k} 4$

		      Si $c$ est une configuration de $C_k$, alors son image $G(c)$ est une configuration avec un support de taille $5k$.
		      Si on découpe ce support de taille $5k$ en $k$ segments de taille $5$, il y a au plus $31$ choix pour chaque bloc car
		      \emph{01010} est interdit. Alors l'ensemble $G(C_k)$ a une taille d'au plus $31^k$ pour un $k$ suffisamment grand.
		      Or $\frac {32^k} 4 > 31 ^ k$ pour un $k$ suffisamment grand. On a donc qu'il y aura plus de configurations
		      dans $C_k$ que dans son image, donc il existe des configurations de $G_k$ qui ont la même image (deux jumeaux).

		      Cette preuve est incomplète, il faut la généraliser aux dimensions $d > 1$.


		\item \bimpRL\\
		      Soient $J_1$ et $J_2$ deux jumeaux, montrons qu'il existe un orphelin.

		      Soit $n$ un entier tel que
		      \begin{enumerate}
			      \item $J_1$ et $J_2$ rentrent das un carré de taille $n$.
			      \item Le voisinage de chaque cellule ne contient que des cellules à distance inférieure à $n$.
		      \end{enumerate}
		      TODO
	\end{itemize}
\end{proofI}

