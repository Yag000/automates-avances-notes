\section{Automates cellulaires}

\begin{definition}
	Un automate cellulaire est définit par :
	\begin{itemize}
		\item L'entier positif $d$ représentant la dimension de l'espace $(\Z^d)$
		\item Un ensemble fini $S$ d'états
		\item Un "voisinage", \ie, un ensemble de vecteurs de $\Z^d (\vec{v_1},  \vec{v_1}, \ldots ,\vec{v_m})$
		      qui représente la position relative des voisins (relative à la position de la cellule). Les voisins de la
		      cellule $\vec r$ sont $\vec r + \vec {v_1},  \ldots , \vec r + \vec {v_m}$.
		\item Une fonction $f : S ^ m \to S$ de transitions locale qui défini l'état d'une cellule à l'instant $t + 1$
		      en fonction des états de ses voisins a l'instant $t$.
	\end{itemize}
\end{definition}

Les automates cellulaires sont synchrones et homogènes dans le temps et dans l'espace.


\begin{definition}
	Une configuration est une fonction $\Z ^ d \to S$ qui décrit l'état courant de chaque cellule.

	Si $C$ est l'ensemble des configurations, alors un automate cellulaire défini une fonction $G: C \to C$ où si $c \in C$, alors
	$G(x) = c'$ est la configuration obtenue de $c$ en applicant la règle $f$ à toutes les cellules.

	En général, en itère la fonction $G$ et pour une configuration $c$ on s'intéresse à la suite :
	$$ G^0 (c) = e, G^1 (c), G^2 (c), \text{ où } G^i (c) = G ( G^{i - 1 }(c))$$
\end{definition}

\begin{definition}
	Un point fixe (ou nature morte) est une configuration $c$ \tlq $\forall k, G^k(c) = c$.
\end{definition}


\begin{definition}
	Une configuration $c$ est dite périodique (dans le temps) (ou oscillateur) si $\exists k$, \tq , $G^k(c) =c$. Le plus petit
	petit $k$ \tq $G^k(c) =c$ est dite la période d'oscillation.
\end{definition}

\begin{definition}
	Une configuration est dite ultimement fixe si $\exists k, \forall j, \ G^{k+j}(c) =  G^{k}(c)$
\end{definition}


\begin{definition}
	Une configuration est dite ultimement périodique si $\exists k, \exists l>0, \ G^{k+l}(c) =  G^{k}(c)$
\end{definition}


\begin{exemple}
	TODO
\end{exemple}


\subsection{Unicité des automates cellulaires}


\begin{theorem}
	Si deux automates cellulaires $A \et B$ réalisent la meme fonction $G$ alors $A \et B$ ne peuvent pas différer que par
	leur voisinages et les deux voisinages ne peuvent différent que par la presence (ou pas) de voisins inutiles (\cad
	des voisins dont l'état n'a aucune influence sur la valuer de $f$).
\end{theorem}


\begin{exemple}
	exemple
\end{exemple}


\begin{exemple}
	Pour $d = 2$ le Game of Life de Conway \cite{conwayGOL}:
	\begin{itemize}
		\item $S = \set {0,1}$
		\item $V = ((0,1), (1,0), (0,-1), (-1,0), (1,1), (1,-1), (-1,1), (-1,-1)$
		\item La fonction de transition
		      \begin{itemize}
			      \item Une cellule vivante reste vivante \ssi elle a 2 ou 3 vision vivant
			      \item Une cellule morte devient vivante \ssi 3 voisins sont vivants
		      \end{itemize}
	\end{itemize}


	Une implementation sur navigateur est disponible sur: \url{https://conwaylife.com/}.
\end{exemple}

\begin{definition}

	On apelle automates cellulaires élémentaires les automates cellulaires qui respectent :
	\begin{itemize}
		\item $d = 1$
		\item $S = \set{0,1}$
		\item $d = (-1,0,1)$
	\end{itemize}

	Les automates élémentaires sont exactement $2^8 = 256$.
\end{definition}

\begin{notation}
	Wolfgram a introduit un numérotation \cite{wolfgramClass} pour les automates cellulaires élémentaires ou chaque automate
	est associé a un entier entre 0 et 255.

	TODO: explain how it works
\end{notation}

\begin{definition}
	Soit $s \in S$ un état, on appelle $s$-support l'ensemble de cellules qui ount
	un état différent de $s$.
\end{definition}

\begin{notation}
	Souvent on impose qu'un état $q_0$ est un état "quiescent" (typiquement c'est l'état 0).

	Il est stable par $f$, \ie, $f(\underbrace{q_0,\ldots, q_0}_m)$
\end{notation}

\begin{definition}
	On apelle configurations finies ($q_0$-finies) les configurations ayant un $q_0$-support fini.
\end{definition}
