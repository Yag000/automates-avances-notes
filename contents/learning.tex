\section{Apprentissage de langages rationnels par queries et contrexemples}

Nous allons étudier un algorithme qui permet de \quotes{deviner} un automate reconnaissant un "langage" rationnel inconnu $L$,
en suivant un algorithme à complexité polynomiale \quotes{raisonnable}.
L'idée est que le ""learner"" (ou "étudiant"), une entité dont l'objectif est de retrouver le langage, pose des
questions à un ""oracle"" (ou "teacher") qui connait le langage cherché.

Les questions sont de deux types :
\begin{itemize}
	\item Les ""membership queries"" : est-ce qu'un "mot" $w$ donné appartient au langage $L$ ?
	\item Les ""conjectures"" : on présente un automate "conjecture". Si cet automate reconnaît $L$, alors
	      l'"oracle" répond \quotes{Ok}.
	      Sinon, il retourne un "mot" ""contre-exemple"" $w$. Ce "mot" vérifie soit $w \in L$ (alors qu'il
	      n’est pas accepté par notre automate), soit $w \notin L$ (alors qu'il est accepté par notre automate).
\end{itemize}

Cet algorithme découle du travail de \cite{angluinLearning}.


\subsection{Définitions}


\begin{definition}[Clôture par préfixes / (\blue{suffixes})]
	Un ensemble $S$ est ""clos par préfixes"" (\blue{suffixes}), si
	$$\forall w \in S, \text{ tout préfixe (\blue{suffixe}) de $w$ est dans } S.$$
\end{definition}

\begin{exercice}
	Indiquez lesquels des ensembles suivants sont "clos par préfixes" :
	\begin{enumerate}
		\item $\set{0,2,10,010} \ \times$
		\item $\set{110,1,0,\motvide,11} \ \checkmark$
		\item $\set{1110,10,1} \ \times$
		\item $\set{011,0,\motvide,11,01} \ \times$
		\item $\set{111,\motvide,11,1,0} \ \checkmark$
	\end{enumerate}
\end{exercice}

Pendant le processus, le "learner" maintient une ""table"", qui est un triplet $(S,E,T)$ où $S$ est un ensemble "clos par préfixes",
$E$ est un ensemble "clos par suffixes", et $T$ une fonction $(S \cup S\alphabet) \times E \to \set{0,1}$. Une "table" peut être
représentée par une matrice où:
\begin{itemize}
	\item les colonnes sont indexées par l'ensemble $E$ de "mots" "clos par suffixes";
	\item les lignes sont indexées par l'ensemble de "mots" $S \cup S\cdot \alphabet$;
	\item la case de la ligne $s$ et de la colonne $e$ contient $T(s,e)$.
\end{itemize}


\begin{definition}
	Si $s \in S \cup S\alphabet$, soit $n = \abs E$, on définit

	$$ \begin{array}{cccc}
			\text{\intro{row}}: & S\cup S\alphabet & \to     & [0,1]^n                       \\
			                    & s                & \mapsto & (T(s,e_1), \ldots, T(s, e_n))
		\end{array}$$


\end{definition}

\begin{definition}
	Une "table" est \emph{""close""} si
	$$ \forall t = sa \in S\alphabet, \exists s'\in S, \row t = \row {s'}$$
\end{definition}

\begin{definition}
	Une "table" est \emph{""cohérente""} si
	$$ \forall s_1, s_2 \in S, \row{s_1} = \row{s_2} \implies \forall a \in \alphabet, \row{s_1a} = \row{s_2a}$$
\end{definition}


\begin{exemple} \label{ex:tables}
	\ \newline
	\begin{itemize}
		\item La "table" suivante n'est pas "close" car $\nexists s \in S, \row {ab} = \row s$
		      \begin{center}
			      $$
				      \begin{tabular}{c|c|c|c}
					       &            & \multicolumn{2}{c}{$E$}     \\ \cline{3-4}
					       &            & $\motvide$              & a \\ \hline
					      \multirow{3}{*}{$S$}
					       & $\motvide$ & 0                       & 1 \\
					       & a          & 1                       & 1 \\
					       & b          & 0                       & 0 \\ \hline
					      \multirow{3}{*}{$S\alphabet$}
					       & aa         & 0                       & 0 \\
					       & ab         & 1                       & 0 \\
					       & bb         & 0                       & 1 \\ \hline
				      \end{tabular}
			      $$
		      \end{center}

		\item La "table" suivante est "close" et "cohérente"
		      \begin{center}
			      \begin{tabular}{c|c|c}
				                 & $\motvide$ & a \\ \hline
				      $\motvide$ & 0          & 0 \\
				      a          & 1          & 1 \\ \hline

				      b          & 1          & 1 \\
				      aa         & 0          & 0 \\
				      ab         & 0          & 0 \\
			      \end{tabular}
		      \end{center}
		\item La "table" suivante n'est pas "cohérente" car $\row {\motvide} = \row a$ mais
		      $\row {\motvide a} = \row a \neq \row {aa}$

		      \begin{center}
			      \begin{tabular}{c|c|c}
				                 & $\motvide$ & a \\ \hline
				      $\motvide$ & 0          & 1 \\
				      a          & 0          & 1 \\
				      b          & 0          & 0 \\ \hline

				      aa         & 0          & 0 \\
				      ab         & 0          & 1 \\
				      bb         & 1          & 1 \\
				      bb         & 1          & 0 \\
			      \end{tabular}
		      \end{center}
	\end{itemize}
\end{exemple}


\subsection{L'algorithme}

\begin{construction}\label{thm:automata-tables}
	À partir de chaque "table" $\Table$ "close" et "cohérente", on peut associer un "automate déterministe" $A = (Q, q_0, F, \delta)$ où :
	$$
		\begin{aligned}
			Q                   = & \setdef{ \row{s}} {s \in S }  \\
			q_0                 = & \set{ \row{\motvide} }        \\
			F                   = & \setdef{ \row{s} }{T(s) = 1 } \\
			\delta(\row{s}, a)  = & \  \row{sa}.
		\end{aligned}
	$$

	On note cet automate $A\Table$.
\end{construction}

\begin{exemple}
	L'automate associé à la "table" close et cohérente de l'exemple \ref{ex:tables} est: \\
	\begin{minipage}{0.5\textwidth}
		\centering
		\begin{tabular}{c|c|c}
			           & $\motvide$ & a \\ \hline
			$\motvide$ & 0          & 0 \\
			a          & 1          & 1 \\ \hline

			b          & 1          & 1 \\
			aa         & 0          & 0 \\
			ab         & 0          & 0 \\
		\end{tabular}
	\end{minipage}
	\begin{minipage}{0.5\textwidth}
		\centering
		\begin{automata}
			\digraph[scale=0.5]{automateTable}{
				rankdir=LR;

				node [shape=circle, style=filled, color=lightblue];
				q0 [label="00"];
				q1 [label="11"];

				start [shape=point];
				start -> q0;

				q0 -> q1 [label="a,b"];
				q1 -> q0 [label="a,b"];

				q1 [shape=doublecircle];
			}
		\end{automata}
	\end{minipage}
\end{exemple}

\vspace{1cm}


L'algorithme suivant permet de retrouver le langage cherché.

\begin{algorithme}
	Initialement, on peut "conjecturer" l'un des automates des Figures \ref{fig:algo-inL} et \ref{fig:algo-notinL}, selon si $\epsilon \in L$ ou non.

	\begin{twoautomata}
		\centering
		\begin{automata}
			\digraph[scale=0.5]{automateBase1}{
				rankdir=LR;

				node [shape=circle, style=filled, color=lightblue];
				q0 [label=" "];

				start [shape=point];
				start -> q0;

				q0 -> q0 [label="Σ"];

				q0 [shape=doublecircle];
			}
		\end{automata}
		\caption{Automate pour $\motvide \in L$}\label{fig:algo-inL}
	\end{twoautomata}
	\begin{twoautomata}
		\centering
		\begin{automata}
			\digraph[scale=0.5]{automateBase2}{
				rankdir=LR;

				node [shape=circle, style=filled, color=lightblue];
				q0 [label=" "];

				start [shape=point];
				start -> q0;

				q0 -> q0 [label="Σ"];
			}
		\end{automata}
		\caption{Automate pour $\motvide \notin L$}\label{fig:algo-notinL}
	\end{twoautomata}

	Tant que l'"oracle" donne un "contre-exemple"
	\begin{itemize}
		\item $w$ = "contre-exemple".
		\item $w$ et tous ses préfixes sont ajoutés à $S$.
		\item Les nouvelles lignes sont remplies en posant des "membership queries".
		\item Tant que $T$ n'est pas "cohérente" ou n'est pas close:
		      \begin{itemize}
			      \item Si $T$ n'est pas "cohérente" (\ie, $\exists s_1, s_2 \in S, \exists a \in \alphabet, \row{s_1} = \row{s_2} \et \row{s_1a} \neq \row{s_2a}$
			            \ie $\exists e \in E, T (s_1ae) \neq T(s_2ae)$)\\
			            alors on ajoute $ae$ et tous ses suffixes à $E$.
			      \item Si $T$ n'est pas "close", alors $\exists t \in S \alphabet, \forall s \in S, \row t \neq \row s$ et
			            on ajoute $t$ à $S$.
		      \end{itemize}
		\item On propose à l'"oracle" l'automate "conjecture" associé à $T$.
	\end{itemize}
\end{algorithme}


\subsection{Propriétés des tables closes et cohérentes}

\begin{notation}
	On note $\Table$ une "table", où $S$ est l'ensemble des lignes. $E$ l'ensemble des "mots" des colonnes et $T$ la fonction de vérité.
	Parfois, par abus de notation, on dira que la table est $T$.
\end{notation}

\begin{definition}
	Un "automate déterministe" $(Q, q_0, F, S)$ est ""en accord"" avec une "table" $\Table$ si
	$$ \forall s \in S \cup S\alphabet, \forall e \in E, \delta(q_0,se) \in F \iff T(se) = 1$$
\end{definition}

\begin{theorem}
	Si $A\Table$ est l'automate issu d'une "table" $\Table$ alors $A$ est "en accord" avec $T$ et tout autre
	automate "en accord" avec $T$ et non isomorphe à $A$ possède au moins un état de plus que $A$.
\end{theorem}

La preuve de ce théorème est un corollaire des trois lemmes suivants.

\begin{lemma} \label{lem:learning-9}
	Si $(S,E,T)$ est "close" et "cohérente" alors dans l'automate $A = \mathcal A (S,E,T)$
	$$\forall s \in S \cup S\alphabet, \delta(q_0,s) = \row s$$
\end{lemma}

\begin{proof}
	Induction sur la taille de $s$.
	\begin{itemize}
		\item Si $\len s = 0$, alors $s = \motvide$ et donc $\delta(q_0, \motvide) = \row \motvide$
		      est vrai par construction de l'automate.
		\item Supposons l'énoncé vrai pour tout "mot" appartenant à $S \cup S\alphabet$ de longueur inférieure ou égale à $n$ et
		      soit $t$ un "mot" de $S \cup S\alphabet$ de longueur $n +1$, alors $ t  = s \cdot x,$ avec $s \in S$, car
		      si $t \in S\alphabet$, $s \in S$ trivialement et si $t \in S$, comme $S$ est clos par préfixes, on a $s \in S$.

		      \begin{eqnarray*}
			      \delta(q_0, t) &=& \delta(\delta(q_0, s), x) \\
			      &=& \delta(\row s, x)  \reason {Par hypothèse d'induction} \\
			      &=& \row {sx}  \reason {Par la définition de $\delta$ dans \ref{thm:automata-tables}} \\
			      &=& \row t
		      \end{eqnarray*}
	\end{itemize}
\end{proof}


\begin{lemma}
	Si la "table" $(S,E,T)$ est "close" et "cohérente", alors l'automate $A = \mathcal A (S,E,T)$
	est "en accord" avec la fonction $T$, \ie
	$$\forall s \in S \cup S\alphabet, \forall e \in E, \delta(q_0,se) \in F \iff T(se) =1$$
\end{lemma}

\begin{proof}
	Induction sur la taille de $e$.

	\begin{itemize}
		\item Si $\len e = 0$, alors $e = \motvide$.
		      Soit $s \in S \cup S \alphabet$, et donc $se = s$.
		      \begin{itemize}
			      \item Si $s \in S$, alors par le lemme précédent,
			            $\delta(q_0, s) = \row {s}$, et comme $\row {s} \in F \iff T(s) = 1$ par construction de l'automate, on retrouve le résultat cherché.
			      \item Si $s \in S\alphabet$, comme la "table" est "close", il existe $s' \in S$, \tq $\row {s'} = \row s$ qui vérifie
			            $\row {s'} \in F \iff T(s') = 1$ par construction de l'automate.

		      \end{itemize}
		\item Supposons l'énoncé vrai pour tout "mot" de $E$ de longueur inférieure ou égale à $n$. 
		Soit $e \in E$ tel que $\len e = n + 1$,
		      et on pose $e = x e'$, avec $x \in \alphabet$. Soit $s \in S \cup S\alphabet$, il existe
		      $s' \in S$ \tq $\row s = \row {s'}$ car la "table" est "close".
		      \begin{eqnarray*}
			      \delta(q_0, se) &=& \delta(\delta(q_0,s),e) \\
			      &=& \delta(\delta(q_0,s),xe') \\
			      &=& \delta(\row s,xe') \reason{par le lemme \ref{lem:learning-9}} \\
			      &=& \delta(\row {s'},xe') \reason{car $\row s = \row {s'}$} \\
			      &=& \delta(\delta (\row {s'}, x), e') \\
			      &=& \delta(\row {s'x}, e') \reason {par définition de $\delta$}\\
			      &=& \delta(\delta(q_0, s'x), e') \reason{par le lemme \ref{lem:learning-9}} \\
			      &=& \delta(q_0, s'x e')
		      \end{eqnarray*}
		      Comme $\len {e'} = n$ on peut appliquer l'hypothèse d'induction et on a :
		      $$ \delta(q_0, se) = \delta(q_0, s'x e') \in F \iff T(s'x e') = 1$$

	\end{itemize}
	Puisque $\row s = \row {s'}$  et $e = xe'$, on a que $T(s'xe') = T(sxe') = T(se)$.
\end{proof}

\begin{lemma}
	Soit $(S,E,T)$ une table "close" et "cohérente", alors si l'automate $A = \mathcal A (S,E,T)$
	a $n$ états, tout automate $A'$ "en accord" avec $T$ et ayant au plus $n$ états est isomorphe à $A$.
\end{lemma}

\begin{proof}
	Soit $A' = (Q', q_0', F', \delta')$ un autre automate "en accord" avec $T$ et ayant au plus $n$ états.

	L'objectif est d'exhiber un isomorphisme entre les deux automates. La construction et la vérification
	de cet isomorphisme suivent les arguments détaillés dans la preuve du Lemme 4 de \cite{angluinLearning}.
\end{proof}

\subsection{Terminaison de l'algorithme}

\begin{lemma}
	Si $(S,E,T)$ est une "table", alors tout automate "en accord" avec $T$ possède au
	moins autant d'états que le nombre de valeurs distinctes de l'ensemble $R = \setdef {\row s} {s \in S}$.
\end{lemma}

\begin{proof}
	Soit $A = (Q,q_0,F,\delta)$ un automate "en accord" avec $T$ et définissons
	$f: R \to Q$, $f(\row s) = \delta(q_0, s)$.

	Soient $s_1 \et s_2$ \tq $\row {s_1} \neq \row{s_2}$, donc $\exists e \in E, T(s_1e) \neq T(s_2e)$.
	Comme $A$ est "en accord" avec $T$, seulement l'un de deux "mots" est reconnu.

	Sans perte de généralité, on suppose $s_1e$ reconnu et $s_2e$ pas reconnu, donc
	$\delta (q_0, s_1e) \in F$, alors que $\delta (q_0,s_2e) \notin F$. Cela implique que
	$\delta (q_0,s_1) \neq \delta (q_0,s_2)$.


	Pour deux rows distincts correspondent deux états distincts de $A$, et donc $A$ a au moins autant d'états que $\abs R$.
\end{proof}

Soit $n$ le nombre d'états de l'"automate minimal" reconnaissant le "langage" inconnu $L$ cherché.

\begin{remarque}
	Le nombre d'états des différents automates "conjectures" présentés à l'"oracle" ne peut qu'augmenter.
	\begin{itemize}
		\item si on ajoute un "mot" à $E$ parce que la "table" n'était pas "cohérente", le nombre de valeurs "rows" distincts augmente
		      d'une unité.
		\item si on ajoute un "mot" à $S$ parce que la "table" n'était pas "close", le nombre de valeurs de "rows" distincts augmente d'une unité.
	\end{itemize}
	On déduit qu'on peut rencontrer une "table" non "close" ou non "cohérente" au plus $n-1$ fois tout au long de l'algorithme.

	En particulier, on déduit qu'après un nombre fini d'étapes de l'algorithme, on arrive toujours à trouver
	une "table" "close" et "cohérente" et à émettre une "conjecture".
\end{remarque}


Combien de "conjectures" le "learner" émet-il avant de donner la bonne ?

Soit $(S,E,T)$ une "table" "close" et "cohérente" et $A(S,E,T)$ l'automate associé.
Supposons que $A$ soit une "conjecture" fausse, et donc que l'"oracle" fournit un "contre-exemple" $t$.

L'automate $A(S,E,T)$ et l'automate du langage cherché sont en désaccord sur le "mot" $t$, donc ces deux automates
ne sont pas équivalents, et donc $A(S,E,T)$ a au moins un état de moins que l'automate cherché.
Donc $A(S,E,T)$ a au plus $n-1$ états. On en déduit que le nombre de fausses "conjectures" émises pas le "learner" est au plus $n-1$.

On en déduit que l'algorithme s'arrête toujours, au pire après avoir rendu $n-1$ "tables" "closes" et
"cohérentes" et après avoir émis au plus $n-1$ fausses "conjectures".


\subsection{Analyse de la complexité}

Elle dépend du nombre $n$ et de la longueur du plus long "contre-exemple" fourni par l'oracle, notée $m$.


\begin{itemize}
	\item A chaque fois qu'on trouve une "table" non "cohérente", on ajoute un "mot" à $E$.
	\item A chaque fois qu'on trouve une "table" non "close", on ajoute un "mot" à $S$.
	\item A chaque fois qu'on émet une ""conjecture"" fausse, on ajoute au plus $m$ "mots" à $S$ (le "contre-exemple" et ses préfixes).
\end{itemize}

On en déduit que $\abs E \leq n$.

La longueur maximale des "mots" de $E$ augmente au plus d'une unité à chaque fois qu'on rend la "table" "cohérente"
(on ajoute à $E$ un "mot" $xe$ avec $e \in E$). On déduit que la longueur maximale d'un "mot" de $E$ est aussi $<n$.

De même, on a que  $\abs S \leq n + m(n-1)$. %TODO: explanation of each number.

La longueur maximale des "mots" de $S$ est inférieure à $m - n -1$.


On déduit que le cardinal maximal de $(S \cup S \Sigma) \times E$ (le nombre maximal de classes de la "table") est : $(k+1)(n+m(n-1))n \in O(mn^2)$.


Considérons le coût de chaque type d'opérations :
\begin{itemize}
	\item Vérifier si une "table" est "close" et "cohérente" se fait en temps polynomial relativement à la taille de la "table".
	\item Ajouter un "mot" à $S$ ou à $E$ nécessite au plus $nm$ "membership queries".
	\item Construire une ""conjecture"" à partir d'une "table" "cohérente" et "close" est faisable en temps polynomial relativement à
	      la taille de la "table".
	\item Un "contre-exemple" nécessite l'addition d'au plus $m$ "mots" à $S$ et cette opérations est effectuée au plus $n-1$ fois.
\end{itemize}

\vspace{0.5cm}

On on déduit le théorème suivant :

\begin{theorem}
	L'algorithme produit toujours un automate isomorphe à l'"automate minimal" du language cherché, de plus, le temps de calcul peut être
	exprimé par un polynôme en $n$ et en $m$.
\end{theorem}

\begin{remarque}
	Si l'"oracle" fournit toujours un "contre-exemple" de taille $\leq n$ (ce qui est toujours possible), alors le polynôme en question dépend de $n$ seulement.
\end{remarque}

