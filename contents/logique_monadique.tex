\section{Logique monadique du second ordre (MSO)}

$ a \neg {\models} \exists x \exists y (x < y)$

Cette position est satisfaite par $w \in \mots$ \ssi $\abs w \geq 2$.

Pour une proposition $\phi$ on a un langage $\La_\phi = \setdef {w \in \mots} {w \models \phi}$.


TODO: exemple


%TODO: Comment and add line breaks
$$\phi_3 = \exists X (\forall x (\forall z (z \geq x) \ra X(x))) \land \forall x (\forall z (z \leq x) \ra \lnot X(x)) \land
	\forall x \forall y (((x < y) \land \forall z(z > x) \ra (z \geq y)) \ra (X (x) \lra \lnot X(y)))$$


$X \rightsquigarrow $ un ensemble de position dans un mot


$X (x)\rightsquigarrow$ vrai \ssi la position $x$ est dans $X$


$\La_{\phi_3} = \setdef {w \in \mots} {\abs w \equiv 0 (\text {mod} 2)}$

\subsection{Syntaxe}

\begin{definition} [La logique du premier ordre]
	\begin{itemize}
		%TODO: Improve grammar
		%TODO, make cool V
		\item $V$ un ensemble de variables : $\set {x,y,z,x_1,y_1,z_1, \ldots}$
		\item \underline{Prédicats numériques} :  $\PP = \set {R_i^j, i > 0, j \geq o}$
		\item \underline{Formules atomiques} : $\alpha ::= Q_0 x | R_i^j(x_1, \ldots, x_j)$
		\item \underline{Formules de premier ordre} : $\phi ::= \alpha | \phi \land \phi | \lnot \phi | \exists x \phi$
	\end{itemize}
\end{definition}


\begin{definition} [La logique du second ordre]
	%TODO: Add X(x) pour les formules atomiques du second ordre
	\begin{itemize}
		%TODO: Improve grammar
		\item un ensemble de variables de second ordre : $X,Y,Z,X_1,Y_1,Z_1 \cdots$
		\item \underline{Formules } : $\phi ::= \alpha | \phi \land \phi | \lnot \phi | \exists x \phi | \exists X \phi$
	\end{itemize}
\end{definition}


\subsection{Sémantique}

\subsubsection{Sémantique de la logique monadique du premier ordre}

Pour $x \in \N$, $x$ est interpreté comme un élément de $\set {1, \cdots, n}$

$$ \interpret {R_i^j}_n \subseteq \set {1, \cdots, n}^j$$
est une interpretation de chaque prédicat $R_i^j$ pour tout $n \geq 1$.
$$ \interpret {<}_n \subseteq \set {1, \cdots, n} \times \set {1, \cdots, n} = \setdef {(i,j)} {i<j}$$


\begin{definition}
	Une $\V$-structure, pour $\V \subseteq V$ es un mot de la forme $(a_1, U_1) \ldots (a_n, U_n)$ où $a_1, \ldots, a_b \in \alphabet$ et
	$U_1, \ldots, U_n \subseteq V$ tels que

	\begin{enumerate}
		\item $U_i \cap U_j = \emptyset, i \neq j$
		\item  $\bigcup_{i=1}^n U_i = \V$
	\end{enumerate}
\end{definition}

\begin{definition}[relation de satisfaction]
	On définit la relation de satisfaction $w \models \phi$ où $w$ est une $\V-structure$ et $\phi$ une formule de premier ordre \tq
	\begin{enumerate}
		\item $\forall x \in free(\phi) \implies x \in \V$ (free = variables libres)
		\item Les quantificateurs distincts dans $\phi$ lient des variables distincts
	\end{enumerate}
\end{definition}

Si $\phi$ est un proposition, \cad, $free(\phi) \neq \emptyset$ alors on a $\La_{\phi} = \setdef {w \text { les } \emptyset-\text{structures}} {w \models \phi} \subseteq \mots$.


Pour une interpretation $\interpret {R_i}){(i,j)}$ de prédicats numérique on défini la relation $w \models \phi$ par induction su la structure de la formule $\phi$.

\begin{itemize}
	\item $w \models Q_a x$ \ssi $w$ contient une lettre $(a,U_i)$ avec $x \in U_i$
	\item $w \models R_i^j(x_1, \ldots, x_j)$ \ssi $\interpret {R_i}){\abs w}(k_1, \ldots, k_j)$ es vrai où les $k1, \ldots, k_j$ sont les positions dean $w$ où les variables
	      $x_1, \cdots x_j$ apparaissent.
	\item $w \models \phi_1 \land \phi_2$ \ssi $w \models \phi_1$ et $w \models \phi_2$
	\item $w \models \lnot \phi$ \ssi $w \neg \models \phi$
	\item $w = (a_1, U_1) \ldots (a_n, U_n)$ est une $\V$-structure.
	      $w \models \exists \phi$ \ssi $\exists i \in \set {1, \ldots, n}$
	      $(a_1, U_1) \ldots  (a_i, U_i \cup \set x)  \ldots (a_n, U_n) \models \phi$
	      %TODO: say what kind of structure it is, V \cup \set x
\end{itemize}

\begin{exemple}
	$$abc \models_? \exists x \exists y (x < y)$$

	$(a, \set x)(b, \emptyset)(c, \emptyset) \models_? \exists y (x \leq y)$ est une $\set x $-structure

	$(a, \set x)(b, \set y)(c, \emptyset) \models_? \exists y (x < y) \iff (1,2) \in \interpret {<}_3 \iff 1 < 2$

	Version 2

	%TODO: petit dessin Photo
\end{exemple}


\begin{exemple}
	TODO: add it (important)
\end{exemple}


\subsubsection{Sémantique de la logique monadique du second ordre}


$V_1$ les variables du premier ordre

$V_2$ les variables du second ordre


\begin{definition}
	Une $(\V_1,\V_2)$-structure est un mot sur $\alphabet \times 2^{\V_1} \times  2^{\V_2}$, \cad un mot de la forme
	$(a_1,U_1,U_1) \cdots (a_n,U_n,V_n)$ où
	\begin{itemize}
		\item Les $U_i$ sont des endebles de variables du premier ordre.
		\item Les $V_i$ sont des endebles de variables du second ordre.
		\item Et
		      $(a_1,U_1) \cdots (a_n,U_n)$ est une $\V_1$-structure.
	\end{itemize}
\end{definition}

\begin{definition}
	On définit la relation de satisfaction $w \models \phi$ où $w$ est une $(\V_1, \V_2)-structure$ et $\phi$ une formule de
	MSO avec les memes contraintes sur les variables que pour la logique du premier ordre %TODO: Add reference
	et de plus $w \models X(x)$ et $w \models \exists X \phi$
\end{definition}




\begin{itemize}
	\item $w \models X (x)$ \ssi $w$ contient une lettre $(a,S,T)$ où $x \in S$ et $X \in T$
	\item $w \models \exists X \phi$ \ssi $\exists J$ un ensemble de positions dans $w$ avec la propreté:
	      la $(\V_1, \V2)$-structure $w'$ obtenue en remplaçant $(a_i, S_i,T_i)$ pour $i \in J$ par $(a_i, S_i, T_i \cup \set x)$ satisfait $\phi$.
\end{itemize}


\begin{exemple}
	TODO
\end{exemple}


\subsection{Relation avec les expressions rationnelles}


\begin{definition}
	Un langage $L$ est définissable dans MSO[<] (la logique monadique du second ordre avec un prédicat binaire <).
	\ssi il existe une formule $\phi \in MSO[<]$ \tlq $L = \setdef {w\in \mots} {w \models \phi}$
\end{definition}

\begin{definition}
	Un langage $L \subseteq \mots$ est définissable dans $FO[<]$ (la logique du premier ordre avec <)
	\ssi $\exists \phi \in FO[x]$ \tlq $L = \setdef {w \in \mots} {w \models \phi}$
\end{definition}

\begin{theorem}
	Un langage est définissable dans $MSO[<]$ \ssi $L$ est régulier.
\end{theorem}


\begin{theorem}
	Un langage est définissable dans $FO[<]$ \ssi il existe une expression rationnelle sans étoile $r$ définissant un langage
	\ssi
	le monoïde syntaxique du langage est apériodique (\cad il ne contient aucun sous groupe non trivial).
\end{theorem}



