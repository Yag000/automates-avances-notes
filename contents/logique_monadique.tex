\section{Logique monadique du second ordre (MSO)}

Est-ce qu'il existe $w \in \mots$ \tq $w \, \models \, \exists x \exists y (x < y)$ soit vraie ?
($x,y$ sont interpretés comme des positions dans le mot $w$.)
Par exemple, $a \, \nvDash \, \exists x \exists y (x < y)$
Ici, cette proposition est satisfaite par $w \in \mots$ \ssi $\abs w \geq 2$.


Pour une proposition $\phi$, on a un langage $\La_\phi = \setdef {w \in \mots} {w \models \phi}$.


\begin{exemple}
	$\Sigma = \set{a,b}$
	$\phi_2 = \exists x \exists y (\forall z (z \geq x) \land Q_a x \land \forall z (z \leq y) \land Q_b y)$
	où $\leq, \geq$ sont interpretés comme les relations habituelles sur $\N$.
	$Q_a x$ signifie qu'en position $x$, on retrouve la lettre $a$.
	$\La_{\phi_2} = \setdef {w \in \kleene{\set {a,b}}} {w \models \phi_2} = a(a+b)^*b$
\end{exemple}


%TODO: Comment and add line breaks
\begin{exemple} \label{example:phi3}

	\begin{eqnarray*}
		\phi_3 &=& \exists X ( \forall x (\forall z (z \geq x) \ra X(x)) \\
		&\land& \forall x (\forall z (z \leq x) \ra \lnot X(x)) \\
		&\land& \forall x \forall y (((x < y) \land \forall z(z > x) \ra (z \geq y)) \ra (X (x) \lra \lnot X(y))))
	\end{eqnarray*}



	$X \rightsquigarrow $ un ensemble de positions dans un mot


	$X (x)\rightsquigarrow$ vrai \ssi $x \in X$


	$\La_{\phi_3} = \setdef {w \in \mots} {\abs w \equiv 0 (\text { mod } 2)}$
\end{exemple}

\subsection{Syntaxe}

\begin{definition} [La ""logique du premier ordre""]
	\ \newline
	\begin{itemize}
		\item $\Vi$ un ensemble de variables : $\set {x,y,z,x_1,y_1,z_1, \ldots}$
		\item \underline{""Prédicats numériques""} :  $\PP = \set {R_i^j, i > 0, j \geq 0}$
		\item \underline{""Formules atomiques""} :
		      \begin{syntaxdef}
			      \syntaxHeader {\alpha} {Q_a x} {Lettre $a$ en position $x$}
			      \syntax {R_i^j(x_1, \ldots, x_j)} {"Prédicats numériques"}
		      \end{syntaxdef}
		\item \underline{""Formules du premier ordre""} :
		      \begin{syntaxdef}
			      \syntaxHeader {\phi} {\alpha} {""Formules atomiques""}
			      \syntax {\phi \land \phi} {Conjonction}
			      \syntax {\lnot \phi} {Négation}
			      \syntax {\exists x \phi} {""Quantificateur existentiel""}
		      \end{syntaxdef}
	\end{itemize}
\end{definition}


\begin{definition} [La ""logique du second ordre""]
	C'est une extension de la "logique du premier ordre" :
	\begin{itemize}
		\item un ensemble de ""variables du second ordre"" : $X,Y,Z,X_1,Y_1,Z_1 \ldots$
		\item \underline{Formules atomiques} :
		      \begin{syntaxdef}
			      \syntaxExtension{\alpha}
			      \syntax {X(x)} {Appartenance à $X$}
		      \end{syntaxdef}
		\item \underline{""Formules du second ordre""} :
		      \begin{syntaxdef}
			      \syntaxExtension{\phi}
			      \syntax {\exists X \phi} {""Quantificateur existentiel sur des ensembles""}
		      \end{syntaxdef}
	\end{itemize}
\end{definition}


\begin{notation}[""Quantificateur universel""]
	On note $\forall x \phi$ la formule $\lnot \exists x (\lnot \phi)$. Cette notation s'étend aussi
	à la logique du second ordre.
\end{notation}

\subsection{Sémantique}

\subsubsection{Sémantique de la logique monadique du premier ordre}

Pour $n \in \N$ la longueur d'un mot, $x$ est interpreté comme un élément de $\set {1, \ldots, n}$.

$ \interpret {R_i^j}_n \subseteq \set {1, \ldots, n}^j$ est une ""interpretation"" de chaque prédicat $R_i^j$ pour tout $n \geq 1$.
$$ \interpret {<}_n \subseteq \set {1, \ldots, n} \times \set {1, \ldots, n} = \setdef {(i,j)} {i<j}$$


\begin{definition}
	Une $\V$-structure, pour $\V \subseteq \Vi$ est un mot de la forme $(a_1, U_1), \ldots, (a_n, U_n)$ où $a_1, \ldots, a_n \in \alphabet$ et
	$U_1, \ldots, U_n \subseteq \Vi$ tels que

	\begin{enumerate}
		\item $U_i \cap U_j = \emptyset, \forall i,j, \, i \neq j$
		\item  $\bigcup_{i=1}^n U_i = \V$
	\end{enumerate}
\end{definition}

\begin{definition}[relation de satisfaction]
	On définit la relation de satisfaction $w \models \phi$ où $w$ est une $\V$-structure et $\phi$ une formule de premier ordre \tq
	\begin{enumerate}
		\item $\forall x \in \text{FV}(\phi) \implies x \in \V$ (FV = ensemble des variables libres)
		\item Les quantificateurs distincts dans $\phi$ lient des variables distinctes
	\end{enumerate}
\end{definition}

Si $\phi$ est un proposition, \cad, $FV(\phi) = \emptyset$ alors on a $\La_{\phi} = \setdef {w \text { les } \emptyset\text{-structures}} {w \models \phi} \subseteq \mots$.


Pour une interprétation $\interpret {R_i^j}_{{i,j}}$ de prédicats numériques, on définit la relation $w \models \phi$ par induction sur la structure de la formule $\phi$.

\begin{itemize}
	\item $w \models Q_a x$ ssi $w$ contient une lettre $(a,U_i)$ avec $x \in U_i$
	\item $w \models R_i^j(x_1, \ldots, x_j)$ ssi $\interpret {R_i}_{\abs w}(k_1, \ldots, k_j)$ est vrai où les $k1, \ldots, k_j$ sont les positions dans $w$ où les variables
	      $x_1, \ldots x_j$ apparaissent.
	\item $w \models \phi_1 \land \phi_2$ ssi $w \models \phi_1$ et $w \models \phi_2$
	\item $w \models \lnot \phi$ ssi $w \nvDash \phi$
	\item si $w = (a_1, U_1) \ldots (a_n, U_n)$ est une $\V$-structure.
	      $w \models \exists \phi$ ssi $\exists i \in \set {1, \ldots, n}$
	      $(a_1, U_1) \ldots  (a_i, U_i \cup \set x)  \ldots (a_n, U_n) \models \phi$
	      (il s'agit d'une $\V \cup {x}$-structure.)
\end{itemize}

\begin{exemple}
	$$abc \models_? \exists x \exists y (x < y)$$

	$V_1 = (a, \set x)(b, \emptyset)(c, \emptyset) $ est une $\set x $-structure et on se demande si
	$V_1 \models_? \exists y (x \leq y)$.

	On prend maintenant $V_2 = (a, \set x)(b, \set y)(c, \emptyset)$ et :

	\begin{eqnarray*}
		V_2 \models_? (x < y) &\iff& (1,2) \in \interpret {<}_3\\
		&\iff& 1 < 2
	\end{eqnarray*}

	Et donc $abc \models \exists x \exists y (x < y)$
\end{exemple}


Soit $\phi$ et $\psi$ deux formules, $\phi \iff \psi$ \ssi $\La_{\phi} = \La_{\psi}$.

\begin{abbreviation}
	$\forall x \phi = \neg \exists x (\neg \phi)$
\end{abbreviation}

\begin{exemple}
	\begin{eqnarray*}
		abc  &\models&\exists x \exists y (\forall z (z \geq x) \land Q_a x \land \forall z (z \leq y) \land Q_b y)\\
		(a, \set x), (c, \emptyset), (b, \emptyset)  &\models &\exists y (\forall z (z \geq x) \land Q_a x \land \forall z (z \leq y) \land Q_b y)\\
		a, \set x), (c, \emptyset), (b, \set y) & \models& (\forall z (z \geq x) \land Q_a x \land \forall z (z \leq y) \land Q_b y)\\
		(a, \set x), (c, \emptyset), (b, \set y)  &\models& (\lnot \exists z (z < x) \land Q_a x \land \lnot \exists z (z > y) \land Q_b y)\\
	\end{eqnarray*}
	\begin{itemize}
		\item $(a, \set x), (c, \emptyset), (b, \set y)  \models \lnot \exists z (z < x) \Leftarrow  (a, \set x), (c, \emptyset), (b, \set y)  \not\models \exists z (z < x) $
		\item $(a, \set x), (c, \emptyset), (b, \set y)  \models Q_a x \Leftarrow  (a, \set x) $
		\item $(a, \set x), (c, \emptyset), (b, \set y)  \models  \lnot \exists z (z > y) \Leftarrow  (a, \set x), (c, \emptyset), (b, \set y)  \not\models \exists z (z > y) $
		\item $(a, \set x), (c, \emptyset), (b, \set y)  \models  Q_b y \Leftarrow  (b, \set y)$
	\end{itemize}

\end{exemple}


\subsubsection{Sémantique de la logique monadique du second ordre}


$V_1$ les variables du premier ordre

$V_2$ les variables du second ordre


\begin{definition}
	Une $(\V_1,\V_2)$-structure est un mot sur $\alphabet \times 2^{\V_1} \times  2^{\V_2}$, \cad un mot de la forme
	$(a_1,U_1,U_1), \ldots, (a_n,U_n,V_n)$ où
	\begin{itemize}
		\item Les $U_i$ sont des ensembles de variables du premier ordre.
		\item Les $V_i$ sont des ensembles de variables du second ordre.
		\item $(a_1,U_1), \ldots, (a_n,U_n)$ est une $\V_1$-structure.
	\end{itemize}
\end{definition}

\begin{definition}
	On définit la relation de satisfaction $w \models \phi$ où $w$ est une $(\V_1, \V_2)-structure$ et $\phi$ une formule de
	MSO avec les mêmes contraintes sur les variables que pour la logique du premier ordre %TODO: Add reference
	plus $w \models X(x)$ et $w \models \exists X \phi$
\end{definition}

L'induction est la même que pour le premier ordre, avec ces cas en plus :
\begin{itemize}
	\item $w \models X (x)$ ssi $w$ contient une lettre $(a,S,T)$ où $x \in S$ et $X \in T$
	\item $w \models \exists X \phi$ ssi $\exists J$ un ensemble de positions dans $w$ avec la propriété:
	      la $(\V_1, \V2)$-structure $w'$ obtenue en remplaçant $(a_i, S_i,T_i)$ pour $i \in J$ par $(a_i, S_i, T_i \cup \set X)$ satisfait $\phi$.
\end{itemize}


\begin{exemple}
	TODO
\end{exemple}


\begin{definition}
	Un langage $L$ est définissable dans MSO[<] (la logique monadique du second ordre avec un prédicat binaire <)
	\ssi il existe une formule $\phi \in MSO[<]$ \tlq $L = \setdef {w\in \mots} {w \models \phi}$
\end{definition}

\begin{definition}
	Un langage $L \subseteq \mots$ est définissable dans $FO[<]$ (la logique du premier ordre avec <)
	\ssi $\exists \phi \in FO[<]$ \tlq $L = \setdef {w \in \mots} {w \models \phi}$
\end{definition}


\subsection{Relation avec les expressions rationnelles}

\begin{theorem}
	Un langage est définissable dans $MSO[<]$ ssi $L$ est régulier.
\end{theorem}


\begin{proofI}
	\begin{itemize}
		\item \fbox{$\Leftarrow$}

		      Soit $A = \AFD$ un automate fini déterministe qui accepte un langage $L$.  On peut supposer que $L \subseteq \mots$ (
		      car $\motvide \in L$, on va prendre la disjonction de la formule obtenue pour $L \setminus \set {\motvide}$ avec $\forall x \neg (x = x)$)

		      Soit $w \in \mots$, Alors $w$ est reconnu par l'automate $A$ ssi $\exists X_0,\ldots,X_{k-1} \subseteq \set{1, \ldots , \abs w}$
		      tels que les propriétés suivantes soient vérifiés :
		      \begin{enumerate}
			      \item $\bigcup_{i=0}^{k - 1} X_i = \set{1, \ldots, \abs w}$
			      \item $\forall i < j, X_i \cap X_j = \emptyset$
			      \item $1 \in X_0$
			      \item $\forall j \in \set{1, \ldots, \abs w}$ si $j \in X_i \et j+i \in X_e$ et si $a$ est la lettre en position $j$ dans le mot
			            $w$, alors $\delta (q_i,a) = q_e$.

			      \item Si $\abs w \in X$ et $a$ est la dernière lettre de $w$, alors $\delta (q_j,a) \in F$.
		      \end{enumerate}

		      Supposons que $w = \decomp a n$ est accepté par $A$. On construit les ensembles $(X_i)_{i \in \enum 0 {k-1}}$
		      tel que $i \in X_j$ ssi après avoid lu les premiers $i-1$ lettres de $w$ on arrive a l'état $q_j$
		      %TODO: Diagram ? 

		      Les ensembles $X_0, \ldots, X_{k-1}$ satisfait les 5 propriétés.
		      \begin{enumerate}
			      \item $X_0 \cup \ldots \cup X_{k-1} = \enum i {\abs w}$. L'inclusion a gauche est vraie par définition. Montrons l'autre inclusion.
			            On considère l'unique chemin dans l'automate $A$ obtenu en lisant les premieres $i-1$ lettres de $w$ à partir de l'état $q_0$.
			            Supposons qu'on arrive dans l'état $q_j$. Par définitions $u \in X_j$.
			      \item $\forall i < j, X_i \cap X_j = \emptyset$ est vrai car l'automate est déterministe.
			      \item $1 \in X_0$ car si on lit les premieres $0$ lettres de $w$ on reste a l'état $q_0$.
			      \item Après les premieres $j-1$ lettres on arrive a $q_i$. $a$ est la lettre en position $j$. Donc après les
			            premieres $j$ lettres on arrive dans l'état $\delta (q_i, a)$. Si après les premières $j$ lettres on arrive dans l'état $q_l$, alors
			            $\delta (q_i, a) = q_l$.
			      \item $w$ est accepté par $A$. Donc si $\abs w \in X_j$ alors après avoir lu les premieres $\abs w - 1$ lettres on arrive dans $q_j$.
			            Si $a$ est la dernière lettre de $w$, alors $\delta (q_j,a)$ est un tat acceptant car $w$ est accepté.
		      \end{enumerate}
		      On construit la formule
		      $$ \phi = \exists X_0 \ldots \exists X_{k-1} (\phi_1 \land \ldots \land \phi_5)$$
		      où
		      \begin{itemize}
			      \item $\phi_1 = \forall x (X_0(x) \lor \ldots \lor X_{k-1}) = \forall x \bigvee_{i=1}^{k-1} X_i(x)$
			      \item $\phi_2 = \forall x \bigwedge_{0\leq i < j \leq k-1}\lnot (X_i(x) \land X_j(x))$
			      \item $\phi_3 = \exists x \left((\forall y, x \leq y) \land X_0(x) \right)$
			      \item $\phi_4 = \forall x \left( \forall y (y = x + 1)  \ra \bigwedge_{0\leq i < l < k} ( (X_i (x) \land X_l(y) )\ra \bigvee_{S_l} Q_a(x))\right)$\\
			            Où $S_l = \setdef {a \in \alphabet} {\delta (q_i,a) = a_l}$
			      \item $\phi_5 = \forall x \left( \forall y (x \geq y)  \ra \bigwedge_{i = 0}^{k-1} ( X_i(x) \ra \bigvee_{T_i} Q_a(x))\right)$\\
			            Où $T_i = \setdef {a \in \alphabet} {\delta (q_i,a) \in F}$

		      \end{itemize}

		\item \fbox{$\Rightarrow$}

		      On suppose que $L$ est défini par une formule de $MSO[<]$.
		      La preuve est par induction sur la structure de la formule.

		      $L \subseteq (\alphabet \times 2^{\V_1} \times 2 ^ {\V_2})^*$.

		      %TODO: Rappel ?
		      $\La$ est l'ensemble de toutes les $(\V_1, \V_2)$-structures, pour $\V_1$ en ensemble de variables du premier
		      ordre et $\V_2$ un ensemble de variables du second ordre.

		      \begin{exercice}
			      Trouver un automate sur l'alphabet $\alphabet \times 2^{\V_1} \times 2 ^ {\V_2}$ qui accepte $\La$
		      \end{exercice}

		      \begin{exercice}
			      Trouver un automate qui accepte les $(\V_1, \V_2)$-structures tel qu'une variable de premier ordre $x$ apparait
			      dans une lettre de la forme $(a,S,T)$. Le but est de \mq le $\lang {Q_a(x)}$ est régulier.
		      \end{exercice}

		      \begin{exercice}
			      Montrer que $\La_{X(x)}$ est régulier.
		      \end{exercice}

		      \begin{exercice}
			      Montrer que $\La_{x< y}$ est régulier.
		      \end{exercice}


		      Soit $\phi = \exists x \psi$ et supposons que $L_{\psi}$ est régulier et donc accepté par un automate $A = \AFD$ sur l'alphabet
		      $\alphabet \times 2^{\V_1} \times 2 ^ {\V_2}$.

		      On définit l'automate $A' = \antuple{Q\times \set {0,1}, (q_0,0), F \times \set 1 , \delta'}$ où

		      \begin{eqnarray*}
			      \delta' &=& \setdef {((q,u),(a,S,T),(q',u))} {u \in \set {0,1}, x \notin S, (q,(a,S,T),q')\in \delta} \\
			      &\cup& \setdef {((q,0),(a,S\setminus \set x,T),(q',1))} { x \in S, (q,(a,S,T),q')\in \delta}
		      \end{eqnarray*}

		      %TODO: diagram ???

		      $A'$ est un automate sur l'alphabet $\alphabet \times 2^{\V_1\setminus \set x} \times 2 ^ {\V_2}$.

		      $(a_1, S_1, T_1), \ldots, (a_n, S_n, T_n) \in (\alphabet \times 2^{\V_1\setminus \set x} \times 2 ^ {\V_2})^*$ est accepté par $A'$
		      \ssi $\exists i \in \enum 1 n$ \tq le mot $w^* = (a_1, S_1, T_1), \ldots,  (a_i, S_i \cup  \set X, T_i),\ldots,(a_n, S_n, T_n)$ est accepté par $A$.
	\end{itemize}
\end{proofI}



\begin{theorem}
	Un langage est définissable dans $FO[<]$ ssi il existe une expression rationnelle sans étoile $r$ définissant un langage
	ssi
	le monoïde syntaxique du langage est apériodique (\cad il ne contient aucun sous-groupe non-trivial).
\end{theorem}

\subsubsection{Éléments idempotents}

\begin{definition}[Éléments idempotents]
	Soit $(M,\cdot , 1)$ un monoïde. Un élément $e \in M$ est \emph{idepmpotent} \ssi $e \cdot e = e$.
\end{definition}

\begin{exercice} \label{ex:puiss-idemp}
	Soit $e \in M$ est idempotent, alors $\forall i \in \N^*. \ e^i = e$.
\end{exercice}

\begin{lemma}
	Soit $(M,\cdot , 1_M)$ un monoïde fini et soit $m \in M$. Si $\abs M = n$ alors $m^{n!}$ est idempotent.
\end{lemma}

\begin{proof}
	Parmi les puissances de $m : m, \ldots , m^{n+1}$ il y  deux qui sont égales, car $\abs M = n$ .
	Soient $k \in \enum 1 n$ et $l \geq 1$ \tq $m ^{k} = m ^{k+l}$ ($k + l \in \enum 1 {n+1}$).
	$m ^{k} = m ^{k+l} \implies m ^{k+2l} = m ^{k+l} m^l = m^k m^l = m^{k+l} = m^k$

	Par induction on a que $\forall i \geq m^{k+il} = m^{k} \implies \forall i \geq 1, j \geq 1, m^{k+il+ j} = m^{k +j}$

	Soit $j = kl - k$ et $i =k$, alors on obtient
	\begin{eqnarray*}
		m ^{k + kl - k} = m^{kl} &\implies& m^{2kl} = m^{kl} \\
		&\iff& m^{kl}  m^{kl} = m^{kl} \\
		&\iff& m^{kl} \text{es un idempotent}
	\end{eqnarray*}

	Donc $\exists x \in \N, \, m^{n!} = m^{klx} = (m^{kl})^x = m^{kl} \implies m^{n!}$ est idempotent.
\end{proof}


\begin{definition}[Puissance idempotente]
	Une \emph{puissance idempotente} d'un élément $m \in M$ est un élément de la forme $m^i \ (i\geq 1)$ tel que $m^i$ est idempotent.
\end{definition}

\begin{lemma}\label{lem:puiss-idemp}
	Une puissance idempotente de $m \in M$ est unique, dans le sens suivant : si $m^i$ et $m^j$ sont des idempotents, alors $m^i = m^j$.
\end{lemma}

\begin{proof}
	$m^i$ idempotent $\implies (m^i)^j = m^i$, voire \ref{ex:puiss-idemp}.

	Pareil pour $m^j$ idempotent $\implies (m^j)^i = m^j$.
	Donc $m^i = m^{ij} = m^{ji} = m^{j}$
\end{proof}

\subsubsection{Relations de Green}

\begin{definition}
	Soit $(M,\cdot , 1_M)$ un monoïde. On défini les relations suivantes.

	\begin{itemize}
		\item \emph{prefixe, ou $\geqr$, ou ""R-préordre""}: On dit que $m$ est un préfixe de $n$ \ssi $\exists x \in M, \ n = mx$.
		      On le note $m \geqr n$.

		      Un ensemble de la forme $y \cdot M = \setdef {y \cdot z} {z \in M}$ pour $y \in M$ est appelé un idéal à droite de $M$.

		      $m$ est un préfixe de $n$ ssi $m \cdot M \supseteq n \cdot M$ (à vérifier en exercice).


		\item \emph{suffixe, ou $\geql$, ou ""L-préordre""}: On dit que $m$ est un suffixe de $n$ ssi $\exists y \in M, \ n = ym$.
		      On le note $m \geql n$.

		      TODO: exercice (same as before)


		\item \emph{infixe, ou $\geqj$, ou ""J-préordre""}: On dit que $m$ est un infixe de $n$ ssi $\exists x,y \in M, \ n = xmy$.
		      On le note $m \geqj n$.

		      TODO: exercice (same as before)
	\end{itemize}
\end{definition}

\begin{exercice}
	Montrer que $\geqr,\geqj,\geql$  sont transitives et réflexives et donc des préordres.
\end{exercice}

\begin{notation}
	On défini les relations d'équivalence associées à ces préordres L
	\begin{itemize}
		\item $m R n$ ssi $m \geqr n \et n \geqr m$ (On dit $m$ est ""$R$-équivalent""  $n$)

		      On note par $\classr m$ la $R$-classe d'équivalence de $m$, \ie, $\classr m =
			      \setdef {n \in M} {m R n}$

		\item $m L n$ ssi $m \geql n \et n \geql m$ (On dit $m$ est ""$L$-équivalent""  $n$)

		      On note par $\classl m$ la $L$-classe d'équivalence de $m$, \ie, $\classl m =
			      \setdef {n \in M} {m L n}$

		\item $m J n$ ssi $m \geqj n \et n \geqj m$ (On dit $m$ est ""$J$-équivalent""  $n$)

		      On note par $\classj m$ la $J$-classe d'équivalence de $m$, \ie, $\classj m =
			      \setdef {n \in M} {m J n}$
	\end{itemize}
\end{notation}

\begin{exercice}
	$\forall m \in M, \classr m \subseteq \classj m$
\end{exercice}

\begin{proof}
	Soit $m \in M$ et soit $x \in \classr m$, alors $x R m$ et donc
	$ x \geqr m$ et donc $\exists y $ tel que $m = xy$ et donc $m = 1xy$ et ainsi
	$x \geqj m$ et donc $x \in \classj m$.1
\end{proof}

\begin{lemma}[Eggbox lemma]
	Soit $M$ un monoïde fini et $m,n \in M$,

	\begin{enumerate}
		\item Si $m \geqr n \et m J n$ alors $n \geqr m$ et donc $m R n$
		\item Si $m \geql n \et m J n$ alors $n \geql m$ et donc $m L n$
	\end{enumerate}
\end{lemma}

\begin{proofI}
	\begin{enumerate}
		\item Soit $m,n \in M$ tels que $m \geqr n \et m J n$ alors
		      $$m \geqr n  \implies \exists x \in M, \ n = mx$$
		      $$m J n  \implies \exists y,z \in M, \ m = ynz$$
		      % TODO: make proper {
		      Tout cela implique $m = ymxz$

		      \begin{eqnarray*}
			      m = ymxz &\implies& m = y(ym xz)xz\\
			      &\implies& m = y^2m (xz)^2
		      \end{eqnarray*}


		      Par induction on peut montrer que
		      \begin{equation} \label{eq:eggbox}
			      \forall i \geq 1, m = y^i m (xz)^i
		      \end{equation}

		      Soit $i \geq 1$ tel que $(xz)^i$ est idempotent, alors
		      \begin{eqnarray*}
			      m &=& y^i m (xz)^i \reason{\ref{eq:eggbox}}\\
			      &=& y^i m (xz)^i (xz)^i \reason{$(xz)^i$ idempotent}\\
			      &=&  m (xz)^i \reason{\ref{eq:eggbox}}\\
			      &=&  m x (xz)^{i-1} z \reason {pour $i > 0$} \\
			      &=&  n (xz)^{i-1} z
		      \end{eqnarray*}

		      Donc $m = n (xz)^{i-1} z \implies$ $n$ est un préfixe de $m$ $\implies n \geqr m$.

		      On a alors que $m R n$.
		\item Exercice
	\end{enumerate}
\end{proofI}


\begin{definition}
	On dit $m \intro[$H$-classe]{H} n$ ssi $m R n $ et $m L n$. On a $H = R \cap L$.
\end{definition}

\begin{exercice}
	L'intersection d'une "$R$-classe" avec une "$L$-classe" d'équivalence est soit vide soit une "$H$-classe" d'équivalence.
\end{exercice}

\begin{proof}
	Si $\classr m \cap \classl n \neq \emptyset$, alors l'énoncé est vrai.

	Soit $s \in \classr m \cap \classl n$. On va montrer que $\classr m \cap \classl n = \classh s$

	\begin{itemize}
		\item Soit $s' \in \classr m \cap \classl n$. On sait $s' R s$ car $s' \in \classr m$ donc $s' R m R s$.
		      Pareil, $s' L s$ car  $s' L n L s$.
		      Et donc $s' H s$. Et donc $\classr m \cap \classl n \subseteq \classh s$

		\item \begin{eqnarray*}
			      s' \in \classh s &\implies& s R s' \et s L s'\\
			      &\implies& s' R m \et s' L n\\
			      &\implies& s' \in \classr m \cap \classl n
		      \end{eqnarray*}

	\end{itemize}
\end{proof}

\begin{lemma}
	Soit $\Hc$ une $H$-classe incluse dans une $J$-classe $\Jc$. Alors, soit
	\begin{enumerate}
		\item $\forall m,n \in \Hc, mn \notin \Jc$ \label{lem:not-group}
		\item $\Hc$ est un groupe (Attention !! L'élément neutre de $\Hc$ n'est pas forcément $1_M$)
	\end{enumerate}
\end{lemma}

\begin{proof}

	Supposons que $\exists m_0, n_0 \in \Hc$ tels que $m_0,n_0 \in \Jc$. On suppose donc la négation de \ref{lem:not-group}.
	Montrons que $\Hc$ est un groupe:
	\begin{itemize}
		\item Étape 1: Soit $m,n \in \Hc$, montrons que $mn \in \Jc$.

		\item Étape 2: Soit $m,n \in \Hc$, montrons que $mn \in \Hc$.

		      \begin{exercice}
			      Montrer que $mn L n$.
		      \end{exercice}

		      Et donc $mn \in \Hc$
		\item Étape 3: Soit $m \in \Hc$ on va montrer que $\Hc$ a un élément neutre et que ses éléments ont une inverse.
		      Par \ref{lem:puiss-idemp}, il existe $i > 1$ tel que $m^i$ est idempotent.
		      Par l'étape 2, on sait que $m^i \in \Hc$, donc on peut conclure que $\Hc$ contient un élément idempotent, notons le par
		      $e$. On va montrer que $e$ est l'élément neutre de $\Hc$, \cad, $\forall m \in \Hc, \ m\cdot e m  e \cdot m = m$.

		      Soit $n \in \Hc$, donc $n L e \implies \exists x, \ n = x\cdot e
			      \implies n e = (x e) e =  x (ee) =  x e = n$

		      De la même manière, $n \in \Hc \implies n R e \implies \exists y, \ n = ey \implies en = eey = ey = n$
		      Donc $e n = n$.

		      Il reste à montrer que chaque élément possède un inverse : $\forall n \in \Hc, \exists n' \in \Hc$ tel que $n n' = e = n'n$

		      On observe que $\Hc$ contient un unique idempotent. Soit $e'$ un autre idempotent, par le même argument
		      que ci-dessus, $e'$ est un élément neutre donc : TODO ... $e = 'e$

		      Soit $n \in \Hc$. Par le lemme \ref{lem:puiss-idemp}, il existe $j \geq 1$ tel que $n^j$ est idempotent, mais $n^j \in \Hc$ donc $n^j = e$. Et ainsi,
		      $n^{j-1}$ est l'inverse de $n$.
	\end{itemize}
\end{proof}

\begin{definition}
	Un monoïde $\monoide$ est apériodique ssi $\forall m \in M, \exists i \geq 1, \text { \tq } m^i = m^{i+1}$
\end{definition}


\begin{theorem}[Schützenberger]
	Soit $L\subseteq A^*$ un langage régulier. Les affirmations suivantes sont équivalentes :
	\begin{enumerate}
		\item $L$ est définissable par une expression rationnelle sans-étoile.
		\item $L$ est définissable dans la logique $FO[<]$.
		\item Le monoïde syntaxique de $L$ est apériodique.
	\end{enumerate}
\end{theorem}

TODO
