\section{Logique monadique du second ordre (MSO)}

Est-ce qu'il existe $w \in \mots$ \tq $w {\models} \exists x \exists y (x < y)$ soit vraie ?
($x,y$ sont interpretés comme des positions dans le mot $w$.)
Par exemple, $a {\nvDash} \exists x \exists y (x < y)$
Ici, cette proposition est satisfaite par $w \in \mots$ \ssi $\abs w \geq 2$.


Pour une proposition $\phi$, on a un langage $\La_\phi = \setdef {w \in \mots} {w \models \phi}$.


\begin{exemple}
	$\Sigma = \set{a,b}$
	$\phi_2 = \exists x \exists y (\forall z (z \geq x) \land Q_a x \land \forall z (z \leq y) \land Q_b y)$
	où $\leq, \geq$ sont interpretés comme les relations habituelles sur $N$.
	$Q_a x$ signifie qu'en position $x$, on retrouve la lettre $a$.
	$La_{\phi_2} = \setdef {w \in \kleene{\set {a,b}}} {w \models \phi_2} = a(a+b)*b$
\end{exemple}


%TODO: Comment and add line breaks
\begin{exemple}
	$$\phi_3 = \exists X (\forall x (\forall z (z \geq x) \ra X(x))) \land \forall x (\forall z (z \leq x) \ra \lnot X(x)) \land
		\forall x \forall y (((x < y) \land \forall z(z > x) \ra (z \geq y)) \ra (X (x) \lra \lnot X(y)))$$


	$X \rightsquigarrow $ un ensemble de positions dans un mot


	$X (x)\rightsquigarrow$ vrai \ssi $x \in X$


	$\La_{\phi_3} = \setdef {w \in \mots} {\abs w \equiv 0 (\text { mod} 2)}$
\end{exemple}

\subsection{Syntaxe}

\begin{definition} [La logique du premier ordre]
	\begin{itemize}
		\item $\Vi$ un ensemble de variables : $\set {x,y,z,x_1,y_1,z_1, \ldots}$
		\item \underline{Prédicats numériques} :  $\PP = \set {R_i^j, i > 0, j \geq o}$
		\item \underline{Formules atomiques} :
		      \begin{syntaxdef}
			      \syntaxHeader {\alpha} {Q_a x} {TODO}
			      \syntax {R_i^j(x_1, \ldots, x_j)} {Prédicats numériques}
		      \end{syntaxdef}
		\item \underline{Formules du premier ordre} :
		      \begin{syntaxdef}
			      \syntaxHeader {\phi} {\alpha} {Formules atomiques}
			      \syntax {\phi \land \phi} {Conjonction}
			      \syntax {\lnot \phi} {Négation}
			      \syntax {\exists x \phi} {Quantificateur existentiel}
		      \end{syntaxdef}
	\end{itemize}
\end{definition}


\begin{definition} [La logique du second ordre]
	C'est une extension de la logique du premier ordre :
	\begin{itemize}
		\item un ensemble de variables du second ordre : $X,Y,Z,X_1,Y_1,Z_1 \cdots$
		\item \underline{Formules atomiques} :
		      \begin{syntaxdef}
			      \syntaxExtension{\alpha}
			      \syntax {X(x)} {Appartenance a $X$}
		      \end{syntaxdef}
		\item \underline{Formules du second ordre} :
		      \begin{syntaxdef}
			      \syntaxExtension{\phi}
			      \syntax {\exists X \phi} {Quantificateur existentiel sur des ensembles}
		      \end{syntaxdef}
	\end{itemize}
\end{definition}


\subsection{Sémantique}

\subsubsection{Sémantique de la logique monadique du premier ordre}

Pour $n \in \N$ la longueur d'un mot, $x$ est interpreté comme un élément de $\set {1, \cdots, n}$

$$ \interpret {R_i^j}_n \subseteq \set {1, \cdots, n}^j$$
est une interpretation de chaque prédicat $R_i^j$ pour tout $n \geq 1$.
$$ \interpret {<}_n \subseteq \set {1, \cdots, n} \times \set {1, \cdots, n} = \setdef {(i,j)} {i<j}$$


\begin{definition}
	Une $\V$-structure, pour $\V \subseteq \Vi$ est un mot de la forme $(a_1, U_1) \ldots (a_n, U_n)$ où $a_1, \ldots, a_n \in \alphabet$ et
	$U_1, \ldots, U_n \subseteq \Vi$ tels que

	\begin{enumerate}
		\item $U_i \cap U_j = \emptyset, \forall i,j i \neq j$
		\item  $\bigcup_{i=1}^n U_i = \V$
	\end{enumerate}
\end{definition}

\begin{definition}[relation de satisfaction]
	On définit la relation de satisfaction $w \models \phi$ où $w$ est une $\V-structure$ et $\phi$ une formule de premier ordre \tq
	\begin{enumerate}
		\item $\forall x \in FV(\phi) \implies x \in \V$ (FV = variables libres)
		\item Les quantificateurs distincts dans $\phi$ lient des variables distinctes
	\end{enumerate}
\end{definition}

Si $\phi$ est un proposition, \cad, $FV(\phi) = \emptyset$ alors on a $\La_{\phi} = \setdef {w \text { les } \emptyset-\text{structures}} {w \models \phi} \subseteq \mots$.


Pour une interprétation $\interpret {R_i^j}_{{i,j}}$ de prédicats numériques, on définit la relation $w \models \phi$ par induction sur la structure de la formule $\phi$.

\begin{itemize}
	\item $w \models Q_a x$ ssi $w$ contient une lettre $(a,U_i)$ avec $x \in U_i$
	\item $w \models R_i^j(x_1, \ldots, x_j)$ ssi $\interpret {R_i}_{\abs w}(k_1, \ldots, k_j)$ est vrai où les $k1, \ldots, k_j$ sont les positions dans $w$ où les variables
	      $x_1, \cdots x_j$ apparaissent.
	\item $w \models \phi_1 \land \phi_2$ ssi $w \models \phi_1$ et $w \models \phi_2$
	\item $w \models \lnot \phi$ ssi $w \nvDash \phi$
	\item si $w = (a_1, U_1) \ldots (a_n, U_n)$ est une $\V$-structure.
	      $w \models \exists \phi$ ssi $\exists i \in \set {1, \ldots, n}$
	      $(a_1, U_1) \ldots  (a_i, U_i \cup \set x)  \ldots (a_n, U_n) \models \phi$
	      (il s'agit d'une $\V \cup {x}$-structure.)
\end{itemize}

\begin{exemple}
	$$abc \models_? \exists x \exists y (x < y)$$

	$(a, \set x)(b, \emptyset)(c, \emptyset) \models_? \exists y (x \leq y)$ est une $\set x $-structure

	$(a, \set x)(b, \set y)(c, \emptyset) \models_? (x < y) \iff (1,2) \in \interpret {<}_3 \iff 1 < 2$

	Version 2

	%TODO: petit dessin Photo
\end{exemple}


Soit $\phi$ et $\psi$ deux formules, $\phi \iff \psi$ \ssi $\La_{\phi} = \La_{\psi}$.

\begin{abbreviation}
	$\forall x \phi = \neg \exists x (\neg \phi)$
\end{abbreviation}

\begin{exemple}
	TODO: add it (important)
\end{exemple}


\subsubsection{Sémantique de la logique monadique du second ordre}


$V_1$ les variables du premier ordre

$V_2$ les variables du second ordre


\begin{definition}
	Une $(\V_1,\V_2)$-structure est un mot sur $\alphabet \times 2^{\V_1} \times  2^{\V_2}$, \cad un mot de la forme
	$(a_1,U_1,U_1) \cdots (a_n,U_n,V_n)$ où
	\begin{itemize}
		\item Les $U_i$ sont des ensembles de variables du premier ordre.
		\item Les $V_i$ sont des ensembles de variables du second ordre.
		\item $(a_1,U_1) \cdots (a_n,U_n)$ est une $\V_1$-structure.
	\end{itemize}
\end{definition}

\begin{definition}
	On définit la relation de satisfaction $w \models \phi$ où $w$ est une $(\V_1, \V_2)-structure$ et $\phi$ une formule de
	MSO avec les memes contraintes sur les variables que pour la logique du premier ordre %TODO: Add reference
	plus $w \models X(x)$ et $w \models \exists X \phi$
\end{definition}

L'induction est la même que pour le premier ordre, avec ces cas en plus :
\begin{itemize}
	\item $w \models X (x)$ ssi $w$ contient une lettre $(a,S,T)$ où $x \in S$ et $X \in T$
	\item $w \models \exists X \phi$ ssi $\exists J$ un ensemble de positions dans $w$ avec la proprieté:
	      la $(\V_1, \V2)$-structure $w'$ obtenue en remplaçant $(a_i, S_i,T_i)$ pour $i \in J$ par $(a_i, S_i, T_i \cup \set X)$ satisfait $\phi$.
\end{itemize}


\begin{exemple}
	TODO
\end{exemple}


\begin{definition}
	Un langage $L$ est définissable dans MSO[<] (la logique monadique du second ordre avec un prédicat binaire <).
	\ssi il existe une formule $\phi \in MSO[<]$ \tlq $L = \setdef {w\in \mots} {w \models \phi}$
\end{definition}

\begin{definition}
	Un langage $L \subseteq \mots$ est définissable dans $FO[<]$ (la logique du premier ordre avec <)
	\ssi $\exists \phi \in FO[<]$ \tlq $L = \setdef {w \in \mots} {w \models \phi}$
\end{definition}


\subsection{Relation avec les expressions rationnelles}

\begin{theorem}
	Un langage est définissable dans $MSO[<]$ ssi $L$ est régulier.
\end{theorem}


\begin{proof}
	%TODO: use \setminus instead of \
	\begin{itemize}
        \item \fbox{$\Leftarrow$}

		      Soit $A = \AFD$ un automate fini déterministe qui accepte un langage $L$.  On peut supposer que $L \subseteq \mots$ (
		      car $\motvide \in L$, on va prendre la disjonction de la formule obtenue pour $L \setminus \set {\motvide}$ avec $\forall x \neg (x = x)$)

		      Soit $w \in \mots$, Alors $w$ est reconnu par l'automate $A$ ssi $\exists X_0,\cdots,X_{k-1} \subseteq \set{1, \cdots , \abs w}$
		      tels que les propriéts suivantes soient vérifiés :
		      \begin{enumerate}
			      \item $\bigcup_{i=0}^{k - 1} X_i = \set{1, \cdots, \abs w}$
			      \item $\forall i < j, X_i \cap X_j = \emptyset$
			      \item $1 \in X_0$
			      \item $\forall j \in \set{1, \cdots, \abs w}$ si $j \in X_i \et j+i \in X_e$ et si $a$ est la lettre en position $j$ dans le mot
			            $w$, alors $\delta (q_i,a) = q_e$.

			      \item Si $\abs w \in X$ et $a$ est ;a dernière lettre de $w$, alors $\delta (q_j,a) \in F$.
		      \end{enumerate}

		      Supposons que $w = \decomp a n$ est accepté par $A$. On construit les ensembles $(X_i)_{i \in \enum 0 {k-1}}$
		      tel que $i \in X_j$ ssi après avoid lu les premiers $i-1$ lettres de $w$ on arrive a l'état $q_j$
		      %TODO: Diagram ? 

		      Les ensembles $X_0, \cdots, X_{k-1}$ satisfait les 5 propriétés.
		      \begin{enumerate}
			      \item $X_0 \cup \cdots \cup X_{k-1} = \enum i {\abs w}$. L'inclusion a gauche est vraie par définition. Montrons l'autre inclusion.
			            On considère l'unique chemin dans l'automate $A$ obtenu en lisant les premieres $i-1$ lettres de $w$ à partir de l'etat $q_0$.
			            Supposons qu'on arrive dans l'état $q_j$. Par définitions $u \in X_j$.
			      \item $\forall i < j, X_i \cap X_j = \emptyset$ est vrai car l'automate est déterministe.
			      \item $1 \in X_0$ car si on lit les premieres $0$ lettres de $w$ on reste a l'état $q_0$.
			      \item Après les premieres $j-1$ lettres on arrive a $q_i$. $a$ est la lettre en position $j$. Donc après les
			            premieres $j$ lettres on arrive dans l'état $\delta (q_i, a)$. Si après les premières $j$ lettres on arrive dans l'état $q_l$, alors
			            $\delta (q_i, a) = q_l$.
			      \item $w$ est accepté par $A$. Donc si $\abs w \in X_j$ alors après avoir lu les premieres $\abs w - 1$ lettres on arrive dans $q_j$.
			            Si $a$ est la dernière lettre de $w$, alors $\delta (q_j,a)$ est un tat acceptant car $w$ est accepté.
		      \end{enumerate}
		      On construit la formule
		      $$ \phi = \exists X_0 \cdots \exists X_{k-1} (\phi_1 \land \cdots \land \phi_5)$$
		      où
		      \begin{itemize}
			      \item $\phi_1 = \forall x (X_0(x) \lor \cdots \lor X_{k-1}) = \forall x \bigvee_{i=1}^{k-1} X_i(x)$
			      \item $\phi_2 = \forall x \bigwedge_{0\leq i < j \leq k-1}\lnot (X_i(x) \land X_j(x))$
			      \item $\phi_3 = \exists x \left((\forall y, x \leq y) \land X_0(x) \right)$
			      \item $\phi_4 = \forall x \left( \forall y (y = x + 1)  \ra \bigwedge_{0\leq i < l < k} ( (X_i (x) \land X_l(y) )\ra \bigvee_{S_l} Q_a(x))\right)$\\
			            Où $S_l = \setdef {a \in \alphabet} {\delta (q_i,a) = a_l}$
			      \item $\phi_5 = \forall x \left( \forall y (x \geq y)  \ra \bigwedge_{i = 0}^{k-1} ( X_i(x) \ra \bigvee_{T_i} Q_a(x)))\right)$\\
			            Où $T_i = \setdef {a \in \alphabet} {\delta (q_i,a) \in F}$

		      \end{itemize}

		\item \fbox{$\Rightarrow$}

		      On suppose que $L$ est défini par une formule de $MSO[<]$.
		      La preuve est par induction sur la structure de la formule.

		      $L \subseteq (\alphabet \times 2^{\V_1} \times 2 ^ {\V_2})$.

		      %TODO: Rappel ?
		      $\La$ esl l'ensemble de toutes les $(\V_1, \V_2)$-structures, pour $\V_1$ en ensemble de variables du premier
		      ordre et $\V_2$ un ensemble de variables du second ordre.

		      \begin{exercice}
			      Trouver un automate sur l'alphabet $\alphabet \times 2^{\V_1} \times 2 ^ {\V_2}$ qui accepte $\La$
		      \end{exercice}

		      \begin{exercice}
			      Trouver un automate qui accepte les $(\V_1, \V_2)$-structures \tq une variable de premier ordre $x$ apparait
			      dans une lettre de la forme $(a,S,T)$. Le but est de \mq le $\lang {Q_a(x)}$ est régulier.
		      \end{exercice}

		      \begin{exercice}
			      Montrer que $\La_{X(x)}$ est régulier.
		      \end{exercice}

		      \begin{exercice}
			      Montrer que $\La_{x< y}$ est régulier.
		      \end{exercice}



		      Soit $\phi = \exists x \psi$ et supposons que $L_{\psi}$ est régulier et donc accepté par un automate $A = \AFD$ sur l'alphabet
		      $\alphabet \times 2^{\V_1} \times 2 ^ {\V_2}$.

		      On définit l'automate $A' = \antuple{Q\times \set {0,1}, (q_0,0), F \times \set 1 , \delta'}$
		      où $\delta' = \setdef {((q,u),(a,S,T),(q',u))} {u \in \set {0,1}, x \notin S, (q,(a,S,T),q')\in \delta}
			      \cup
			      \setdef {((q,0),(a,S\ \set x,T),(q',1))} { x \in S, (q,(a,S,T),q')\in \delta}$
		      %TODO: diagram ???

		      $A'$ est un automate sur l'alphabet $\alphabet \times 2^{\V_1\setminus \set x} \times 2 ^ {\V_2}$.

		      %TODO: add final step
	\end{itemize}
\end{proof}



\begin{theorem}
	Un langage est définissable dans $FO[<]$ ssi il existe une expression rationnelle sans étoile $r$ définissant un langage
	ssi
	le monoïde syntaxique du langage est apériodique (\cad il ne contient aucun sous-groupe non-trivial).
\end{theorem}





