\section{Minimisation d'automates}

\subsection{Morphismes d'automates}

\begin{definition}[Morphisme d'automates]
	Soit $A = \antuple{Q, q_0, F, \delta}$ et $ A' = \antuple{Q', q_0', F', \delta'} $ sur un alphabet $\alphabet$.
	Alors un morphisme d'automates $\phi: A \to A'$ est une fonction $\phi: Q \to Q'$ \tlq
	\begin{enumerate}
		\item $\phi (q_0) = q_0'$ \label{morph:1}
		\item $\forall q \in Q,\quad  \phi (q) \in F' \iff q \in F$ \label{morph:2}
		\item $\forall q \in Q, \forall a \in \alphabet,\quad \delta'(\phi (q), a) = \phi (\delta(q,a))$ \ie $\phi \circ \delta_a = \delta_a '\circ \phi$ \label{morph:3}
	\end{enumerate}

	% https://tex.stackexchange.com/questions/218274/how-can-i-draw-commutative-diagrams-in-latex
	\[
		\begin{tikzcd}
			Q \arrow{r}{\delta_a} \arrow[swap]{d}{\phi} & Q \arrow{d}{\phi} \\
			Q' \arrow{r}{\delta_a} & Q'
		\end{tikzcd}
	\]

\end{definition}


\begin{exercice}
	Soit $\phi : A \to A'$ un morphisme d'automates déterministes, alors $\lang A = \lang {A'}$
\end{exercice}

\begin{proof}
	\begin{eqnarray*}
		w \in \lang A &\iff& \kleene \delta (q_0, w) \in F \\
		&\iff& \phi (\kleene \delta (q_0, w)) \in F' \reason{par \ref{morph:1}}\\
		&\iff& \kleene \delta( \phi (q_0), w) \in F' \reason{par \ref{morph:3}}\\
		&\iff& \kleene \delta( q_0', w) \in F' \reason{par \ref{morph:2}}\\
		&\iff& w \in \lang {A'}
	\end{eqnarray*}
\end{proof}


\begin{definition}[Quotient d'un automate]
	Soit $L \subseteq \mots$ et $w \in \mots$. Le quotient $\quot w L$ est défini par
	$$ \quot w L = \setdef {u \in \mots} {wu \in L} $$
\end{definition}

\begin{definition}
	Soit A = $\AFD$ sur l'alphabet $\alphabet$ et $q \in Q$ alors on définit
	$$ L_q = \setdef {w \in \mots} {\kleene \delta (q,w) \in F} $$
\end{definition}


\begin{lemma}
	Soit A = $\AFD$ un automate fini déterministe, $q \in Q$ et $w \in \mots$. Si $\deltaS {q_0} w = q$ alors
	$L_q = \quot w \lang A$
\end{lemma}

\begin{proof}
	\begin{eqnarray*}
		u \in L_q &\iff& \deltaS q u \in F \\
		&\iff& \deltaS {\deltaS {q_0} w} u \in F \\
		&\iff& \deltaS {q_0} {wu} \in F \\
		&\iff& wu \in \lang A \\
		&\iff& u \in \quot w \lang A
	\end{eqnarray*}
\end{proof}

\begin{coro}
	Si $L \subseteq \mots$ est un langage régulier alors l'ensemble $\setdef {\quot w L} {w \in \mots}$ est fini.
\end{coro}

\begin{proof}
	Si $L = \lang A$, avec A = $\AFD$, alors $\abs {\setdef {\quot w L} {w \in \mots}} \leq \abs Q$. Car $\forall w \in \mots,$ $\quot w L$ est
	le langage accepté par $A$ à partir de l'état $\deltaS {q_0} w$.
\end{proof}

\begin{coro} \label{coro:2}
	Si $\AFN$ accepte un langage $L$ alors $\abs Q \geq $ \# quotients de $L$
\end{coro}

\begin{definition} [Automate des quotients]

	Soit $L \subseteq \mots$ un langage régulier, soit $Q = \setdef {\quot w L} {w \in \mots}$ et $\AFD$. Alors l'automate des quotients de $L$
	est défini par

	\begin{itemize}
		\item L'état initial est $\quot {\motvide} L$
		\item $F = \setdef {\quot w L} {w \in L}$
		\item $\delta (\quot w L, a) = \quot {(wa)} L$
	\end{itemize}
\end{definition}

\begin{remarque}
	Si $\quot w L = \quot {(w')} L$, alors $\quot {(wa)}L = \quot {(w'a)} L$.

	Soit $w,w' \in \mots$ \tq $\quot w L = \quot {(w')} L$. Soit $a \in \alphabet$ et $u \in \mots$,
	\begin{eqnarray*}
		u \in \quot {(wa)}L &\iff& wau \in L \\
		&\iff& au \in \quot w L \\
		&\iff& au \in \quot {w'} L \\
		&\iff& w'au \in L \\
		&\iff& (w'a)u \in L \\
		&\iff& u \in \quot {(w'a)}L
	\end{eqnarray*}
	Donc la fonction de transition $\delta$ de l'automate des quotients est bien définie.
\end{remarque}

\begin{prop}
	Étant donné un automate $A$ qui accepte le langage $L$, l'automate des quotients accepte aussi $L$ et il est minimal parmi les automates acceptant $L$.
\end{prop}

\begin{proof}
	$\deltaS L w = \quot w L$ et donc $\underbrace{\deltaS L w \in F}_{w \in \lang A} \iff w \in L$

	Et ainsi, par le corollaire \ref{coro:2}, l'automate des quotients est minimal.
\end{proof}

\begin{lemma}\label{lem:reach}
	Soit $L \subseteq \mots$ un langage régulier. Soit $A_L$ l'automate des quotients de $L$ et soit $B$ un autre automate acceptant $L$.
	Soit $\reach B$ un sous-automate accessible de $B$. Alors on a un morphisme surjectif (un quotient) d'automates $\reach B \twoheadrightarrow A_L$.

	\begin{tikzcd}[row sep=large]
		&\reach B \arrow[dr, hook] \arrow[dl, twoheadrightarrow] \\
		A_L & & B
	\end{tikzcd}

	Si $B = <F,t_0,F,\delta>$ alors $\reach B =  <F',t_0,F',\delta'>$ où
	\begin{itemize}
		\item $F' = \setdef {t \in F} {\exists w \in \mots, \deltaS {t_0} w = t}$
		\item $F' = F \cap F'$
		\item $\delta ' = \delta \cap (F' \times \alphabet \times F')$
	\end{itemize}
\end{lemma}


\begin{exercice}
	On définit $\phi : \reach B \to A$.

	Si $t \in F', \exists w \in \mots$ \tq $ t = \deltaS {t_0} w$
	On définit $\phi (t) = \quot w L$

	Montrer que $\phi$ est bien définie, \ie, $\phi (t) = \quot {w'} L \implies \quot w L = \quot {w'} L$.
\end{exercice}

\begin{proof}
	Vérifions $\phi$ surjective.

	Si $w \in \mots$ et $\quot w L$ est un état de $A_L$ alors $\phi (\deltaS {t_0} w) = \quot w L$

	Si $\abs B = \abs {A_L}$, alors $B$ est un automate accesible (Sinon $\abs A \leq \abs {\reach B} < \abs B \lightning)$.

	Donc $B = \reach B$

	De plus, le morphisme $\phi$ est une bijection et $\quot {\phi}$ est un morphisme d'automates.
	Donc $B \cong A_L$.
\end{proof}


\subsection{Minimisation d'automates}

\begin{rappel}
	Pour tout langage rationnel $L$, il existe un unique (à isomorphisme près) automate déterministe minimal
	(avec le plus petit nombre possible d'états) qui reconnait $L$.
\end{rappel}

\begin{remarque}
	Ce n'est pas vrai pour les automates non-déterministes.


	\begin{twoautomata}
		\digraph[scale=0.5]{minex310}{
			rankdir=LR;

			node [shape=circle, style=filled, color=lightblue];
			q0 [label="q_0"];
			q1 [label="q_1"];
			q2 [label="q_1"];

			start [shape=point];

			start -> q0;
			q0 -> q1 [label="b"];
			q1 -> q1 [label="a"];
			q1 -> q2 [label="a"];
			q2 -> q1 [label="b"];
			q2 [shape=doublecircle];
		}
		\caption*{$(ba^+)^+$}
	\end{twoautomata}
	\begin{twoautomata}
		\digraph[scale=0.5]{minex311}{
			rankdir=LR;

			node [shape=circle, style=filled, color=lightblue];
			q0 [label="q_0"];
			q1 [label="q_1"];
			q2 [label="q_1"];

			start [shape=point];

			start -> q0;
			q0 -> q1 [label="b"];
			q1 -> q2 [label="a"];
			q2 -> q2 [label="a"];
			q2 -> q1 [label="b"];
			q2 [shape=doublecircle];
		}
		\caption*{$(ba^+)^+$}

	\end{twoautomata}
\end{remarque}

L'automate des résiduels de L est l'automate minimal reconnaissant L.


Nous allons voir trois algorithmes de minimisation différents:
\begin{itemize}
	\item Algorithme de Brzozowski
	\item Algorithme de Moore
	\item Algorithme de Hopcroft
\end{itemize}


\subsubsection{Algorithme de Brzozowski}

\begin{prop}
	Si $A$ est un automate déterministe et accesible, \ie, $\forall q \in Q, \exists \text{ un chemin } q_0 \to q$,
	et  $A^ {\sim} = \detA (\mirr(A)) $ est l'automate obtenu en déterminisant l'automate miroir de $A$. Alors
	$A^{\sim}$ est l'automate minimal qui reconnait $\mirror {\lang A}$.
\end{prop}


\begin{proof}
	Soit $A = (Q, q_0, F)$, $L = \lang A$. $\mirr (A) = (Q, F, q_0), M = \lang {\mirr (A)} = \mirror L$.

	On note $X \cdot u$ et $\delta (X,u)$, $X$ étant un ensemble d'états et $u$ un mot, l'ensemble $\bigcup\limits_{q \in X} \delta (q,u)$.

	Nous allons montrer que $A^{\sim}$ est l'automate minimal pour $M$. Pour cela, il suffit de montrer que $A^{\sim}$
	est isomorphe à l'automate des résiduels de $M$.

	Pour cela, il faut \mq si $u$ et $v$ sont deux mots quelconques \tq $\quot u M = \quot v M$, alors $F \cdot u =  F \cdot v$ dans $A^{\sim}$.

	Soit $p$ un état de $F \cdot u$. Puisque $A$ est accesible, il existe un chemin de $q_0 \to p$ dans $A$, et donc il existe un chemin $p \to q_0$ dans $\mirr(A)$.
	Soit $w$ l'étiquette de ce chemin, alors le mot $uw$ est l'étiquette d'un chemin réussi de $\mirr (A)$. En conséquence, $uw \in M$.

	Mais $uw\in M \iff w \in \quot u M \iff w\in \quot v M \iff vw \in M$.

	Donc $vw$ est l'étiquette d'un chemin réussi $\Gamma$ dans $\mirr (A)$. Étant donné que $A$ est déterministe, tout chemin aboutissant dans $q_0$
	et dont l'étiquette a $w$ comme suffixe doit passer par $p$.

	On en  déduit que $p \in Tv$.
\end{proof}


\begin{coro}
	Soit $A$ un automate, alors l'automate $\detA (\mirr (\detA(\mirr (A))))$ est l'automate minimal pour $\lang A$.
\end{coro}

\begin{proof}
	En effet,

	$\underbrace{\detA (\mirr (
			\underbrace{\detA(\mirr (A))}_{\text{Un automate déterministe et accesible qui reconnait }\mirror {\lang A}}
			))}_{\text{la proposition précédente garantit que cet automate est minimal pour }  \mirror{\mirror {\lang A} } = \lang A}
	$
\end{proof}

\begin{complexite}
	La complexité de l'algorithme est en $O(2^n)$. En effet, la dernière détermination exécutée travaille sur un automate de taille au plus $2^n$
	et produit un automate de taille $2^n$. Sa complexité reste aussi en $O(2^n)$.
\end{complexite}


\begin{remarque}
	Alors que d'autres algorithmes, \tq Moore, ont une complexité polynomiale, Brzozowski à différence des autres, ne nécessite pas
	que l'automate donné soit déterministe.
\end{remarque}

\subsubsection{Algorithme de Moore}


\begin{definition}[Congruence d'automates]
	Soit $A = (Q,q_0,F,S)$ un automate déterministe et soit $\sim$ une relation d'équivalence définie sur l'ensemble $Q$. On dit que
	$\sim$ est une congruence si $\sim$ satisfait les conditions suivantes :
	\begin{enumerate}
		\item Compatibilité aves les transitions: Si $q \sim q'$ alors $\forall a \in \alphabet, \delta (q,a) \sim \delta (q',a)$
		\item Saturation de $A$: Si $q \sim q'$ alors $q \in F \iff q' \in F$
	\end{enumerate}
\end{definition}


\begin{definition}
	Si $A = \AFD$ est un AFD et $\sim$ une congruence définie sur $Q$. On définit l'automate quotient :
	$$ A/\sim = (Q',q_0',F',\delta') $$

	avec \begin{itemize}
		\item $Q' = \setdef {[q]} {q\in Q}$ (chaque état est étiqueté par une classe d'équivalence)
		\item $q_0' = [q_0]$
		\item $F' =\setdef  {[q]} {q \in F}$
		\item $\delta'([q], a) = [p]$ \ssi $p \in [qa]$
	\end{itemize}

\end{definition}

\begin{prop}
	Si $A$ est un automate déterministe et $\sim$ une congruence sur $A$, alors $\lang A = \lang {A/\sim}$.
\end{prop}


\begin{definition}[Congruence de Nerode]
	Si $A = (Q, \set {q_0}, F, \delta)$ un AFD, $\forall q, q' \in Q$ on définit $q \cong q' \iff L_q = L_{q'}$.
\end{definition}

\begin{rappel}
	$L_q = \setdef w {\deltaS q w \in F}$.
	Donc
	\begin{eqnarray*}
		q \cong q' \iff L_q = L_{q'} &\iff& \forall w \in \mots, w \in L_q \mssi w \in L_{q'} \\
		&\iff& \forall w, \deltaS q w \in F \mssi  \deltaS {q'} w \in F
	\end{eqnarray*}

	On en déduit que
	\begin{equation}\label{eq:congnot}
		q \not\cong q' \iff \exists w, \deltaS q w \in F \et \deltaS {q'} w \notin F
	\end{equation}
	où $\deltaS q w \notin F$ et $\deltaS{q'} w \in F$ sont deux états non équivalents, dits \emph{séparables}.
\end{rappel}

Si $q$ et $q'$ sont séparables et $w$ est un mot qui satisfait \ref{eq:congnot}, on dira que $w$ \emph{sépare} $q$ et $q'$.


Pour calculer la congruence de Nérode on introduit une famille de congruences :

\begin{equation*}
	\cong_i, i  \in \N, q \cong_i q' \iff \forall w, \abs w \leq i \implies \deltaS q w \in F \iff  \deltaS {q'} w \in F
\end{equation*}


\begin{definition}
	Définition alternative

	\begin{itemize}
		\item $q \cong_0 q' \mssi q\in F \iff q' \in F$
		\item $q \cong_{i+1} q' \mssi q \cong_{i} q' \et  \forall a \in \alphabet, \delta (q,a) \cong_i \delta (q', a)$
	\end{itemize}
\end{definition}

\begin{remarque}
	Par définition $\cong_{i+1}$ induit une partition de $Q$ au moins aussi fine que celle induite par $\cong_i$.

	Les classes de $\cong_{i+1}$ sont obtenues en partitionnant des classes de $\cong_i$.
\end{remarque}

\begin{remarque}
	Puisque $Q$ est fini, le processus de partitionnement doit se stabiliser. Autrement dit,

	$$\exists k \in \N^*, \text{\tq} \cong_{k} \ = \ \cong_{k+j} \forall j \in \N^* $$
\end{remarque}


\begin{prop}
	Si $k \in \N$ \tq $ \cong_k \ = \ \cong_{k+j} \forall j \in \N$, alors $\cong_k \ = \ \cong$
\end{prop}

\begin{proof}
    %TODO: Should be feasible.
	Soit $q,q' \in Q$, $q \cong_k q'$, \mq $L_q = L_{q'}$. Exercice.
\end{proof}

\begin{definition}[Algorithme de Moore]
	L'algorithme consiste à créer un automate à partir des classes d'équivalence sur $\cong$.
	Les états sont les classes d'équivalence, on obtient les transitions en regardant le comportement
	d'un représentant de la classe et les états finaux sont les classes qui ont un représentant qui est un état final.
\end{definition}


\begin{complexite}
	L'algorithme effectue $n$ étapes et chaque étape dépense $O(n)$, donc sa complexité est en $O(n^2)$ (au pire).

	En fait, en moyenne $O(n \log n)$ et même $O(n \log \log n)$,  \cite{David2010TheAC}
\end{complexite}


