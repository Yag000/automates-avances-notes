\section{Monoïdes}

\begin{definition}[Monoïde]
	Un monoïde est un tuple $(M, \times, 1)$ où $M$ est un ensemble, $\times : M \times M \to M$ est une operation binaire sur $M$,
	$1\in M$ \tq
	\begin{enumerate}
		\item $\times$ est une opération  associative ($\forall x,y,z \in M , x \times (y \times z) = (x \times y) \times z$)
		\item $1$ est un élément neutre pour $\times$, \cad, $\forall x \in M, 1 \times x = x \times 1 = x$
	\end{enumerate}
\end{definition}

\begin{exemple}
    \ \newline
	\begin{itemize}
		\item $(\N, +, 0)$
		\item $(\mots, \cdot, \motvide)$
		\item $(\Z, +, 0)$ est un monoïde, mais aussi un groupe car tout élément possède un symétrique.
		\item Si $Q$ est un ensemble, alors $(Q ^ Q, \circ, \text{id})$
	\end{itemize}
\end{exemple}


\begin{definition}[Morphisme de monoïdes]
	Soit $(M,\times_M, 1_M)$ et $(N,\times_N, 1_N)$ des monoïdes. Une fonction $h :  M \to N$ est appelée un morphisme de monoïdes \ssi
	\begin{enumerate}
		\item $h(1_M) = 1_N \reason{$h$ préserve l'élément neutre}$
		\item $\forall x, y \in M, h (x \times_M y) = h(x) \times_N h(y) \reason{$h$ préserve la multiplication}$
	\end{enumerate}
\end{definition}


\begin{prop}
	Soit $\alphabet$ un ensemble et $(M,\cdot, 1)$ un monoïde. Soit $f : \alphabet \to M$ une fonction. Il existe
	un unique morphisme de monoïdes $\bar f : \mots \to M$ \tq $\bar f (a) =  f (a),  \forall a \in \alphabet$.

	\begin{tikzcd}
		\alphabet \arrow[r, hook]
		\arrow[d, "f", name=A]
		& \mots \arrow[dl, dashrightarrow, "\exists ! \bar f", name=B] \\
		M
	\end{tikzcd}
	%TODO: Add circular arrow
\end{prop}


\begin{proofI}
	\begin{itemize}
		\item Existence:

		      Soit $w\in \mots$ de la forme $\motDecomp w n$ où $w_i \in \alphabet$.

		      Soit $\bar f (w) =^{def} f(w_1) \cdot \ldots \cdot f(w_n)$ et $\bar f (\motvide) = 1_M$. Alors pour $a\in \alphabet, \bar f (a) = f (a)$.

		      Si $\motDecomp w n$ et $\motDecomp v m \in \mots$.
		      \begin{eqnarray*}
			      \bar f(wv) = \bar f(\decomp w n \decomp v m) &=& f(w_1) \cdot \ldots \cdot f(w_n) \cdot f(v_1) \cdot \ldots \cdot f(v_m) \\
			      &=& \left( f(w_1) \cdot \ldots \cdot f(w_n)\right)\cdot\left(f(v_1) \cdot \ldots \cdot f(v_m) \right)\\
			      &=& \bar f(\decomp w n) \cdot \bar f (\decomp v m) \\
			      &=& \bar f(w) \cdot \bar f (v)
		      \end{eqnarray*}

		      Donc $\forall w, v \in \mots, \bar f (wv) = \bar f (w)\bar f(v) \et \bar f (\motvide) = 1_M$ et ainsi, $\bar f$ est un morphisme de monoïdes.


		\item Unicité:

		      Si $h : \mots \to M$ est un morphisme de monoïdes \tq $h(a) = f(a), \forall a \in \mots$. Alors $h(\motvide) = 1_M$.

		      Comme $h$ préserve la multiplication, on a
		      $h (\decomp a n) = h(a_1) \ldots h(a_n) = f(a_1) \ldots f(a_n) = \bar f (\decomp a n)$.

		      Et donc $h = \bar f$
	\end{itemize}
\end{proofI}

On dit que $\mots$ est librement engendré sur $\alphabet$.

\begin{definition}[Librement engendré]
	$H$ est librement engendré sur $\alphabet$ si
	$$\forall f : \alphabet \to M \text{ où } M  \text { est un monoïde}, \exists ! \bar f : H \to M \et i : \alphabet \to H, \bar f \circ i   = f $$

	\begin{tikzcd}
		\alphabet \arrow[r, hook, "i"]
		\arrow[d, "f"]
		& H  \arrow[dl, dashrightarrow, "\exists ! \bar f"] \\
		M
	\end{tikzcd}

	Et on a donc que $ H \cong \mots$.
\end{definition}

\begin{exemple}
	Montrons que $(\N, + , 0)$ est librement engendré sur $\set 1$.

	\begin{tikzcd}
		\set 1 \arrow[r, hook]
		\arrow[d, "f"]
		&  (\N, +, 0) \arrow[dl, dashrightarrow, "\exists ! \bar f"] \\
		(M,\cdot, 1_M)
	\end{tikzcd}

	Alors on a que $f(1) = m \in M$ et donc 
    $$f(n) =  f(\underbrace{1+\ldots+1}_{n \text{ fois}}) = \underbrace{f(1)\cdot\ldots\cdot f(1)}_{n \text{ fois}} = \underbrace{m\cdot\ldots\cdot m}_{n \text{ fois}} = m^n$$
	Donc, si on pose $i: \set 1 \to (\N, +, 0)$ \tq $f(1) = 1$, alors quelque soit le choix de $m$, \ie, quelque soit la fonction $f$ choisie (car elle dépend juste du choix $m$), alors
	on peut toujours poser $\bar f (\N, +, 0)$ avec $f(0) = 1_M$ et $f(1) = m$ qui vérifie bien que $\bar f \circ i = f$. Ainsi, $(\N, +, 0)$ est librement engendré sur $\set 1$.
\end{exemple}

\begin{remarque}
	$\N \cong \set{1}^*$
\end{remarque}

\subsection{Reconnaissance par monoïdes}

\begin{definition}[Reconnaissance par monoïdes]
	Soit $\alphabet$ un alphabet fini. Soit $L \subseteq \mots$ un langage et soit $\phi : \mots \to M$ un morphisme de monoïdes où $(M, \circ, 1_M)$ est un monoïde fini. On dit que
	$\phi$ reconnait $L$ \ssi il existe $P \subseteq M$ \tq $L = \quot {\phi} (P)$, \cad $L = \setdef {w \in \mots} {\phi (w) \in P}$.
\end{definition}

\begin{exercice}
	$\phi$ reconnaît $L$ \ssi $L = \quot {\phi} (\phi (L))$.
\end{exercice}

\begin{proofI}
	\begin{itemize}
        \item \bimpRL

		      Soit $P = \phi(L) \subseteq M$, alors on a que $\quot {\phi} (P) = \quot {\phi} (\phi (L)) = L$ et donc $\phi$ reconnait $L$.

		\item \bimpLR

		      On sait qu'il existe $P$ \tq $L = \setdef {w \in \mots} {\phi(w) \in P}$. Regardons ce qu'est $\phi (L)$.
		      On a que $\phi (L) = \setdef {\phi (w)} {w \in \mots, \phi(w) \in P} = P$. Et donc on a bien $\quot {\phi} (\phi (L)) = \quot {\phi} (P) = L$.
	\end{itemize}
\end{proofI}

\begin{prop}
	Soit $L \subseteq \mots$ un langage sur l'alphabet $\alphabet$. Alors $L$ est reconnaissable \ssi $L$ est reconnu par un morphisme de monoïdes $\phi : \mots \to M$ où $M$ est fini.
\end{prop}


\begin{proofI}
	\begin{itemize}
		\item \bimpRL

		      Soit $\phi : \mots \to M$ un morphisme de monoïdes avec $M$ fini qui reconnait le langage $L$.
		      $\exists P \subseteq M, L = \quot {\phi} (P)$.

		      Soit $A$ l'automate suivant : $\antuple {M, 1_M, P, \delta}$ où

		      $$ \begin{array}{rcl}
				      \delta : M \times \alphabet & \to     & M                \\
				      (m,a)                       & \mapsto & m \cdot \phi (a)
			      \end{array} $$

		      On peut montrer par induction sur la longueur du mot que

		      $$ \begin{array}{rcl}
				      \delta ^* : M \times \mots & \to     & M        \\
				      (1_M, w)                   & \mapsto & \phi (w)
			      \end{array} $$


		      \begin{eqnarray*}
			      A \text{ accepte le mot } w &\iff&  \deltaS {1_M} w \in P \iff \phi (w) \in P \\
			      &\iff&  w \in \quot {\phi}(P) \iff w \in L.
		      \end{eqnarray*}

		      Donc $A$ accepte le langage $L$.


		\item \bimpLR

		      Soit $A = \AFD$ un automate déterministe complet qui accepte $L$.
		      Soit

		      $$ \begin{array}{rcl}
				      \phi : \mots & \to     & Q^Q        \reason{$(Q^Q, \circ,\text{id}_Q) $ est bien un monoïde} \\
				      w            & \mapsto & ( q \mapsto \deltaS q w )
			      \end{array} $$

		      On a bien :
		      \begin{itemize}
			      \item $Q^Q$ est un monoïde fini.
			      \item $\phi (\motvide) (q) = \deltaS q {\motvide} = q$, donc $\begin{array}{rcl}
					            \phi (\motvide) : Q & \to     & Q \\
					            q                   & \mapsto & q
				            \end{array} $
			            est la fonction identité sur $Q$ et donc $\phi$ préserve l'identité (élément neutre).
			      \item $\phi (w) \circ \phi (w') = \phi (ww'), \ \forall w, w' \in \mots$ car
			            $\deltaS {\deltaS q w} {w'} = \deltaS q {ww'}$.
			            Donc $\phi$ préserve la multiplication.


			            Soit $P \subseteq Q^Q$ défini par $P = \setdef {f : Q \to Q} {f(q_0)\in F}$.

			            \begin{eqnarray*}
				            \quot {\phi} (P) &=& \setdef {w \in \mots} {\phi (w) \in P} \\
				            &=& \setdef {w \in \mots} {\phi (w) (q_0)\in F} \\
				            &=& \setdef {w \in \mots} {\deltaS {q_0} w \in F}\\
				            &=& \setdef {w \in \mots} {w \text { est accepté par } A}\\
				            &=& L
			            \end{eqnarray*}
		      \end{itemize}

		      Donc $L$ est reconnu par $\phi$.
	\end{itemize}
\end{proofI}

\begin{definition}
	On va appeler $\phi (\mots)\subseteq Q^Q$ le \textbf{monoïde de transition de l'automate} $A$.
\end{definition}


\begin{definition}[Congruence]
	Soit $(M, \cdot, 1_M)$ un monoïde. Une congruence sur $M$ est une relation d'équivalence $\sim$ $\subseteq M \times M$ \tlq
	$$\forall m,m'\in M, \ m \sim m' \iff \forall n,r \in M, n\cdot m \cdot r \sim n \cdot m' \cdot r$$
\end{definition}

\begin{definition}[Congruence syntaxique d'un langage]
	Soit $L \subseteq \mots$ un langage. Soit $\sim_L \subseteq \mots \times \mots$ la relation d'équivalence définie par :
	$$ w \sim_L w' \mssi \forall u,v \in \mots, uwv \in L \iff u w' v \in L$$
\end{definition}

\begin{prop}
	$\sim_L$ est une congruence sur $\mots$.
\end{prop}

\begin{proof}
	Soit $m, m' \in \mots$ tels que $m \sim_L m'$, \cad $\forall u,v \in \mots, umv \in L \iff um'v \in L$.
	Soit $n \in \mots$, montrons que $nmr \sim_L nm'r$ \cad $\forall u, v \in \mots, u(nmr)v \in L \iff u(nm'r)v \in L$.
	Soit $u,v \in \mots$, posons $u' = un, v' = rv$, alors, on veut montrer que $u'mv' \in L \iff u'm'v' \in L$,
	ce qui est vrai car $m \sim_L m'$.
\end{proof}


\begin{prop}
	Si $\monoide$ est un monoïde et $\sim$ $\subseteq M \times M$ est une congruence sur $M$, alors $M/\sim = \setdef {[m]} {m \in M}$, où $[m]$ note la classe d'équivalence de $m$ par rapport à $\sim$.

	$M/\sim$ est un monoïde et $ \begin{array}{rcl}
			h : M & \to     & M/\sim \\
			m     & \mapsto & [m]
		\end{array} $ est un morphisme de monoïdes.
\end{prop}

\begin{proof}
	$(M/\sim, \cdot, [1_M])$ est un monoïde où $[m][m'] = [mm']$.

	À montrer que cette multiplication est bien définie, \cad si $m_1 \sim m_2 \et m_1' \sim m_2'$ alors $m_1 m_1' \sim m_2 m_2'$ :
	\begin{eqnarray*}
		m_1 &\sim& m_2 \\
		m_1' &\sim& m_2' \\
		1_M \cdot  m_1 \cdot m_1' &\sim& 1_M \cdot m_2 \cdot m_1' \reason{car $\sim$ est une congruence} \\
		m_1 \cdot m_1' &\sim&  m_2 \cdot m_1' \\
		\text{Mais } m_2 \cdot m_1' &\sim& m_2 \cdot m_2' \reason{car $\sim$ est une congruence} \\
		\implies m_1 \cdot m_1' &\sim& m_2 \cdot m_2'
	\end{eqnarray*}

	Donc la multiplication sur $M/\sim$ est bien définie.


	$[1_M]$ est un élément neutre car $[m]\cdot [1_M] = [m \cdot 1_M] = [m]$ et pareil pour $[1_M]\cdot [m] = [m]$.

	$h$ est un morphisme de monoïdes:

	\begin{itemize}
		\item  $h(1_M) = [1_M]$
		\item  $h(m\cdot n) = [m \cdot n] = [m] \cdot [n] = h(m) \cdot h(n)$
	\end{itemize}
\end{proof}

\begin{remarque}
	$h$ est un \textbf{morphisme surjectif} (aussi appelé un quotient).
\end{remarque}

\begin{definition}
	Soit $L \subseteq \mots$ un langage et soit $\sim_L$ la congruence syntaxique sur $\mots$.
	Le monoïde quotient $\mots / \sim_L$ est appelé le monoïde syntaxique de $L$.
\end{definition}

\begin{prop}
	Avec les notations ci-dessus, $L$ est reconnu par le morphisme
	$ \begin{array}{rcl}
			\phi : \mots & \to     & \mots/\sim_L \\
			w            & \mapsto & [w]
		\end{array} $.
\end{prop}


\begin{remarque}\label{rem:quot-mono}
	Si $w \in L$ et $w' \sim_L w$ alors $w' \in L$, parce que  $w' \sim_L w$, donc $\motvide w \motvide \in L \implies \motvide w' \motvide \implies w' \in L$.

\end{remarque}

\begin{proof}
	Soit $P \subseteq \mots / \sim_L$ donné par $P = \setdef {[w]} {w \in L}$.

	On doit \mq $L = \quot {\phi} (P)$, \cad, $L = \setdef {w \in \mots} {\phi (w) \in P}$

	En utilisant la remarque \ref{rem:quot-mono}, on a que si $ w\in L$ alors $[w] \subseteq L$.

	On en conclut que
	$$ L = \bigcup_{w \in L} [w]    \iff L = \quot {\phi} (P) $$
\end{proof}


\begin{prop}
	Le monoïde syntaxique a une propriété similaire à \ref{lem:reach}:
	$\forall \monoide$ monoïde qui reconnait un langage $L$, on a le diagramme suivant:

	\begin{tikzcd}[row sep=large]
		&(N, \cdot , 1_N) \arrow[dr, hook] \arrow[dl, twoheadrightarrow] \\
		(\mots / \sim_L, \cdot,  [\motvide]) & & (M, \cdot, 1_M)
	\end{tikzcd}
\end{prop}

\begin{terminologie}
	On dit que le monoïde syntaxique de $L$ divise tout monoïde qui accepte $L$.
\end{terminologie}


\begin{lemma}
	Si $N$ divise $M$ et $N$ reconnait un langage $L$ alors $M$ reconnait $L$.

\end{lemma}

\begin{proof}

	\begin{tikzcd}[row sep=large]
		&T \arrow[dr, hook] \arrow[dl, twoheadrightarrow] \\
		N & & M
	\end{tikzcd}
	$\phi : \mots \to N $ et $P \subseteq N$ \tq $L = \quot {\phi} (P)$

	Exercice : \mq on peut construire $\phi'' : \mots \to M$ qui reconnait $L$.
\end{proof}


\begin{prop}
	Un monoïde  $M$ reconnait  un langage $L$ \ssi le monoïde syntaxique $\mots / \sim_L$ divise $M$.
\end{prop}

\begin{proofI}
	\begin{itemize}
		\item \bimpRL

		      $\mots / \sim_L$ reconnait $L$. Par le lemme, si $\mots / \sim_L$ divise $M$, alors $M$ reconnait $L$.

		\item \bimpLR

		      Supposons que $M$ reconnait $L$. Donc on a $\phi : \mots \to M$ et $P \subseteq M$ avec $\quot {\phi} (P) = L$.

		      Soit $N = \phi (\mots)$ l'image directe de $\mots$ par $\phi$.

		      Alors $N \hookrightarrow M$ est un sous-monoïde de $M$.

		      Je voudrais avoir un quotient $h : N \twoheadrightarrow \mots/\sim$.

		      Si $n \in N$ alors $\exists w \in \mots$ \tq $\phi(w) = n$. On définit $h(n) = [w]$.

		      Il faut \mq :
		      \begin{enumerate}
			      \item $h$ est bien défini (si $\phi (w') = \phi (w) \implies [w] = [w']$). \label{prop:recon1}
			      \item $h$ est un morphisme de monoïdes.
			      \item $h$ est surjectif.
		      \end{enumerate}

		      Montrons cela:
		      \begin{itemize}
			      \item \ref{prop:recon1}: Supposons que $\phi(w) = \phi (w')$.

			            On veut \mq $[w]=[w']$ \cad $w \sim_L w'$, \cad $\forall u,v \in \mots, uwv \in L \iff uw'v \in L$.

			            Soient $u,v \in \mots$.
			            \begin{eqnarray*}
				            uwv \in L &\iff& \phi (uwv) \in P\\
				            &\iff& \phi (u)\phi(w) \phi(v) \in P \reason{car $\phi$ est un morphisme} \\
				            &\iff& \phi (u)\phi(w') \phi(v) \in P \reason{car $\phi (w) = \phi (w')$ } \\
				            &\iff& \phi (uw'v) \in P\\
				            &\iff& uw'v \in L
			            \end{eqnarray*}

			            Donc $w \sim_L w'$.
		      \end{itemize}
		      Exercice : finir la preuve.
	\end{itemize}
\end{proofI}
