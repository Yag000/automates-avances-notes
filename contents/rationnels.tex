
\section{Langages Rationnels}


\digraph[scale=0.5]{langage}{
	rankdir=TB;

	Node0 [shape=ellipse, style=filled, color=lightblue, label=<Langages<br/>rationnels>];
	Node1 [shape=ellipse, style=filled, color=lightgreen, label=<Automates finis<br/>déterministes>];
	Node2 [shape=ellipse, style=filled, color=lightgreen, label=<Automates finis<br/>non déterministes<br/>+ ε-transitions>];
	Node3 [shape=ellipse, style=filled, color=lightgreen, label=<Expressions<br/>rationnelles>];
	Node4 [shape=ellipse, style=filled, color=lightgreen, label=<Logique monadique<br/>du second ordre>];
	Node5 [shape=ellipse, style=filled, color=lightgreen, label=<Théorie<br/>des monoïdes>];

	Node0 -> Node1;
	Node0 -> Node2;
	Node0 -> Node3;
	Node0 -> Node4;
	Node0 -> Node5;
}

\subsection{Définitions de base}

\begin{definition}[alphabet]
	Un \textbf{alphabet} est un ensemble fini de lettres ou de symboles.
\end{definition}

\begin{definition}[mot]
	Un \textbf{mot} sur un alphabet $\alphabet$ est une séquence de lettres de $\alphabet$.
	On écrit $\motDecomp w n$ où $w_i \in \alphabet, \forall i \in \{1, \ldots, n\}$.
	La longueur d'un mot $w$ est notée $\len w = n$, pour $\motDecomp w n$.
\end{definition}

\begin{notation}
	On note $\motvide$ le mot vide et $\mots$ l'ensemble des mots sur $\alphabet$.
\end{notation}

\begin{definition}[concaténation]
	La \textbf{concaténation} de deux mots $w$ et $v$ est notée $wv$.
	Si $\motDecomp w n$ et $\motDecomp v m$, alors $wv = \decomp w n \decomp v m$.
\end{definition}

\begin{definition}[langage]
	Un langage sur un alphabet $\alphabet$ est un sous-ensemble de $\mots$.
\end{definition}

\begin{definition}[concaténation de langages]
	Soient $L_1, L_2$ deux langages sur $\alphabet$, leur \textbf{concaténation} est le langage
	\[
		L_1L_2 = \setdef {w_1w_2} {w_1 \in L_1, w_2 \in L_2}
	\]
\end{definition}

\begin{exemple}
	$$ \alphabet = \set{a, b}, \quad L = \set{\red a, \red {ab}, \red{bb}}, \quad K = \set{\blue b, \blue {ab}} $$
	$$ LK = \set{\red a \blue b, \red {ab} \blue b, \red {bb} \blue b, \red a \blue{ab}, \red {ab} \blue {ab}, \red {bb} \blue {ab}} $$
\end{exemple}


\begin{definition}[étoile de Kleene]
	Soit $L$ un langage sur $\alphabet$, son \textbf{étoile de Kleene} est le langage
	$$
		\kleene L = \bigcup_{n \in \N} L^n
	$$
	où $L^0 = \set\motvide$ et $L^{n+1} = LL^n$.
\end{definition}

\begin{remarque}
	Est-ce que $\kleene L = \setdef {w^n} {w \in L, n \in \N}$ ?

	Non, on peut trouver un contre-exemple avec $L = \set{a, b}$.

	On a bien que $ab \in \kleene L$, mais $ab \notin \setdef {w^n} {w \in L, n \in \N} = \setdef {a^n} {n \in \N} \cup \setdef {b^n} {n \in \N}$.
\end{remarque}

\begin{remarque}
	Est-ce que $L(M \cap N) = LM \cap LN$ ?

	Non, on peut trouver un contre-exemple avec $L = \set{a, ab}$, $M = \set{b}$ et $N = \set{\motvide}$.

	On a que $M \cap N = \emptyset$, donc $L(M \cap N) = \emptyset$.

	Mais $LM = \set{ab, abb}$ et $LN = \set{a, ab}$, donc $LM \cap LN = \set{ab} \neq \emptyset$.
\end{remarque}

\begin{exercice}
	Montrer que la concaténation de langages est distributive par rapport à l'union, \ie que $L(M \cup N) = LM \cup LN, \forall L, M, N$ langages.
\end{exercice}

\subsection{Langages rationnels : définitions}

\begin{definition}[langage rationnel]
	Soit $\alphabet$ un alphabet fini, l'ensemble \eratsym \ des expressions rationnelles sur $\alphabet$ est défini comme suit :
	\begin{itemize}
		\item $\motvide \in \eratsym$
		\item $\emptyset \in \eratsym$
		\item $\forall a \in \alphabet, a \in \eratsym$
		\item $\forall E, F \in \eratsym, E + F \in \eratsym$
		\item $\forall E, F \in \eratsym, EF \in \eratsym$
		\item $\forall E \in \eratsym, \kleene E \in \eratsym$
	\end{itemize}
\end{definition}

\begin{definition}[sémantique des expressions rationnelles]
	Soit $r \in \erat$, on définit le langage $\lang r$ associé à $r$ par induction sur la structure de $r$ :
	\begin{itemize}
		\item $\lang \motvide = \set\motvide$
		\item $\lang \emptyset = \emptyset$
		\item $\lang a = \set a$
		\item $\lang {E + F} = \lang E \cup \lang F$
		\item $\lang {EF} = \lang E \lang F$
		\item $\lang {\kleene E} = \kleene {\lang E}$
	\end{itemize}
\end{definition}

\begin{exemple}
	$ \alphabet = \set{a, b}, \quad r = \kleene {(a + b)} a \in \erat $, alors
	$ \lang r = \setdef {wa} {w \in \mots} $
\end{exemple}


\begin{definition}[langage rationnel]
	Un langage $L$ sur un alphabet $\alphabet$ est dit \textbf{rationnel} s'il existe une \exprat $r \in \erat$ \tlq $L = \lang r$.

\end{definition}

\begin{exemple}
	$\setdef {a^n} {n \in \N}$ est un langage rationnel engendré par l'expression rationnelle $\kleene a$.

	Cependant, $\setdef {a^n b^n} {n \in \N}$ n'est pas un langage rationnel.
\end{exemple}
