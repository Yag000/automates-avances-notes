\section{String matching}


\begin{definition}
	Étant donné deux chaines de caracteres $T$ le texte et $M$ le motif.
	On dit que $T$ présente une occurrence ce $M$ si
	$$\forall j \in \enum 1 {\len M -1}, \exists i, 0 \leq 1 \leq \len T - \len M, \text{ \tq } T[i + j] = M[j]$$
\end{definition}

\begin{remarque}
	Le texte $T$ contient le facteur $u$ \ssi $T$ a un préfixe qui appartient $\mots u$.
\end{remarque}

\begin{remarque}
	L'algorithme trivial qui teste toutes les positions de $T$ et vérifie s'il y a une occurrence de $u$ en position $i$ est en $O(\abs T \abs u)$.
\end{remarque}

\begin{remarque}
	Pour chercher les occurrences de $u$ dans $T$ on peut construire un automate qui reconnait $\mots u$ et le faire travailler sur $T$.
	À chaque fois qu'on passe par un état finale, on vient de voir une occurrence de $M$.
\end{remarque}

\begin{remarque}
	Avec l'algorithme naif, chaque caractère de $T$ est analysé $\abs u$ fois. Avec les algorithmes pas les automates chaque caractère est analysé une seule fois.
\end{remarque}

\begin{remarque}
	L'approche pas automates présente quelques problèmes d'efficacité :
	\begin{itemize}
		\item Si l'automate est non déterministe, le temps de calcul peut devenir tres lourd.
		\item Si on le déterministe avant, also on risque de trouver un algorithme de taille exponentielle.
	\end{itemize}
\end{remarque}

\subsection{Knuth-Morris-Pratt}

L'algorithme Knuth-Morris-Pratt améliore l'algorithme naif en introduisant des décalage d'amplitude
$> 1$ et fait en sorte que chaque caractère de $T$ soit analysé une seule fois.


\begin{definition}
	Si $w$ est un mot, on appelle bord de $w$ le mot le plus long qui est en même temps préfixe et suffixe de $w$.
\end{definition}

Si au cours de la vérification de presence du mot $u$ à partir d'une position
on a un échec au niveau de la $j$-ème lettre du motif, alors :

\begin{itemize}
	\item On calcule le bord $a$ di préfixe de $u$ de longueur $j-1$.
	\item Et on décale $u$ de sorte que le $e$ au début est position;e sous e $q$ final.
\end{itemize}

\begin{exemple}
	TODO: explain
	\begin{eqnarray*}
		T &=&  ABABAABCBABABACAB \\
		u &=& ABABACA
	\end{eqnarray*}
\end{exemple}
