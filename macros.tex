
%% Code from base-yago : https://github.com/Yag000/base-yago

%% This is my base style for my latex documents in french.

%% Color

\usepackage{color}
\usepackage{xcolor}

\definecolor{red}{rgb}{1,0,0}
\definecolor{green}{rgb}{0,1,0}
\definecolor{darkgreen}{rgb}{0,0.5,0}
\definecolor{blue}{rgb}{0,0,1}
\definecolor{violet}{rgb}{0.4,0,0.8}

\newcommand{\red}[1]{\textcolor{red}{#1}}
\newcommand{\green}[1]{\textcolor{green}{#1}}
\newcommand{\darkgreen}[1]{\textcolor{darkgreen}{#1}}
\newcommand{\blue}[1]{\textcolor{blue}{#1}}
\newcommand{\violet}[1]{\textcolor{violet}{#1}}


\hypersetup{
	colorlinks,
	linkcolor={},
	citecolor={blue!50!black},
	urlcolor={blue!80!black}
}

%% Basic algebra

\newcommand{\N}{\mathbb{N}}

\newcommand{\Z}{\mathbb{Z}}
\newcommand{\Zn}[1]{\mathbb{Z}/#1\mathbb{Z}}
\newcommand{\F}{\mathbb{F}}
\newcommand{\Q}{\mathbb{Q}}

\newcommand{\R}{\mathbb{R}}
\newcommand{\Rn}{\R^n}

\newcommand{\C}{\mathbb{C}}

\newcommand{\K}{\mathbb{K}}

\newcommand{\pgcd}[2]{\text{pgcd}(#1,#2)}

\newcommand{\goatp}{\mathfrak{p}}
\newcommand{\goatm}{\mathfrak{m}}

\newcommand{\car}{\text{car}}

\newcommand{\degExt}[2]{[ #1 : #2 ]}

\newcommand{\id}{\text{id}}


%% Polynomials

\newcommand{\QX}{\mathbb{Q}[X]}
\newcommand{\RX}{\mathbb{R}[X]}
\newcommand{\CX}{\mathbb{C}[X]}

\newcommand{\AX}{A[X]}

\newcommand{\Pt}{\tilde{P}}
\newcommand{\Qt}{\tilde{Q}}


%% Vector spaces

\newcommand{\sprod}[2]{\left\langle #1, #2 \right\rangle}

\newcommand{\LE}{\mathcal{L}(E)}

\newcommand{\spec}{\text{Spec}}

\newcommand{\abs}[1]{\left|#1\right|}
\newcommand{\norm}[1]{\abs{\abs{#1}}}
\newcommand{\normop}[1]{\abs{\abs{\abs{#1}}}}

\newcommand{\dist}[2]{\text{ d}\left(#1, #2\right)}
\newcommand{\vect}[1]{\text{Vect}\left( #1 \right)}

\newcommand{\Mn}[2]{\mathcal{M}_{#1}(#2)}

%% Probability notation

\newcommand{\Pro}{\mathscr{P}}
\newcommand{\E}{\mathbb{E}}

\newcommand{\1}{\mathds{1}}


%% Functions

\newcommand{\restr}[2]{#1_{\restriction #2}}
\newcommand{\im}[1]{\text{Im}\left(#1\right)}

%% Logic

\newcommand{\set}[1]{\left\{#1\right\}}
\newcommand{\setdef}[2]{\set{#1 \mid #2}}
\newcommand{\parts}[1]{\mathcal{P}\left(#1\right)}


\newcommand{\contradict}{\lightning}

%% Abbreviations

\newcommand{\ssi}{si et seulement si }
\newcommand{\mssi}{\text{ si et seulement si }}
\newcommand{\tlq}{telle que }
\newcommand{\tq}{tel que }
\newcommand{\et}{\text{ et }}
\newcommand{\ou}{\text{ ou }}
\newcommand{\cad}{c'est-à-dire }
\newcommand{\mq}{montrer que }
\newcommand{\mqs}{montrons que }
\newcommand{\ie}{\textit{i.e.} }
\newcommand{\eg}{\textit{e.g.} }


%% Environments

\theoremstyle{plain}
\newtheorem{theorem}{Théorème}[section]
\newtheorem{coro}[theorem]{Corollaire}
\newtheorem{lemma}[theorem]{Lemme}
\newtheorem{prop}[theorem]{Proposition}

\theoremstyle{definition}
\newtheorem{definition}[theorem]{Définition}
\newtheorem{notation}[theorem]{Notation}
\newtheorem{exemple}{Exemple}[subsection]
\newtheorem{exercice}{Exercice}[subsection]

\theoremstyle{plain}
\newtheorem{remarque}{Remarque}[subsection]
\newtheorem{rappel}{Rappel}[subsection]

%% Title

\usepackage{ifthen}

\newcommand{\titlepageY}[3]{%
	\fancypagestyle{toc}{%
		\fancyhf{}%
		\fancyhead[L]{\rightmark}%
		\fancyhead[R]{\thepage}%
	}

	\pagestyle{toc}

	\begin{titlepage}
		\newcommand{\HRule}{\rule{\linewidth}{0.5mm}}
		\center

		\HRule\\[0.4cm]

		\textsc{\Large #1}\\[0.5cm]
		\ifthenelse{ \equal{#3}{} }
		{ }
		{\textsc{\large #3}\\[0.5cm] }

		\HRule\\[1.5cm]


		{\large\textit{Auteur}}\\
		#2


		\vfill\vfill\vfill

		{\large\today}

		\vfill

	\end{titlepage}
}



%% Basic automata theory

%%%% Basic definitions

\newcommand{\alphabet}{\Sigma}

\newcommand{\mot}{w}

\newcommand{\decomp}[2] {
	#1_1 #1_2 \ldots #1_{#2}
}

\newcommand{\motDecomp}[2] {
	#1 = \decomp{#1}{#2}
}

\newcommand{\motvide}{\varepsilon}
\newcommand{\len}[1]{|#1|}


\newcommand{\kleenestar}{*}
\newcommand{\kleene}[1]{#1^{\kleenestar}}
\newcommand{\mots}{\kleene{\alphabet}}

%%%% Languages

\newcommand{\La}{\mathcal{L}}
\newcommand{\lang}[1]{\La\left(#1\right)}
\newcommand{\langstar}{\kleene{\lang}}



\newcommand{\V}{\mathcal{V}}
\newcommand{\PP}{\mathcal{P}}


\newcommand{\B}{\mathbb{B}}


\newcommand{\Vi}{\mathbf{V}}

\newcommand{\quot}[1]{#1^{-1}}

%%%% Regex

\newcommand{\eratsym}{\kl {ERat}}
\newcommand{\erat}{\text{\eratsym}}

\newcommand{\exprat}{\kl{expression rationnelle} }

%%%% Automata

\newcommand{\antuple}[1]{\langle #1 \rangle}
\newcommand{\AFD}{\antuple{Q, q_0, F, \delta}}
\newcommand{\AFN}{\antuple{Q, I, F, \delta}}
\newcommand{\mirror}[1]{\widetilde{#1}}

\knowledgenewrobustcmd \reach [1] {\cmdkl{\text{reach}} \left(#1\right)}


\newcommand{\deltaS}[2]{\kleene \delta \left(#1,#2\right)}

\newcommand{\detA}{\text{\kl{det}}}
\newcommand{\mirr}{\text{\kl{mirr}}}

%% Environments

\newcommand{\reason}[1]{\quad\left(\textit{#1}\right)}

\newenvironment{automata} { } { }

\newenvironment{twoautomata} { \begin{automata} \begin{minipage}{0.5\textwidth} \begin{figure}[H] }
				{ \end{figure}  \end{minipage} \end{automata}}




\renewcommand{\phi}{\varphi}
\renewcommand{\epsilon}{\varepsilon}

\newcommand{\monoide}{(M, \cdot, 1_M)}


\theoremstyle{plain}
\newtheorem*{complexite}{Complexité}


\theoremstyle{plain}
\newtheorem*{idee}{Idée}

\newtheorem{terminologie}{Terminologie}[section]
\newtheorem{abbreviation}{Abbréviation}[section]
\newtheorem{construction}{Construction}[section]
\newtheorem{algorithme}{Algorithme}[section]

\newcommand{\ra}{\rightarrow}
\newcommand{\lra}{\leftrightarrow}


\newcommand{\interpret}[1]{\kl[interpret]{\llbracket} #1 \kl[interpret]{\rrbracket}}

%% Syntax


\newcommand{\syntaxHeader}[3]{#1 &::=& #2 &\quad\text{#3}\\}
\newcommand{\syntaxExtension}[1]{\syntaxHeader{#1}{\cdots}{}}
\newcommand{\syntax}[2]{&|&#1&\quad\text{#2}\\}

\newenvironment{syntaxdef} {\begin{mathpar} \begin{array}{lcll}} {\end{array} \end{mathpar}}


\newcommand{\enum}[2]{\set{#1,\ldots,#2}}

% TODO: change to geq
\newcommand{\geqr}{\kl[Rprefixe]{>_R}\, }
\newcommand{\geql}{\kl[Lsuffix]{>_L} \, }
\newcommand{\geqj}{\kl[Jinfix]>{_J}\, }


\newcommand{\classr}[1]{\kl[classR]{[} #1 \kl[classR]{]_R}}
\newcommand{\classl}[1]{\kl[classL]{[} #1 \kl[classL]{]_L}}
\newcommand{\classj}[1]{\kl[classJ]{[} #1 \kl[classJ]{]_J}}
\newcommand{\classh}[1]{\kl[classH]{[} #1 \kl[classH]{]_H}}


\newcommand{\Hc}{\mathbb{H}}
\newcommand{\Jc}{\mathbb{J}}

\newcommand{\bimpRL}{\fbox{$\Leftarrow$}}
\newcommand{\bimpLR}{\fbox{$\Rightarrow$}}

\newcommand{\row}[1]{\text{\kl {row}}\left(#1\right)}

\newcommand{\low}[1]{\text{low}\left(#1\right)}
\newcommand{\high}[1]{\text{high}\left(#1\right)}
\newcommand{\etiquette}[1]{\text{etiquette}\left(#1\right)}

\newenvironment{proofI}[1][\proofname]{%
	\begin{proof}[#1]$ $\par\nobreak\ignorespaces
		}{%
	\end{proof}
}

\newcommand{\aomega}{A^{\omega}}
\newcommand{\ainf}{A^{\infty}}


\newcommand{\lomega}{L^{\omega}}
\newcommand{\linf}{L^{\infty}}

\newcommand{\larr}[1]{\overrightarrow{#1}}


\newcommand{\Bord}[1] {\kl [bord]{\text{Bord}\left( #1 \right)}}

\newcommand{\quotes}[1] {''#1''}

\newcommand{\Table}[1][]{(S, E, T_{#1})}

\newcommand{\cheminArrow}[1]{\kl[cheminB]{\xrightarrow{#1}}}
\newcommand{\cheminArrowT}[1]{\kl[cheminB]{\xrightarrow[t]{#1}}}
\newcommand{\chemin}[3]{#1 \cheminArrow {#3}{ } #2}
\newcommand{\cheminT}[3]{#1 \cheminArrowT {#3} #2}


\newcommand{\cheminD}{\kl[chemin]{\to}}

\newcommand{\compl}[1]{\mathscr{#1}}

\newcommand{\todo}[1]{\red{TODO\ifx#1~ \else: #1\fi}}


\knowledgenewrobustcmd \congN {\cmdkl {\cong}}
